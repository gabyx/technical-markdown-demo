\begin{longtable}[]{@{}
  >{\raggedright\arraybackslash}p{(\columnwidth - 2\tabcolsep) * \real{0.4884}}
  >{\raggedright\arraybackslash}p{(\columnwidth - 2\tabcolsep) * \real{0.5116}}@{}}
\caption{Unterschiede von Integration und
Inklusion}\tabularnewline
\toprule
\textbf{Integration}
 & 
\textbf{Inklusion}
 \\
\midrule
\endhead
Eingliederung von Kindern mit Behinderungen in ein
bestehendes System & Gemeinsames Leben und Lernen aller
Kinder \\ \midrule

Differenziertes System je nach Schädigung & Umfassendes
System für alle \\ \midrule

Zwei-Gruppen-Theorie: Unterscheidung zwischen
\begin{itemize}
\item
  behindert / nicht behindert
\item
  Integrations- und Regelkindern
\item
  Kinder mit und ohne besonderem Förderbedarf
\end{itemize}
 & Theorie der heterogenen Gruppe: Jeder
Mensch ist anders, hat Kompetenzen und Schwächen. Es gibt
viele Minderheiten und Mehrheiten. Eine Zugehörigkeit ist
nicht abhängig von bestimmten individuellen Merkmalen,
sondern ist selbstverständlich \\ \midrule

Finanzielle und personelle Ressourcen für Kinder mit
Etikettierung:
Kinder werden erst ausgesondert und als ``von der Norm
abweichend'' gekennzeichnet, um dann wieder eingegliedert zu
werden & Ressourcen für Systeme:
Kitas und Kindergruppen werden mit Ressourcen gefördert.
Eine Etikettierung und Ausgrenzung einzelner Kinder ist
nicht notwendig \\ \midrule

Gesonderte Förderpläne und spezielle Förderung für Kinder
mit Behinderungen & Ein Curriculum für alle Kinder:
Gemeinsames und individuelles Lernen unter Einsatz von
Binnendifferenzierung \\ \midrule

Anliegen und Auftrag der Sonder- und Heilpädagogik und
spezieller Fachkräfte & Anliegen und Auftrag der
Frühpädagogik und aller Fachkräfte \\ \midrule

Integrationsfachkräfte als Unterstützung für Kinder mit
sonderpädagogischem Förderbedarf & Inklusionsfachkräfte als
Unterstützung für Erzieher*innen, Kindergruppen und das
ganze System \\ \midrule

Kontrolle durch Fachpersonen & Kollegiales Problemlösen im
Team \\
\bottomrule
\end{longtable}
