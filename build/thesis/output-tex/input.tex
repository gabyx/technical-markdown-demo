% Options for packages loaded elsewhere
\PassOptionsToPackage{unicode,linktoc=all,hidelinks}{hyperref}
\PassOptionsToPackage{hyphens}{url}
\PassOptionsToPackage{dvipsnames,svgnames,x11names}{xcolor}
%
\documentclass[
  ngerman,
  11pt,
  paper=a4,
  twoside,
  titlepage=true,
  openright,
  abstract=on,
  toc=listofnumbered,
  numbers=noenddot,
  chapterprefix=true,
  headings=optiontohead,
  svgnames,
  dvipsnames]{scrreprt}
\usepackage{amsmath,amssymb}
\usepackage{iftex}
\ifPDFTeX
  \usepackage[T1]{fontenc}
  \usepackage[utf8]{inputenc}
  \usepackage{textcomp} % provide euro and other symbols
\else % if luatex or xetex
  \usepackage{unicode-math}
  \defaultfontfeatures{Scale=MatchLowercase}
  \defaultfontfeatures[\rmfamily]{Ligatures=TeX,Scale=1}
  \setmainfont[]{Latin Modern Roman}
  \setsansfont[]{Latin Modern Sans}
  \setmonofont[]{Latin Modern Mono}
  \setmathfont[]{Latin Modern Math}
\fi
% Use upquote if available, for straight quotes in verbatim environments
\IfFileExists{upquote.sty}{\usepackage{upquote}}{}
\IfFileExists{microtype.sty}{% use microtype if available
  \usepackage[]{microtype}
  \UseMicrotypeSet[protrusion]{basicmath} % disable protrusion for tt fonts
}{}
\makeatletter
\@ifundefined{KOMAClassName}{% if non-KOMA class
  \IfFileExists{parskip.sty}{%
    \usepackage{parskip}
  }{% else
    \setlength{\parindent}{0pt}
    \setlength{\parskip}{6pt plus 2pt minus 1pt}}
}{% if KOMA class
  \KOMAoptions{parskip=half}}
\makeatother
\usepackage{xcolor}
\usepackage{longtable,booktabs,array}
\usepackage{calc} % for calculating minipage widths
% Correct order of tables after \paragraph or \subparagraph
\usepackage{etoolbox}
\makeatletter
\patchcmd\longtable{\par}{\if@noskipsec\mbox{}\fi\par}{}{}
\makeatother
% Allow footnotes in longtable head/foot
\IfFileExists{footnotehyper.sty}{\usepackage{footnotehyper}}{\usepackage{footnote}}
\makesavenoteenv{longtable}
\setlength{\emergencystretch}{3em} % prevent overfull lines
\providecommand{\tightlist}{%
  \setlength{\itemsep}{0pt}\setlength{\parskip}{0pt}}
\setcounter{secnumdepth}{3}
\newlength{\cslhangindent}
\setlength{\cslhangindent}{1.5em}
\newlength{\csllabelwidth}
\setlength{\csllabelwidth}{3em}
\newlength{\cslentryspacingunit} % times entry-spacing
\setlength{\cslentryspacingunit}{\parskip}
\newenvironment{CSLReferences}[2] % #1 hanging-ident, #2 entry spacing
 {% don't indent paragraphs
  \setlength{\parindent}{0pt}
  % turn on hanging indent if param 1 is 1
  \ifodd #1
  \let\oldpar\par
  \def\par{\hangindent=\cslhangindent\oldpar}
  \fi
  % set entry spacing
  \setlength{\parskip}{#2\cslentryspacingunit}
 }%
 {}
\usepackage{calc}
\newcommand{\CSLBlock}[1]{#1\hfill\break}
\newcommand{\CSLLeftMargin}[1]{\parbox[t]{\csllabelwidth}{#1}}
\newcommand{\CSLRightInline}[1]{\parbox[t]{\linewidth - \csllabelwidth}{#1}\break}
\newcommand{\CSLIndent}[1]{\hspace{\cslhangindent}#1}
\ifLuaTeX
\usepackage[bidi=basic]{babel}
\else
\usepackage[bidi=default]{babel}
\fi
\babelprovide[main,import]{ngerman}
% get rid of language-specific shorthands (see #6817):
\let\LanguageShortHands\languageshorthands
\def\languageshorthands#1{}
\ifLuaTeX
  \usepackage{selnolig}  % disable illegal ligatures
\fi
\IfFileExists{bookmark.sty}{\usepackage{bookmark}}{\usepackage{hyperref}}
\IfFileExists{xurl.sty}{\usepackage{xurl}}{} % add URL line breaks if available
\urlstyle{same} % disable monospaced font for URLs
\hypersetup{
  pdftitle={Mit KITAplus einen Schritt Richtung Inklusion},
  pdfauthor={I. N.},
  pdflang={de-DE},
  colorlinks=true,
  linkcolor={DarkBlue},
  filecolor={DarkBlue},
  citecolor={DarkBlue},
  urlcolor={MediumBlue},
  pdfcreator={LaTeX via pandoc}}

% Set include paths
\makeatletter
\providecommand*{\input@path}{}
\edef\input@path{{/Users/isabellenussli/Desktop/HFH/Praxisprojekt/HFHWork/tools/convert/includes/}{/Users/isabellenussli/Desktop/HFH/Praxisprojekt/HFHWork/src/thesis/}{./}\input@path}% prepend
\makeatother

% Calculations =======================================
\usepackage{etoolbox}
\usepackage{calc}
% ====================================================

%% Geometry ==========================================
\newlength{\innermargin}
\setlength{\innermargin}{2.15cm}
\usepackage[nomarginpar,
            textwidth=16.0cm,
            textheight=252mm,
            bottom=2.0cm,
            headheight=15pt,
            headsep=20pt,
            footskip=20pt,
            bindingoffset=7mm,
            inner=\innermargin,
            twoside
            ]{geometry}
% ====================================================

%% Grafics ===========================================
\usepackage{graphicx}
\usepackage{tikz}
\usepackage[export]{adjustbox}
\usepackage[final]{pdfpages}
\usepackage{wrapfig}
\usepackage[section]{placeins} % float barrier at
                               % each section
\usepackage{svg}
\svgsetup{extractpath=files/generated/svg-extract, inkscapepath=files/generated/svg-inkscape}
% ====================================================

%% Linespacing =======================================
\usepackage[nodisplayskipstretch]{setspace}
\onehalfspacing
% ====================================================

%% Caption ===========================================
\usepackage[%
    format=plain,%
    justification=justified,%
    margin=3mm,%
    font={normal,small},%
    labelfont=it%
]{caption}

\DeclareCaptionStyle{figcaption}{
    justification=centering,
    font={normal,small},
    labelfont=it,
    aboveskip=5pt,
}

\DeclareCaptionStyle{tablecaption}{
    justification=raggedright,
    font={normal,small},
    labelfont=it,
    belowskip=3pt%
}

\usepackage[]{subcaption}
\captionsetup[figure]{style=figcaption}
\captionsetup[table]{style=tablecaption}
\captionsetup[longtable]{style=tablecaption}
% ====================================================

%% Header/Footer =====================================
\usepackage[automark,headsepline]{scrlayer-scrpage}
\clearpairofpagestyles
\setkomafont{pagehead}{\normalfont\rmfamily\bfseries}
%\setkomafont{pagenumber}{\normalfont\rmfamily}

\automark[chapter]{chapter}
\renewcommand{\chaptermarkformat}{}
\renewcommand{\sectionmarkformat}{\thesection\autodot\enskip}
\cfoot[\pagemark]{\pagemark}
\lehead{Kapitel \thechapter}
\rohead{\headmark}
\rehead{}
\lohead{}

% Footnotes
\usepackage[perpage,para]{footmisc}
% ====================================================

%% Enumitems =========================================
\usepackage{enumitem}
\setlist[description,1]{leftmargin=0.5cm,font=\rmfamily}

%% Tables ============================================
\usepackage{array}
\usepackage{longtable}
\usepackage{booktabs}
\usepackage{collcell}
\usepackage{makecell}
\usepackage{footnotehyper}
\usepackage{tablefootnote}
\AtBeginEnvironment{longtable}{%
\setlist[itemize]{nosep=0pt,
                 leftmargin=*,
                 label=\textbullet,
                 after=\end{minipage},
                 before=\begin{minipage}[t]{\linewidth}
}
}

% Define default table spacing...
\usepackage{cellspace}
\setlength{\cellspacetoplimit}{0.2\baselineskip}
\setlength{\cellspacebottomlimit}{0.2\baselineskip}
% ====================================================

%% Quotes =============================================
\usepackage{relsize}% http://ctan.org/pkg/{relsize,etoolbox}
\AtBeginEnvironment{quote}{\smaller}
\usepackage[font=itshape,vskip=5pt]{quoting}

%% Todo Notes =========================================
% \usepackage[textsize=tiny,disable]{todonotes}
% \newcommand{\disscomment}[1]{%
%     \todo[color=black!40]{#1}%
% }%

%% Glossary ==========================================
% \usepackage[hyperfirst=true]{glossaries}
% \glstoctrue
% \renewcommand*{\glstextformat}[1]{\textnormal{#1}}
% \newcommand{\glsp}{\protect\gls}
% \loadglsentries{chapter/GlossaryEntries}
%\makeglossaries
% ====================================================

%% Own styles! ========================================
\ProvidesPackage{GeneralMacros}[general style definitions]
\RequirePackage{graphicx}
\RequirePackage{xparse}
\RequirePackage{svg}

% Image with caption
\NewDocumentCommand\imageWithCaption{mmmo}{
\begin{figure}[h]
    \centering
    \adjustbox{center}{\adjustbox{cfbox=white!90!black 2pt 0pt}{\includegraphics[#3]{#1}}}
    \caption{#2}
    \IfValueT{#4}{\label{#4}}
\end{figure}
}

% SVG with caption
\NewDocumentCommand\svgWithCaption{mmmo}{
\begin{figure}[h]
    \centering
    \adjustbox{center}{\adjustbox{cfbox=white!90!black 2pt 0pt}{\includesvg[inkscapearea=page,#3]{#1}}}
    \caption{#2}
    \IfValueT{#4}{\label{#4}}
\end{figure}
}

\newfontfamily\DejaSans{DejaVu Sans}
\NewDocumentCommand\emojiFont{m}{
{\DejaSans #1}%
}

\NewDocumentCommand{\includePDF}{m O{1} m O{}}
{%
    \foreach \x in {#2,...,#3} {%
        \begin{center}%
        \makebox[\textwidth]{\includegraphics[page=\x, #4]{#1}}%
        \end{center}%
        \clearpage%
    }
}

% Ruler commands 
% https://tex.stackexchange.com/a/309705/7083
% \xhrulefill[height=-2pt,thickness=1pt,fill=3cm]
\ExplSyntaxOn
\NewDocumentCommand{\xhrulefill}{O{}}
 {
  \group_begin:
  \severin_xhrulefill:n { #1 }
  \group_end:
 }

 \NewDocumentCommand{\xhrule}{O{}}
 {
  \group_begin:
  \severin_xhrule:n { #1 }
  \group_end:
 }


\keys_define:nn { severin/xhrulefill }
 {
  height .dim_set:N    = \l_severin_xhrule_height_dim,
  thickness .dim_set:N = \l_severin_xhrule_thickness_dim,
  fill .skip_set:N     = \l_severin_xhrule_fill_skip,
  height .initial:n    = 0pt,
  thickness .initial:n = 0.4pt,
  fill .initial:n      = 0pt plus 1fill,
 }

\cs_new_protected:Nn \severin_xhrule:n
 {
    \keys_set:nn { severin/xhrulefill } { #1 }
    \rule[\l_severin_xhrule_height_dim]{\l_severin_xhrule_fill_skip}{\l_severin_xhrule_thickness_dim}
}

\cs_new_protected:Nn \severin_xhrulefill:n
 {
  \keys_set:nn { severin/xhrulefill } { #1 }
  \leavevmode
  \leaders\hrule 
    height \dim_eval:n { \l_severin_xhrule_thickness_dim + \l_severin_xhrule_height_dim }
    depth  \dim_eval:n { -\l_severin_xhrule_height_dim }
  \skip_horizontal:N \l_severin_xhrule_fill_skip
  \kern 0pt
}

\ExplSyntaxOff

\ProvidesPackage{MathMacros}[math style definitions]

\RequirePackage{tikz}
\usepackage[customcolors,markings]{hf-tikz}
\usetikzlibrary{calc,tikzmark}

% Left Subscripts
\RequirePackage{leftidx}

% Math
\RequirePackage{mathdots}

% Markdown converted definitions
\ProvidesPackage{MathMacros}[math style definitions]

\RequirePackage{tikz}
\usepackage[customcolors,markings]{hf-tikz}
\usetikzlibrary{calc,tikzmark}

% Left Subscripts
\RequirePackage{leftidx}

% Math
\RequirePackage{mathdots}

% Markdown converted definitions
\ProvidesPackage{MathMacros}[math style definitions]

\RequirePackage{tikz}
\usepackage[customcolors,markings]{hf-tikz}
\usetikzlibrary{calc,tikzmark}

% Left Subscripts
\RequirePackage{leftidx}

% Math
\RequirePackage{mathdots}

% Markdown converted definitions
\input{includes/generated/Math.tex}
% ====================================================

%% Custom Part Heading ===============================
\addtokomafont{part}{\Huge\selectfont\rmfamily\bfseries}
\addtokomafont{partprefix}{\Large\rmfamily\bfseries}
%\newcommand*\partcolor{\color{blue!50}}% Part is coloured blue
\renewcommand*\partheadstartvskip{\vspace*{.2\textheight}}
\renewcommand*\partheadmidvskip{%
    \par
    \vspace*{30pt}%
}
\makeatletter
\renewcommand*\partheadendvskip{%
    \vspace{\baselineskip}%
    \partquotenote%
    \vfil\newpage%
    \if@twoside%
        \if@openright%
        \null%
        \thispagestyle{plain}%
        \vspace*{\fill}%
        \partnote%
        \vspace*{\fill}%
        \fi%
    \else%
         \thispagestyle{plain}%
         \vspace*{\fill}%
         \partnote%
         \vspace*{\fill}%
         \newpage%
    \fi%
    \if@tempswa%
    \twocolumn%
    \fi%
}
\makeatother

\newcommand\partnote{}
\newcommand\partquotenote{}
\renewcommand\partformat{\hfill\color{lightgray}\partname~\fontsize{60}{60}\selectfont\thepart\if@altsecnumformat.\fi}
%=====================================================

%% Custom Chapter Heading ============================
\renewcommand*\chapterheadstartvskip{\vspace*{-0.5cm}}
\renewcommand*\chapterheadendvskip{%
    \noindent{\setlength{\parskip}{0pt}\hrulefill\par}%
    \vspace*{0.5\baselineskip}%
}
\renewcommand*{\chapterformat}{%
    \parbox{\textwidth}{\hfill\chapappifchapterprefix{\ }\fontsize{60}{60}\selectfont\thechapter\autodot\enskip}%
    \vspace{0ex}%
    }
\addtokomafont{disposition}{\normalfont\bfseries}
\addtokomafont{chapterprefix}{\Large\bfseries\color{lightgray}}
%=====================================================

%% Part Quotes =======================================
\setkomafont{dictumtext}{\itshape\small}
\setkomafont{dictumauthor}{\normalfont}
\renewcommand*\dictumwidth{0.9\linewidth}
\renewcommand*\raggeddictum{\centering}
\renewcommand*\dictumauthorformat[1]{--- #1}
\renewcommand*\dictumrule{
    %\vskip-1ex\hrulefill\par}
}
%=====================================================

%% Heading fonts =====================================
\setkomafont{chapter}{\huge\rmfamily}
\addtokomafont{section}{\rmfamily}
\addtokomafont{subsection}{\rmfamily}
\addtokomafont{subsubsection}{\rmfamily}
%=====================================================

%% Heading spacing ===================================
% \RedeclareSectionCommand[
%   runin=false,
%   afterindent=false,
%   beforeskip=1.0\baselineskip,
%   afterskip=0.3\baselineskip]{section}
% \RedeclareSectionCommand[
%   runin=false,
%   afterindent=false,
%   beforeskip=0.5\baselineskip,
%   afterskip=0.0\baselineskip]{subsection}
% \RedeclareSectionCommand[
%   runin=false,
%   afterindent=false,
%   beforeskip=0.5\baselineskip,
%   afterskip=0.0\baselineskip]{subsubsection}
%=====================================================

%% Table of Content ==================================
\usepackage{titling}
% \usepackage{titletoc}
% \newcommand{\setupTOC}[2]{%
%     \titlecontents{#1}
%     [8em] % ie, 1.5em (chapter) + 2.3em
%     {\rightskip=10mm plus 1fil\hyphenpenalty=10000\normalsize\bfseries\protect\addvspace{15pt}\contentsmargin{3em}}
%     {\contentslabel[{\color{gray}#2\enspace\thecontentslabel}]{8em}}
%     {\hspace*{-8em}}
%     {\hfill\contentspage}
% }
% \setupTOC{chapter}{\chaptername}
% =====================================================

% Layout Placing Settings ==============================
% \setcounter{topnumber}{2}
% \setcounter{bottomnumber}{2}
% \setcounter{totalnumber}{3}     % 2 may work better
% \setcounter{dbltopnumber}{2}    % for 2-column pages
\renewcommand{\topfraction}{0.85}
\renewcommand{\bottomfraction}{0.8}
\renewcommand{\textfraction}{0.07}
\renewcommand{\floatpagefraction}{0.8}
%\renewcommand{\dbltopfraction}{.66}
%\renewcommand{\dblfloatpagefraction}{.8}

% Allow display break
%\allowdisplaybreaks[3]
%=======================================================

% Tocless Command ======================================
\newcommand{\nocontentsline}[3]{}
\newcommand{\tocless}[2]{\bgroup\let\addcontentsline=\nocontentsline#1{#2}\egroup}

% Clever Refs ==========================================
\usepackage[capitalise]{cleveref} % load it after thmtools (there is some problem)
\crefname{appsec}{appendix}{appendices}
%=======================================================

% General Settings
\KOMAoptions{cleardoublepage=empty}
\KOMAoptions{parskip=half}
\raggedbottom

\title{Mit KITAplus einen Schritt Richtung Inklusion}
\usepackage{etoolbox}
\makeatletter
\providecommand{\subtitle}[1]{% add subtitle to \maketitle
  \apptocmd{\@title}{\par {\large #1 \par}}{}{}
}
\makeatother
\subtitle{Eine quantitative Studie zur Fremdbetreuung von Kindern mit
sonderpädagogischem Förderbedarf im KITAplus-Programm}
\author{I. N.}
\date{Juni 2022}

\begin{document}
%
%
\begin{titlepage}
\makeatletter
\vphantom{A}\vspace{0.5cm}
\begin{center}
\includesvg[inkscapearea=page,width=0.5\textwidth]{files/HFHLogo.svg}
\end{center}
\vspace{2cm}
\begin{center}
\fontsize{22pt}{24pt}\selectfont\textbf{\@title}
\end{center}
\begin{center}
\fontsize{16pt}{18pt}\selectfont
Eine quantitative Studie zur Fremdbetreuung von Kindern mit
sonderpädagogischem Förderbedarf im KITAplus-Programm\\
\end{center}
\begin{center}
Masterarbeit im Studiengang Sonderpädagogik\\ mit Vertiefungsrichtung
Heilpädagogische Früherziehung
\end{center}
\begin{center}
eingereicht von
\end{center}
\begin{center}
\textbf{\@author}
\end{center}
\begin{center}
im \@date\\
\vspace{0.5cm}
Begleitung: M. L.\\
\vfill
Luzern, 2022
\end{center}
\makeatother
% \tikz [remember picture, overlay] %
% \node [shift={(1cm,-1cm)}] at (current page.north west) %
% [anchor=north west] %
% {\includesvg[inkscapearea=page,inkscapepath=svgsubdir]{files/HFHLogo.svg}};
\end{titlepage}%
\cleardoublepage
\pagenumbering{roman}
\pagestyle{scrheadings}
\setcounter{page}{1}
%
\newcommand{\abstracttitle}{Abstract}
\chapter*{\abstracttitle}
\thispagestyle{plain}
Die Schweiz hat laut Fischer, Häfliger und Pestalozzi (2021) viel in die
familienergänzende Betreuung im Vorschulalter investiert. Die Auswahl
ist gross, jedoch sind Eltern von Kindern mit Beeinträchtigung in der
Suche eines Fremdbetreuungsplatzes benachteiligt. Das Angebot ist
kantonsabhängig, z.T. nicht vorhanden oder extrem teuer. Die Studie
widmet sich folgender Frage: «Welche Kinder mit sonderpädagogischem
Förderbedarf nehmen im Kanton Luzern das KITAplus-Programm in Anspruch?»
Die quantitative Erhebung bei 37 Fachpersonen der Heilpädagogischen
Früherziehung / KITAplus-Mitarbeitenden zeigt eine enorme Heterogenität
der KITAplus-Kindergruppe, sowie Anliegen der Kindertagesstätten. Alle
Kinder weisen einen erhöhten Unterstützungsbedarf auf. Es werden
vorwiegend Kinder mit mittlerer und starker Beeinträchtigung integriert.
Das Hauptanliegen der Kitas gegenüber dem beantragten Coaching kann mit
dem Begriff «soziale Teilhabe» zusammengefasst werden.
\cleardoublepage%
\newcommand{\acknowledgementtitle}{Danksagung}
\thispagestyle{plain}
\chapter*{\acknowledgementtitle}
Lorem ipsum dolor sit amet, consectetur adipiscing elit. Morbi ultricies
aliquet facilisis. Aliquam sit amet nisl suscipit, mattis metus ac,
molestie lorem. Fusce erat nisi, bibendum eu nibh eget, rhoncus ultrices
velit. Aenean posuere est lectus, in porta ipsum euismod eget. Integer
efficitur diam eget libero ornare commodo. Vivamus ultrices, lectus
vulputate condimentum fermentum, purus nunc sodales lacus, sit amet
maximus neque ante vitae turpis. Ut ultrices, eros tincidunt dictum
condimentum, leo felis rhoncus lectus, ac imperdiet lectus sem et neque.

Mauris egestas ex at ipsum consequat egestas. Nullam sit amet odio ante.
Suspendisse molestie scelerisque iaculis. Vestibulum et suscipit velit.
Donec nulla ante, luctus vestibulum dui tincidunt, dapibus vulputate
nibh. Fusce nec magna quis turpis dignissim vehicula. Praesent a libero
metus. Morbi ullamcorper mi ut nulla euismod, a dapibus sem vulputate.
Etiam sagittis lectus in elit dignissim, sit amet pharetra neque
ultrices.

Mauris id posuere lacus. Nunc efficitur ac risus lobortis dignissim.
Etiam maximus orci nec magna dapibus malesuada. Maecenas eget venenatis
nibh. Praesent vestibulum risus pellentesque purus elementum ornare. In
aliquet dapibus ante, et venenatis velit cursus at. Ut non auctor dui.
Praesent porta nec lectus in auctor. Aliquam nec tellus non libero
lobortis posuere eget in velit. Duis aliquam tristique risus semper
luctus. Nulla accumsan urna massa, ac condimentum velit gravida eu.
Quisque tellus libero, fermentum sed velit non, consequat blandit massa.
\cleardoublepage%
{
\hypersetup{linkcolor=Black}
\tableofcontents
\cleardoublepage
}
%
\cleardoublepage
\pagenumbering{arabic}
\newcommand{\adjustTableFormat}{
\begingroup\singlespacing\fontsize{10pt}{11pt}\selectfont\setlength{\LTpre}{0pt}\setlength{\LTpost}{6pt}
}
\newcommand{\adjustTableFormatEnd}{
\endgroup
}

\hypertarget{sec:hinfuxfchrungzurarbeit}{%
\chapter{Hinführung}\label{sec:hinfuxfchrungzurarbeit}}

Die Thematik rund um institutionelle Fremdbetreuung ist hochaktuell. Ein
Grund dafür könnte der Wandel des traditionellen Familienbildes sein.
Die Forderung von Gleichstellung von Mann und Frau hat in den letzten
Jahren zugenommen. Die Vereinbarkeit von Beruf und Familie rückt dabei
ins Zentrum. Laut Bundesamt für Statistik (BFS,
\protect\hyperlink{ref-bfs}{2020}) ist die Zahl der berufstätigen Frauen
gestiegen. Damit gewinnt die familien- und schulergänzende
Kinderbetreuung an Stellenwert. Es wird dabei zwischen institutioneller
und nicht institutioneller Betreuung unterschieden.

Laut Fischer, Häfliger \& Pestalozzi
(\protect\hyperlink{ref-angebotsmangel}{2021}) besuchen heute über 116
000 Kinder eine Kindertagesstätte. Den Eltern steht ein breites Angebot
vor Ort zur Verfügung. Wie bereits erwähnt, gestaltet sich die Suche für
Eltern von Kindern mit Beeinträchtigung schwieriger. Es fehlen
entsprechende Fremdbetreuungsplätze, wie auch Finanzierungsmechanismen,
welche die behinderungsbedingten Mehrkosten regeln und decken. Diese
Diskriminierung gilt es auszugleichen. Nach einer Hochrechnung von
Fischer et al. (\protect\hyperlink{ref-angebotsmangel}{2021}) wird von
ca. 3000 Kindern mit Behinderungen ausgegangen, welche ein
Fremdbetreuungsangebot beanspruchen würden, sofern der Zugang ermöglicht
wird. Die Schweiz hat im Jahr 2014 der Ratifizierung der
UNO-Behindertenrechtskonvention zugestimmt und sich dazu verpflichtet “…
Hindernisse zu beheben, mit denen Menschen mit Behinderungen
konfrontiert sind, sie gegen Diskriminierungen zu schützen und ihre
Inklusion und ihre Gleichstellung in der Gesellschaft zu fördern”
(\protect\hyperlink{ref-uno}{Schweizerische Eidgenossenschaft}).
Demzufolge haben Kinder mit speziellen Bedürfnisse ein Recht auf
Bildung, Betreuung, sowie Erziehung und dies von Geburt an
(\protect\hyperlink{ref-stamm}{Stamm, 2009}). In der Schweiz zählt der
Frühbereich (Kinder von null bis vier Jahren) nicht zum staatlichen
Bildungssystem. Es besteht daher kein Rechtsanspruch auf einen
Betreuungsplatz und wird nicht unentgeltlich angeboten wie zum Beispiel
der Kindergarten (ebd.).

Im Rahmen dieser Arbeit soll die Inklusionsthematik in
Kindertagesstätten (Kita) aufgearbeitet und erläutert werden. Im Zentrum
stehen Kinder mit sonderpädagogischem Förderbedarf, welche das
KITAplus-Programm in Luzern in Anspruch nehmen.

\hypertarget{sec:fragestellung}{%
\section{Fragestellungen}\label{sec:fragestellung}}

Folgende drei Fragestellungen stehen im Zentrum und sollen mithilfe der
Forschungsstudie beantwortet werden.

\begin{enumerate}
\def\labelenumi{\arabic{enumi}.}
\item
  Welche Kinder mit sonderpädagogischem Förderbedarf nehmen im Kanton
  Luzern das KITAplus-Programm in Anspruch?
\item
  Aufgrund welcher Anliegen wird das KITAplus im Kanton Luzern
  beantragt?
\item
  Lassen sich Prädikatoren für eine Beantragung von KITAplus
  ausarbeiten?
\end{enumerate}

Das Forschungsprojekt umfasst zwei standardisierte und anonymisierte
Fragebögen, welche pro Kind ausgefüllt werden. Heilpädagogische
Fachpersonen (\cref{sec:erhebungsfragebogenallgemein}) und
KITAplus-Mitarbeitende (\cref{sec:erhebungsfragebogen}) bearbeiten je
einen Fragebogen. Damit werden «Persönliche Informationen» und «Angaben
zur Fremdbetreuungssituation» der Kinder zusammengetragen. Im Anschluss
werden die Fragebögen ausgewertet und mit einem explorativen und
deskriptiven Blick beschrieben (Stichprobe, \cref{sec:stichprobe} und
Ergebnisse, \cref{sec:evaluation}).

\hypertarget{sec:aufbau}{%
\section{Aufbau der Arbeit}\label{sec:aufbau}}

In \cref{sec:grundlagen} steht die Aufarbeitung von Literatur rund um
die Thematik Inklusion im Vordergrund. Es werden unter anderem die
Begriffe Integration und Inklusion erläutert und einander
gegenübergestellt. Zudem werden Gelingensbedingungen beschrieben, sowie
kursierende Vorstellungen über Inklusion aufgegriffen. Des weiteren wird
der Schlüsselfaktor “Haltung” von pädagogischen Fachpersonen in der
Thematik erläutert, bevor die Integrationschance von Kindern mit
speziellen Bedürfnissen seitens der Kita-Mitarbeitenden eingeschätzt
wird. Zudem werden Vorstellungen von Peers gegenüber Kindern mit
speziellen Bedürfnissen aufgegriffen. Der Begriff “soziale Teilhabe”
wird erläutert und in Zusammenhang mit Inklusion gestellt. Danach werden
Vorteile und Risiken von Inklusion aufgezählt. Im zweitletzten Kapitel
wird auf die Frage eingegangen, welche Kinder mit sonderpädagogischem
Förderbedarf integrativ / inklusiv in Kindertagesstätten betreut werden.
Der Abschluss des Kapitels bildet die Erläuterung des KITAplus-Programm
im Kanton Luzern.

In \cref{sec:forschungsmethode} steht das Forschungsmethodische Vorgehen
im Vordergrund. Darin sind Forschungsansatz, Erhebungsinstrument und
Ablauf der Erhebung geschildert. Zudem wird die verwendete Methode der
Datenaufbereitung und Datenanalyse beschrieben. Der Abschluss des
Kapitels bildet die Stichprobe (\cref{sec:stichprobe}), in welcher erste
Ergebnisse der Erhebung präsentiert sind. Das Kapitel Ergebnisse folgt
(\cref{sec:evaluation}). Darin findet die Gegenüberstellung ausgewählter
Daten und die Beantwortung der Fragestellungen statt. Danach folgt die
Diskussion (\cref{sec:diskussion}), in welcher ausgewählte Ergebnisse
interpretiert werden. Die Arbeit schliesst mit dem Ausblick
(\cref{sec:fazit}) ab.

\hypertarget{sec:grundlagen}{%
\chapter{Grundlagen}\label{sec:grundlagen}}

In \cref{sec:hinfuxfchrungzurarbeit} sind rechtliche Aspekte kurz
genannt worden. Die Partizipation und Teilhabe von Menschen mit
besonderen Bedürfnissen rücken dabei in den Vordergrund. Diese gilt es
in allen Lebensbereichen wie zum Beispiel der Bildung, dem Arbeitsmarkt
oder in der Benützung öffentlicher Verkehrsmittel zu berücksichtigen und
voranzutreiben.

Dabei fallen Wörter wie Integration und Inklusion. Folgende Fragen
stellen sich in diesem Diskurs:

Ist Integration gleichzusetzen mit Inklusion? Welche Unterschiede
zwischen Inklusion und Integration lassen sich formulieren? Wie sieht
die Forschungsgrundlage in der frühen Bildung dazu aus? Gibt es
Gelingensfaktoren, welche der Integration von Kindern mit besonderen
Bedürfnissen in Kindertagesstätten begünstigen?

Die aufgezählten Fragen sind wegleitend für die Aufarbeitung der Theorie
in den nachfolgenden Abschnitten. Als Erstes wird der Begriff Inklusion
mithilfe von Theorie erörtert und anschliessend dem Integrationsbegriff
gegenübergestellt. Zuvor wird jedoch der Begriff “Behinderung”
erläutert.

\hypertarget{begriff-behinderung}{%
\section{Begriff “Behinderung”}\label{begriff-behinderung}}

In dieser Arbeit trifft man auf Bezeichnungen wie “Kinder mit
Behinderung, Kinder mit Beeinträchtigung, Kinder mit speziellen
Bedürfnissen, Kinder mit Förder- oder Unterstützungsbedarf”. Damit sind
allgemein Kinder gemeint, welche in der Bewältigung des Alltags, in
ihrer Aktivität und Partizipation eingeschränkt und auf Unterstützung
angewiesen sind. Es ist selbstverständlich, dass jüngere Kinder vermehrt
auf Unterstützung der Eltern angewiesen sind. Nichtsdestotrotz gibt es
Kinder, welche einen erhöhten Unterstützungsbedarf benötigen und einen
sonderpädagogischen Förderbedarf aufweisen. In dieser Arbeit stehen
solche Kinder im Vordergrund.

Das ICF-Modell (International Classification of Functioning and
Disability) der WHO ist dabei wegleitend \footnote{Weiterführende
  Informationen zum ICF-Modell sind auf der folgenden Seite zu finden:\\
  \url{https://www.soziale-initiative.net/wp-content/uploads/2013/09/icf_endfassung-2005-10-01.pdf}}.

\begin{quote}
Behinderung wird von der WHO als mehrdimensionales Konstrukt verstanden,
zu dem neben Körperfunktionen und Strukturen auch die Dimension der
Aktivität und Partizipation gehören. Alle Dimensionen stehen in
Wechselwirkung miteinander sowie mit den personenbezogenen und
umweltbezogenen Kontextfaktoren. Behinderung wird nicht als Zustand
einer Person, sondern einer Situation verstanden.
(\protect\hyperlink{ref-albersmittendrin}{Albers, 2011, S. 31})
\end{quote}

\hypertarget{sec:integrationinklusion}{%
\section{Begriffserklärung Inklusion}\label{sec:integrationinklusion}}

In wissenschaftlichen Diskursen liefern sich Inklusion und Integration
eine regelrechte Schlacht. Wocken (\protect\hyperlink{ref-wocken}{2009,
S. 2}) benennt diese sogar als “babylonische Sprachverwirrung” und dies
zurecht. In Fachdiskussionen sind dazu unterschiedliche Positionen
vertreten. Die einen benützen die Begriffe als Synonyme und andere
plädieren auf eine klare Differenzierung und Abstufung: “Inklusion ist
mehr und anders als Integration” (\protect\hyperlink{ref-wocken}{Wocken,
2009, S. 2}). Übrig bleibt ein unscharfer Begriff, welcher in aller
Munde ist. Bei jeder Diskussion müsste deshalb nachgefragt werden, was
der Einzelne unter Inklusion versteht, damit die Ausgangslage dieselbe
ist. Nachfolgend werden die beiden Begriffe voneinander differenziert.
Schattenmann (\protect\hyperlink{ref-schattenmann_inklusion_2014}{2014,
S. 26}) weist daraufhin, dass eine Annäherung an den Begriff Inklusion
nicht ohne Abgrenzung zu Integration erfolgen kann. Sie unternimmt fünf
Annäherungsversuche um Inklusion zu erläutern.

\hypertarget{sec:annuxe4herungsversuch}{%
\subsection{Fünf Annäherungsversuche zum
Inklusions-Begriff}\label{sec:annuxe4herungsversuch}}

Als Erstes verweist sie auf das Alltagsverständnis. Inklusion wird zum
Beispiel mit “Einschluss” oder “Dazugehören” assoziiert
(\protect\hyperlink{ref-schattenmann_inklusion_2014}{Schattenmann, 2014,
S. 26}). Die zweite Annäherung, die sprachgeschichtliche, weist in
dieselbe Richtung. Der Begriff Inklusion lässt sich vom
mittellateinischen Substantiv “inclusio” ableiten. Dieses kann mit
“Einschliessung” oder wie bereits genannt mit “Einschluss” im Sinne von
Zugehörigkeit interpretiert werden.

In der dritten Annäherung wird die historische Entwicklung hinzugezogen.
Schattenmann (\protect\hyperlink{ref-schattenmann_inklusion_2014}{2014})
verweist dabei auf das fünfstufige Modell (\cref{fig:extinktion}, von
Extinktion zu Inklusion). Wocken (\protect\hyperlink{ref-wocken}{2009})
weist jedoch explizit daraufhin, dass dieses fünfstufige Modell nicht
als strenge Abfolge von epochalen Etappen angesehen werden soll. Die
menschliche Geschichte verlaufe selten linear. Ein Beispiel von Menschen
mit Behinderung verdeutliche dies. Die Tötung behinderter Menschen fand
bereits im Mittelalter statt und erfuhr leider in der Zeit des
Nationalsozialismus neuen Aufschwung. Wocken
(\protect\hyperlink{ref-wocken}{2009, S. 12}) plädiert auf die
Umbenennung von “Stufenmodell” hin zu „Qualitätsstufen der
Behindertenpolitik und -pädagogik”. Diese Umbenennung ist zu begrüssen,
da sie auf die Abstufung und Differenzierung von Integration und
Inklusion aufmerksam macht. Sie schafft ausserdem ein klares Verhältnis
zwischen den zwei Begriffen.

\svgWithCaption{files/Extinktion.svg}{Von Extinktion zu Inklusion in
Saalfrank \& Zierer
(\protect\hyperlink{ref-saalfrank_inklusion_2017}{2017, S. 36}) (im
Anschluss an Aehnelt, o.J.)}{width=0.6\textwidth}[fig:extinktion]

Für ihn stellt Integration ein “Antragsrecht” dar, welchem nachgekommen
werden kann oder nicht (\protect\hyperlink{ref-wocken}{Wocken, 2009, S.
14}). In der Realität stösst man an Grenzen in der Integration. Dabei
spielen vorhandene “Ressourcen” oder die zugeschriebene
“Integrationsfähigkeit” eines Kindes mit Beeinträchtigung eine Rolle.
Auf der Stufe der Inklusion ist jedes Kind “integrationsfähig”, sie
müssen keine Vorbedingungen erfüllen. Vielmehr muss die Umgebung
“integrationsfähig” gestaltet werden (ebd.). Eine klare Abstufung der
beiden Begriffe wird im Stufenmodell ersichtlich. Die fünfte Stufe
“Inklusion” kann dabei als höchste anzustrebende Qualitätsstufe
interpretiert werden.

Einen Aufschwung erlebte Inklusion dank der Verabschiedung der
UN-Behindertenrechtskonvention im Jahr 2006 in New York. Zwei Jahre
später, im Jahr 2008, tritt sie in Kraft. Die Schweiz hat das Abkommen
am 15. Mai 2014 ratifiziert. Das Übereinkommen fordert allgemeine
Menschenrechte für Menschen mit Behinderungen ein
(\protect\hyperlink{ref-uno}{Schweizerische Eidgenossenschaft}).

Nach Wocken (\protect\hyperlink{ref-wocken}{2009}) drängt die Bewegung
auf die Auflösung der Etikettierung von Menschen mit einer
Beeinträchtigung und fordert stattdessen die Anerkennung der Vielfalt
als Normalzustand ein.

Schattenmann (\protect\hyperlink{ref-schattenmann_inklusion_2014}{2014})
nennt als vierte Annäherung zu Inklusion die Abgrenzung zu Integration.
Es wird aufgezeigt, was Inklusion nicht ist. Dazu wird in
\cref{sec:inklusionversusintegration} (Inklusion versus Integration)
ausführlich Stellung genommen. Die zentralen Punkte der jeweiligen
Ideologie werden in Tabellenform festgehalten und einander
gegenübergestellt.

Als fünfte und letzte Annäherung erwähnt sie die theoretische Grundlage
am Beispiel der Luhmannschen Systemtheorie. In diesem theoretischen
Konzept wird das Begriffspaar “Inklusion und Exklusion” zur Beschreibung
von sozialer Ungleichheit hinzugezogen. Schattenmann
(\protect\hyperlink{ref-schattenmann_inklusion_2014}{2014, S. 41})
verweist für weitere Informationen auf Farzin (2006).

Laut Saalfrank \& Zierer
(\protect\hyperlink{ref-saalfrank_inklusion_2017}{2017}) kann und muss
Inklusion sehr breit gefasst werden, da sie den Anspruch hat gerechte
Verhältnisse weltweit zu schaffen.

Heimlich (\protect\hyperlink{ref-inklusionQualituxe4t_Heimlich}{2015})
nennt Folgendes bezüglich Integration und Inklusion:

\begin{quote}
Inklusion beinhaltet im Unterschied zu Integration ein erweitertes
Verständnis von selbstbestimmter sozialer Teilhabe, in dem von
vornherein auf Situationen und Institutionen der Aussonderung verzichtet
wird, die Unterschiedlichkeit der Mitglieder eines Gemeinwesens
(Heterogenität) als Bereicherung für alle betrachtet wird und alle die
gleiche Möglichkeit haben, an diesem Gemeinwesen zu partizipieren und zu
diesem Gemeinwesen beizutragen.
(\protect\hyperlink{ref-inklusionQualituxe4t_Heimlich}{Heimlich, 2015,
S. 29})
\end{quote}

Demzufolge sollen alle am Unterricht teilnehmen können. Ein gemeinsames
Lernen wird angestrebt, in einer Umgebung, die sich den Kindern anpasst.
Saalfrank \& Zierer
(\protect\hyperlink{ref-saalfrank_inklusion_2017}{2017}) unterscheiden
auf Inklusion bezogen fünf Gruppen. Diese sind in \cref{fig:dimensionen}
dargestellt. Die Grafik verdeutlicht, dass es nicht nur Kinder mit einer
Beeinträchtigung zu berücksichtigen gilt, sondern der Blickwinkel
breiter gefasst werden muss. Diese Grafik hilft, die Vielfalt der
anzutreffenden Kindern fassbarer zu machen und den Fokus auf Kinder mit
besonderen Bedürfnissen zu entschärfen.

\svgWithCaption{files/Dimension.svg}{Dimensionen der Inklusion von
Saalfrank \& Zierer
(\protect\hyperlink{ref-saalfrank_inklusion_2017}{2017, S.
35})}{width=0.6\textwidth}[fig:dimensionen]

Die Erläuterungen um Inklusion und Integration könnten noch weiter
ausgeführt und diskutiert werden. Jedoch würde es den Rahmen dieser
Arbeit sprengen. Der Abschluss der Ausführungen wird mit der
tabellarischen Gegenüberstellung in
\cref{sec:inklusionversusintegration} vollzogen.

\hypertarget{sec:inklusionversusintegration}{%
\subsection{Inklusion versus
Integration}\label{sec:inklusionversusintegration}}

Die wichtigsten Grundsätze von Inklusion und Integration sind wie folgt
von Hinz (\protect\hyperlink{ref-hinz2002}{2002, S. 11}) und Hundegger
(\protect\hyperlink{ref-kindergartenheute2019}{2019, S. 5}) festgehalten
worden. Der Fokus wird dabei auf das Bildungssystem gelegt.

\begin{longtable}[]{@{}
  >{\raggedright\arraybackslash}p{(\columnwidth - 2\tabcolsep) * \real{0.4884}}
  >{\raggedright\arraybackslash}p{(\columnwidth - 2\tabcolsep) * \real{0.5116}}@{}}
\caption{Unterschiede von Integration und
Inklusion}\tabularnewline
\toprule
\textbf{Integration}
 & 
\textbf{Inklusion}
 \\
\midrule
\endhead
Eingliederung von Kindern mit Behinderungen in ein
bestehendes System & Gemeinsames Leben und Lernen aller
Kinder \\ \midrule

Differenziertes System je nach Schädigung & Umfassendes
System für alle \\ \midrule

Zwei-Gruppen-Theorie: Unterscheidung zwischen
\begin{itemize}
\item
  behindert / nicht behindert
\item
  Integrations- und Regelkindern
\item
  Kinder mit und ohne besonderem Förderbedarf
\end{itemize}
 & Theorie der heterogenen Gruppe: Jeder
Mensch ist anders, hat Kompetenzen und Schwächen. Es gibt
viele Minderheiten und Mehrheiten. Eine Zugehörigkeit ist
nicht abhängig von bestimmten individuellen Merkmalen,
sondern ist selbstverständlich \\ \midrule

Finanzielle und personelle Ressourcen für Kinder mit
Etikettierung:
Kinder werden erst ausgesondert und als ``von der Norm
abweichend'' gekennzeichnet, um dann wieder eingegliedert zu
werden & Ressourcen für Systeme:
Kitas und Kindergruppen werden mit Ressourcen gefördert.
Eine Etikettierung und Ausgrenzung einzelner Kinder ist
nicht notwendig \\ \midrule

Gesonderte Förderpläne und spezielle Förderung für Kinder
mit Behinderungen & Ein Curriculum für alle Kinder:
Gemeinsames und individuelles Lernen unter Einsatz von
Binnendifferenzierung \\ \midrule

Anliegen und Auftrag der Sonder- und Heilpädagogik und
spezieller Fachkräfte & Anliegen und Auftrag der
Frühpädagogik und aller Fachkräfte \\ \midrule

Integrationsfachkräfte als Unterstützung für Kinder mit
sonderpädagogischem Förderbedarf & Inklusionsfachkräfte als
Unterstützung für Erzieher*innen, Kindergruppen und das
ganze System \\ \midrule

Kontrolle durch Fachpersonen & Kollegiales Problemlösen im
Team \\
\bottomrule
\end{longtable}

Im Laufe der Arbeit wechseln sich die Begriffe Inklusion und Integration
miteinander ab. Dies, weil Begriffe der jeweiligen Fachliteratur
übernommen werden. Zu beachten gilt es ausserdem, dass in der englischen
Literatur oftmals der Inklusion’s Begriff verwendet wird. Grundsätzlich
ist die Forschungsarbeit von der Haltung geprägt, dass zurzeit grosse
Bemühungen und Bestrebungen Richtung Inklusion vorhanden sind, diese
jedoch selten in dessen Reinform umgesetzt werden. Der
Integrationsgedanke und die Umsetzung diesbezüglich sind vorherrschend.

Mit dem Anspruch und Diskussionen rund um die Thematik Inklusion wachsen
Erwartungen und Ansprüche, diese auch im Frühbereich umzusetzen.
Hinzukommt, dass die Nachfrage an familien- und schulergänzenden
Fremdbetreuungsangeboten steigt (siehe
\cref{sec:hinfuxfchrungzurarbeit}). Kindertagesstätten müssen sich
früher oder später mit der Thematik auseinandersetzen. Im Gegensatz zu
den Volksschulen in der Schweiz, sind sie nicht gezwungen einen
Bildungsauftrag zu verfolgen und zu erfüllen. Die Betreuung von Kindern
steht im Vordergrund. Demzufolge sollten die Hürden, Inklusion zu leben,
so klein sein wie in keinem anderen Bereich. Die Heterogenität und
Individualität der jungen Kinder sollten in diesem Setting Platz haben.
In \cref{sec:inklusionkita} wird auf anzutreffende Vorstellungen zu
Inklusion in Kindertagesstätten eingegangen. Zuvor wird in
\cref{sec:oekosystemischesmodel} das ergänzte Ökosystemische Modell
(nach Bronfenbrenner, 1981) durch die “European Agency for Special Needs
and Inclusive Education” erläutert. Es stellt alle Ebenen, welche
Einfluss auf die frühkindliche Inklusion nehmen, in einem Modell dar.

\hypertarget{sec:oekosystemischesmodel}{%
\section{Ökosystemisches Modell der frühen
Inklusion}\label{sec:oekosystemischesmodel}}

Laut Lanners (\protect\hyperlink{ref-lanners}{2018}) wurde in den
letzten Jahren im Bereich der schulischen Integration von Kindern mit
besonderem Bildungsbedarf grosse Fortschritte erzielt. Nichtsdestotrotz
bestehe nach wie vor Handlungsbedarf. Die European Agency for Special
Needs and Inclusive Education (EASNIE) lancierte 2015 das dreijährige
Projekt “Inklusive frühkindliche Bildung und Erziehung”. Dieses hat sich
zur Aufgabe gemacht, die wichtigsten Faktoren einer hochwertigen
inklusiven frühkindlichen Bildung und Erziehung für alle Kinder im
Vorschulalter (ab drei bis ca. sieben Jahre) zu untersuchen (EASNIE,
\protect\hyperlink{ref-europaischeagentur}{2017}). Ihre Erkenntnisse
sind im Internet zugänglich.\footnote{Die Dokumente und Berichte zum
  Projekt “Inklusive Frühkindliche Bildung” sind auf der Homepage in
  verschiedenen Sprachen verfügbar:
  https://www.european-agency.org/Deutsch/publications} Neben dem
erweiterten Ökosystemischen Modell gilt es den erarbeiteten
“Selbstreflexionsbogen” hervorzuheben.\footnote{Selbstreflexionsbogen
  für das Umfeld der inklusiven frühkindlichen Bildung und Erziehung
  (EASNIE, \protect\hyperlink{ref-web-selbstreflex}{2022}).} Er ist für
die Vorschule gedacht und bezieht sich auf die Befragung von Prozess-
Strukturbezogenen Faktoren (ebd.). Dieser kann zum Beispiel in einer
Kindertagesstätte eingesetzt werden, damit laufende Massnahmen bezüglich
Inklusion ausgewertet werden, Mitarbeitende auf die Thematik zu
sensibilisieren und Ziele oder Anpassungen innerhalb der Organisation
vorzunehmen. Im Anschluss wird einzig auf das Ökosystemische Modell
eingegangen.

Das Ökosystemische Modell der frühen Inklusion vereint alle gewonnenen
Aspekte aus dem dreijährigen Projekt. Nach EASNIE
(\protect\hyperlink{ref-europaischeagentur}{2017}) kann das Modell “…
als Rahmen für die Planung, Verbesserung, Beobachtung/Überwachung und
Evaluation der Qualität der IECE auf lokaler, regionaler und nationaler
Ebene dienen” (S. 9).

Zudem haben individuelle Zielsetzungen und Anforderungen Platz darin.
Das Modell (\cref{fig:oekosystemischesmodell}) umfasst fünf Dimensionen,
welche im Abschlussbericht (EASNIE,
\protect\hyperlink{ref-europaischeagentur}{2017, S. 12}) beschrieben
sind. Sie alle beeinflussen entweder direkt oder indirekt die aktive
Teilnahme eines Kind im integrativen Setting im Vorschulbereich
(\protect\hyperlink{ref-lanners}{Lanners, 2018}).

Der Kern des Modells bildet die erste Dimension. Sie enthält die drei
wichtigsten Ergebnisse der “Inklusiven Frühkindlichen Bildung und
Erziehung” welche beim Kind angestrebt werden. Diese sind “aktive
Teilnahme, Zugehörigkeit und Lernen” (EASNIE,
\protect\hyperlink{ref-europaischeagentur}{2017, S. 12}). Auf der
zweiten Dimension werden fünf Schlüsselprozesse beschrieben, in welche
das Kind direkt involviert ist und auf welcher Fördermöglichkeiten
definiert werden können. Die dritte Dimension beschreibt unterstützende
Strukturen in der Vorschule und die vierte Dimension Strukturen in der
Gesellschaft. Die Letzte weist auf förderliche Strukturen in regionaler
und nationaler Ebene hin.

\svgWithCaption{files/OekosystemischesModell.svg}{Ökosystemisches Modell
der frühkindlichen Inklusion (EASNIE,
\protect\hyperlink{ref-europaischeagentur}{2017, S.
10})}{width=1.0\textwidth}[fig:oekosystemischesmodell]

Das Ökosystematische Modell verdeutlicht Wechselwirkungen und
Überschneidungen auf lokalen, regionaler und nationaler Ebene (EASNIE,
\protect\hyperlink{ref-europaischeagentur}{2017}). Auf nationaler Ebene
wird zum Beispiel festgehalten, dass allen Kindern Zugang zu Bildungs-
und Betreuungsangeboten gewährt wird. Demnach muss auf der vierten
Dimension sichergestellt werden, dass entsprechende Fort- und
Weiterbildungen angeboten werden, damit qualifiziertes Personal
vorhanden ist. Auf der dritten Dimension ist die Kindergartenstätte
gefordert genügend ausgebildetes Personal für die Betreuung
zusammenzustellen und eine Lernumgebung zu schaffen, in welcher
Heterogenität willkommen ist. In der zweiten Dimension steht zum
Beispiel die Kindertagesstätte der Aufgabe gegenüber, allen Bedürfnissen
gerecht zu werden und den Kindern positive soziale Interaktionen in der
Gruppe zu ermöglichen. Fehlen Weiterbildungsangebote, könnte ein Mangel
an gut ausgebildetem Personal entstehen, welches sich wiederum auf die
Qualität der Betreuung auswirken kann.

In einem weiteren Schritt wird in \cref{sec:gelingensfaktoren} auf die
Ebene der Kindertagesstätte eingegangen. Es sollen Faktoren aufgezeigt
werden, welche eine inklusive Betreuung positiv beeinflussen.

\hypertarget{sec:gelingensfaktoren}{%
\section{Gelingensfaktoren der inklusiven
Betreuung}\label{sec:gelingensfaktoren}}

Kitas haben die Chance eine Vorreiterrolle in dieser ganzen
Inklusionsdebatte einzunehmen. Sie haben die Möglichkeit mit bestem
Beispiel voranzuschreiten und der Gesellschaft wie auch dem Schulsystem
zu zeigen, wie mit Vielfalt umgegangen werden kann. Das Ökologische
Mehrebenenmodell von Heimlich
(\protect\hyperlink{ref-inklusionQualituxe4t_Heimlich}{2015}) zeigt fünf
Kompetenzschwerpunkte, welche für die Entwicklung inklusiver
Kindertageseinrichtungen von Bedeutung sind (siehe
\cref{fig:mehrebenenmodell}). Der Kern des Modells bilden die Kinder mit
ihren individuellen Förderbedürfnissen. Eine weitere Kompetenz ist das
Gestalten und Ermöglichen von inklusiven Spiel- und Lernsituationen. Die
Zusammensetzung eines multiprofessionellen Teams wird auch als
Kompetenzschwerpunkt angesehen. Zudem zählen inklusive
Eintrichtungskonzeptionen und externe Unterstützungssysteme zu den
Schwerpunkten im Mehrebenenmodell. Diese fünf Ebenen tragen dazu bei,
Kindern und ihren individuellen Bedürfnissen gerecht zu werden.

\svgWithCaption{files/Heimlich.svg}{Ökologisches Mehrebenenmodell der
Entwicklung inklusiver Kindertageseinrichtungen Heimlich
(\protect\hyperlink{ref-inklusionQualituxe4t_Heimlich}{2015, S.
35})}{width=1.0\textwidth}[fig:mehrebenenmodell]

Damit eine inklusive Kindertageseinrichtung zustande kommt, nennt
Heimlich (\protect\hyperlink{ref-inklusionQualituxe4t_Heimlich}{2015, S.
35}) die Erarbeitung von fünf Entwicklungsprozessen. Sie werden
nachfolgend umschrieben.

\textbf{Mit Heterogenität umgehen lernen}\\
Pädagogische Fachkräfte sind gefordert die vorhandene Heterogenität
wahrzunehmen und Spielangebote den Fähigkeiten der Kinder anzupassen.
Aus diesem Grund sollten Beobachtungsfähigkeiten der pädagogischen
Fachpersonen besonders geschult werden (ebd.). Dadurch bauen sie
Verständnis für vorhandene Risikofaktoren und Entwicklungsproblemen auf.
Weitere Ressourcen welche hinzugezogen werden könnten sind Eltern und
externe Fachpersonen
(\protect\hyperlink{ref-sarimskiBehinderteKinder2011}{Sarimski, 2011}).
Die Kindertagesstätte sollte zudem die Möglichkeit haben, eine
Zweitmeinung, zum Beispiel von einer Heilpädagogischen Fachperson,
einholen zu können. Dadurch könnte die Qualität der Kita gesteigert
werden und der Besuch einer Kindertagesstätte würde einen stärkeren
präventiven Charakter erhalten.

\textbf{Inklusive Gruppenarbeit gestalten}\\
Das “Spiel” ist der gemeinsame Nenner bei Kindern. Pädagogische
Fachkräfte sollten demnach in der Lage sein solche gemeinsame “Spiel-
und Lernsituationen” bewusst zu initiieren und zu leiten. Nach Heimlich
(\protect\hyperlink{ref-inklusionQualituxe4t_Heimlich}{2015}) sollten
sie sich in das Spiel der Kinder einbringen, sodass neue Spielideen
entstehen können und die Spielintensität der Kinder beibehalten wird.
Pädagogische Fachpersonen sollten zudem in der Lage sein den Raum
flexibel zu nutzen, sowie Spielmaterialien einzusetzen, welche
unterschiedliche Sinneskanäle anregen.

\textbf{Teamarbeit entwickeln}\\
Heimlich (\protect\hyperlink{ref-inklusionQualituxe4t_Heimlich}{2015})
betont die Wichtigkeit der Teamarbeit für den Inklusionsprozess. In der
Inklusionsarbeit setzt sich das Team oftmals aus unterschiedlichen
Professionen zu einem multiprofessionellen Team zusammen. In diesen
können Vorstellungen und Umsetzung betreffend Inklusion gemeinsam
geplant und diskutiert werden. Ein multiprofessionelles Team bietet die
Chance, voneinander zu lernen und bei auftretenden Schwierigkeiten auf
unterschiedliches Wissen zurückgreifen zu können. Nach Rafferty \&
Griffin (\protect\hyperlink{ref-raffertyux5cux26Griffin2005}{2005}) sind
Personen welche über eine höhere Ausbildung verfügen, im allgemeinen
positiver gegenüber Inklusion eingestellt. Sie können ihr Wissen und
ihre Einstellung in die Kindertagesstätte miteinfliessen lassen und den
Inklusionsgedanken weiter vorantreiben.

\textbf{Inklusive Konzeptionen erstellen}\\
Pädagogische Fachkräfte in Kindertagesstätten sollten fähig sein,
Prozesse der institutionellen Entwicklung bewusst wahrzunehmen und in
gemeinsamer Zusammenarbeit Zielsetzungen zur Umsetzung von Inklusion
aufzustellen
(\protect\hyperlink{ref-inklusionQualituxe4t_Heimlich}{Heimlich, 2015}).
Ein Qualitätsmerkmal von inklusiven Kindertageseinrichtungen ist das
fortlaufende Reflektieren des pädagogischen Konzepts.

\textbf{Regionale Netzwerke bilden}\\
Dazugehören Elternarbeit und interdisziplinäre und interinstitutionelle
Zusammenarbeit. Eltern können den Fachpersonen wichtige Informationen
über den Umgang mit ihrem Kind liefern (ebd.). Sie kennen es am Besten.
Dieser Austausch schafft gegenseitiges Vertrauen. Schlussendlich
profitieren beide Parteien voneinander. Seitz \& Finnern
(\protect\hyperlink{ref-seitzfinnern}{2012}) spricht von gleichwertigen
Erziehungspartnerschaften, welche sich auf Augenhöhe begegnen. Weiter
ist die Kindertagesstätte auf “…externe Kooperationspartner und
Unterstützungssysteme angewiesen“
(\protect\hyperlink{ref-inklusionQualituxe4t_Heimlich}{Heimlich, 2015,
S. 36}). Fachpersonen aus anderen Professionen können hinzugezogen
werden, damit den individuellen Förderbedürfnissen von Kindern mit
Beeinträchtigung gerecht werden kann. Dazu zählen zum Beispiel Behörden
oder Institutionen, welche Fremdbetreuung mitfinanzieren
(\protect\hyperlink{ref-tiki}{Lütolf \& Schaub, 2021}).

Hervorzuheben gilt die positive Einstellung / Haltung gegenüber der
anzutreffenden Vielfalt (siehe dazu \cref{sec:haltungsfrage}). Diese
leistet einen wichtigen Beitrag zur Inklusion
(\protect\hyperlink{ref-einstellungPuxe4dagogischerFachkruxe4ften_Weltzien}{Weltzien
\& Söhnen, 2020}).

\hypertarget{sec:forschungsgrundlageneu}{%
\section{Forschungsgrundlage zu Inklusion in
Kindertagesstätten}\label{sec:forschungsgrundlageneu}}

In den nächsten Abschnitten wird anhand von Studien erläutert, welche
Vorstellungen über Inklusion in Kindertagesstätten kursieren und wieso
“Haltung” als Schlüsselfaktor in dieser Thematik bezeichnet werden kann,
nicht nur von Erwachsenen, sondern auch von gleichaltrigen Kindern.
Danach wird die Integrationschance von Kindern mit Beeinträchtigung
unter Berücksichtigung der Art und Schwere der Beeinträchtigung seitens
pädagogischen Fachpersonen eingeschätzt. Im Anschuss wird der Begriff
“soziale Teilhabe” erläutert. Zuletzt werden Vorteile und Risiken von
Inklusion aufgegriffen.

\hypertarget{sec:inklusionkita}{%
\subsection{Anzutreffende Vorstellungen zu
Inklusion}\label{sec:inklusionkita}}

Der diffuse Begriff “Inklusion” erschwert die Umsetzung in der Praxis.
Trescher (\protect\hyperlink{ref-eineKriseDieKeineSeinDarf}{2018})
bestätigt, dass eine grosse begriffliche Unsicherheit bei den
pädagogischen Fachpersonen vorhanden ist. Inklusive Handlungspraxis ist
für sie das, was gegenwärtig praktiziert wird und was im Vorfeld stets
betrieben worden ist. Daraus lässt sich schliessen, dass eine kritische
Auseinandersetzung mit dem Inklusionsparadigma und der eigenen
Handlungspraxis fehlt.

Der Fachbeitrag von Knauf \& Graffe
(\protect\hyperlink{ref-alltagsTheorienUeberInklusion}{2016}) soll einen
Einblick in vorhandene Alltagstheorien gewähren. Dazu wurden 104
Situationsbeschreibungen, welche von pädagogischen Fachkräften als
inklusive Situationen wahrgenommen werden, analysiert. Die gewonnenen
Erkenntnisse über sogenannte Alltagstheorien halten sie in Form von fünf
zugespitzten Thesen fest. Diese werden im Anschluss erläutert. Sie
liefern einen wertvollen Einblick in mögliche, anzutreffende
Vorstellungen über Inklusion in der Praxis.

\begin{description}
\item[These 1:]
\textbf{Inklusion bezieht sich auf Behinderung}
(\protect\hyperlink{ref-alltagsTheorienUeberInklusion}{Knauf \& Graffe,
2016, S. 193})

Die Erhebung zeigt, dass 80 \% der Befragten eine Situation beschreibt,
in welcher ein Kind mit Beeinträchtigung im Zentrum steht. Daraus kann
die Schlussfolgerung gezogen werden, dass in Alltagstheorien der Begriff
“Inklusion” mit Menschen mit einer Beeinträchtigung in Verbindung
gebracht wird (ebd.).

Hinz (\protect\hyperlink{ref-Hinz2013}{2013}) bezeichnet dies als enge
Auffassung. Laut Lüders (\protect\hyperlink{ref-luders2014}{2014})
können weitere Heterogenitätsmerkmale wie zum Beispiel ethnische
Herkunft, sozioökonomischer Status oder Religion miteinbezogen werden.
Saalfrank \& Zierer
(\protect\hyperlink{ref-saalfrank_inklusion_2017}{2017}) nennen fünf
Dimensionen zur Inklusion (siehe \cref{fig:dimensionen} in
\cref{sec:annuxe4herungsversuch}).
\item[These 2:]
\textbf{Der Fokus der inklusiven Arbeit liegt auf einzelnen Kindern}
(\protect\hyperlink{ref-alltagsTheorienUeberInklusion}{Knauf \& Graffe,
2016, S. 194})

In der Situationsanalyse von Knauf \& Graffe
(\protect\hyperlink{ref-alltagsTheorienUeberInklusion}{2016}) stellt
sich heraus, dass ein einzelnes Kind, welches anders ist als andere, im
Vordergrund steht. Es soll in eine Gruppe integriert werden. Diese
Haltung ist der Integration zuzuordnen, welche von der
“Zwei-Gruppen-Theorie” ausgeht. In dieser wird zwischen Kindern mit
sonderpädagogischem Förderbedarf und solchen ohne unterschieden
(\protect\hyperlink{ref-wocken}{Wocken, 2009}). Die Begriffe Inklusion
und Integration werden hier als Synonym verwendet. Rohrmann
(\protect\hyperlink{ref-rohrmann_inklusion_2014}{2014}) hat diese
Entwicklung vorausgesagt.
\item[These 3:]
\textbf{Die Kindergruppe wird als wesentliche Ressource für Inklusion
gesehen} (\protect\hyperlink{ref-alltagsTheorienUeberInklusion}{Knauf \&
Graffe, 2016, S. 194})

Die Gruppe wird als wichtige Ressource für eine gelingende Inklusion
verstanden. Es kristallisiert sich heraus, dass die Befragten die
Kindergruppe als ein Instrument betrachten, welches helfen soll, dass
Kind mit Beeinträchtigung zu integrieren. Nicht die Gruppe steht im
Zentrum sondern das zu integrierende Kind (ebd.). Damit wären wir wieder
bei der Zwei-Gruppen-Theorie (\cref{sec:inklusionversusintegration}).
Kordulla (\protect\hyperlink{ref-peer-learning}{2017}) spricht von
Peer-Learning-Prozessen. In Gruppen mit gleichaltrigen Kindern ergeben
sich Lernchancen für die Erweiterung von sozialen und kognitiven
Fähigkeiten. Jedoch verbirgt sich bei zu grossen Unterschieden die
Gefahr, dass vermehrt Konflikte, negative Emotionen und ein erhöhter
koordinativer Aufwand entstehen könnte. Laut
(\protect\hyperlink{ref-buysse}{Buysse, Goldman \& Skinner, 2002})
erleben Kinder mit Beeinträchtigung in inklusiven Kindertagesstätten
mehr soziale Interaktionen. Sie haben dadurch eine grössere Chance,
Freundschaftsbeziehungen aufzubauen.
\item[These 4:]
\textbf{Inklusion wird als Überwinden von Barrieren und nicht als Abbau
von Barrieren gesehen}
(\protect\hyperlink{ref-alltagsTheorienUeberInklusion}{Knauf \& Graffe,
2016, S. 194})

In der Analyse von Knauf \& Graffe
(\protect\hyperlink{ref-alltagsTheorienUeberInklusion}{2016}) stellt
sich heraus, dass das Überwinden von Barrieren von einzelnen Kindern für
die Befragten im Vordergrund steht. Dabei sollen Barrieren in der
Leistungsfähigkeit und in der sozialen Interaktion mit anderen Kindern
überwunden werden. Dies erinnert an die “Zwei-Gruppen-Theorie” welche
der Integration zugeschrieben wird. Kinder mit besonderen Bedürfnissen
sollen in eine Gruppe integriert werden. Die Entwicklungssrückstände der
betroffenen Kindern sollen minimiert werden. Der Fokus liegt auf
einzelnen Kindern und nicht wie in der Inklusion auf der ganzen Gruppe.
Albers (\protect\hyperlink{ref-albers_vielfalt_2012}{2012}) betont, dass
in der Umsetzung einer inklusiven Pädagogik der Vielfalt der Kinder
Rechnung getragen werden sollte. Dies setzt eine Professionalität der
Kindertageseinrichtung sowie der Mitarbeitenden voraus.
\item[These 5:]
\textbf{Fachkräfte sehen sich als Begleiter und Moderatoren}
(\protect\hyperlink{ref-alltagsTheorienUeberInklusion}{Knauf \& Graffe,
2016, S. 194})

Die Befragten unterstützen die Kinder in ihrer Entwicklung und setzen
sich dafür ein, dass sie mit unterschiedlichen Kindern in Kontakt
treten. Fachpersonen stehen zudem für die Bedürfnisse der Kinder gegen
aussen ein und verteidigen oder werben für das Verständnis von
skeptischen Mitarbeitenden. Die interdisziplinäre Zusammenarbeit mit
externen Fachpersonen sehen sie als Ressource an (ebd.).
\end{description}

\hypertarget{sec:haltungsfrage}{%
\subsection{Haltung als Schlüsselfaktor}\label{sec:haltungsfrage}}

Aus den zusammengetragenen Thesen in \cref{sec:inklusionkita} wird nicht
ersichtlich, wie sich das Kita-Personal zur Thematik “Inklusion in
Kindertagesstätten” positioniert. Es wurde lediglich ein Einblick in das
Verständnis über Inklusion gewährt. Jedoch deuten die Thesen auf eine
Defizit orientierte Einstellung hin. Im Anschluss wird ein Versuch
gestartet, den Hergang, welcher zu dieser Sichtweise führen kann, zu
rekonstruieren.

Auf der Suche nach einem Fremdbetreuungsplatz werden spätestens bei der
Hospitation erste Informationen über das Kind ausgetauscht. Die Leitung
nimmt eine Einschätzung der benötigten Ressourcen, wie zum Beispiel
Personal, vor. Bereits bei diesem Schritt kann eine Familie
zurückgewiesen werden.

Die Kita selbst muss gewisse Vorgaben einhalten. Die Räumlichkeiten
einer Kita geben zum Beispiel die maximale Kinderzahl vor, welche sich
gleichzeitig darin aufhalten dürfen. Eine weitere Richtlinie ist der
Betreuungsschlüssel, welcher der Verband Kinderbetreuung Schweiz
(kibesuisse,
\protect\hyperlink{ref-richtlinienKindertagessstuxe4tten_2020}{2020})
vorgibt. In diesem ist geregelt, wie viele Kinder auf eine
Betreuungsperson hochgerechnet werden können.

Hat eine Platzierung stattgefunden, ist die Wahrscheinlichkeit gross,
dass das Kind mit einer Beeinträchtigung in den Fokus der
Kita-Mitarbeitenden rückt. Es sticht durch besonderes Verhalten oder
zusätzlichen Unterstützungsbedarf im Alltag hervor. Fachkräfte sind
gefordert mit diesen Besonderheiten umzugehen. Diese Kinder weichen von
den Erwartungen des Betreuungspersonals ab und fordern sie in ihren
pädagogischen und didaktischen Fähigkeiten und Kompetenzen. Graumann
(\protect\hyperlink{ref-unerfullbarevision}{2018}) nennt diesbezüglich,
dass je unflexibler der Unterricht oder die Sozialformen sind, desto
eher werden Kinder, welche nicht dem Durchschnitt entsprechen, als
störend empfunden. Die Vielfalt der Kinder kann somit als eine
didaktische Herausforderung betrachtet werden.

Die Haltung einzelner pädagogischen Fachpersonen und der Leitung
bezüglich Inklusion ist dabei entscheidend und beeinflusst ihr Handeln
gegenüber Kindern mit speziellen Bedürfnissen. Die Besonderheiten können
negativ interpretiert oder der allgemeinen Vielfalt der Kinder
zugeschrieben werden. Bestenfalls werden Wege für die Teilhabe im Alltag
gesucht. Im schlimmsten Fall wird eine Kita-Platzierung verweigert oder
aufgelöst.

Daraus kann abgeleitet werden, dass die Haltung des Kita-Personals
massgeblich über Gelingen oder Scheitern einer Kita-Platzierung von
Kindern mit besonderen Bedürfnissen mitbestimmt. Die Haltung gegenüber
Inklusion kann als Gelingensfaktor (siehe \cref{sec:gelingensfaktoren})
gezählt werden.

In der Studie von Rafferty \& Griffin
(\protect\hyperlink{ref-raffertyux5cux26Griffin2005}{2005}) wird
ersichtlich, dass eine allgemeine Bereitschaft der Kindertagesstätten,
Kinder mit besonderen Bedürfnissen aufzunehmen, vorhanden ist. Dies
bestätigt die Befragung pädagogischer Fachkräfte von Wiedebusch, Lohmann
\& Hensen (\protect\hyperlink{ref-wiedebusch2015}{2015}) zur Betreuung
von chronisch kranken Kindern. Die Vorbereitung auf Kinder mit
chronischen Krankheiten in der Berufsausbildung wie auch Erfahrungen in
der Kindertagesstätte prägen die Einstellung von Fachkräften positiv
(\protect\hyperlink{ref-wiedebusch2015}{Wiedebusch et al., 2015}). Dies
bestätigen Engstrand \& Roll-Pettersson
(\protect\hyperlink{ref-zakirova}{2012}).

Mangelndes Wissen von pädagogischen Fachkräften über eine
Beeinträchtigung oder das nicht Einschätzen können der Bedürfnisse von
Kindern mit einer Beeinträchtigung lösen Verunsicherung aus. Es besteht
die Möglichkeit, dass Fachpersonen das Gefühl verspüren, zu wenig Zeit
für das betreffende Kind aufbringen zu können. Dies hinterlässt ein
Gefühl von Unzufriedenheit und kann die Einstellung gegenüber Inklusion
negativ beeinflussen
(\protect\hyperlink{ref-raffertyux5cux26Griffin2005}{Rafferty \&
Griffin, 2005}). Weiter wird erwähnt, dass die Einstellung zu Inklusion
unabhängig davon sei, wie viele Jahre an Arbeitserfahrung jemand
mitbringe (ebd.). Zudem spiele der Schweregrad oder die Art der
Beeinträchtigung eine Rolle. Auf diese Thematik wird im nachfolgenden
Kapitel (\cref{sec:schweregradBeeintruxe4chtigung}) eingegangen.

Die Auswertung des Weiterbildungsprogramm InkluKiT von Weltzien \&
Söhnen
(\protect\hyperlink{ref-einstellungPuxe4dagogischerFachkruxe4ften_Weltzien}{2020,
S. 102}) zeigt, dass Einstellungen zu Inklusion veränderbar sind.
Grundlegend sei dabei die kompetenzorientierte Arbeit mit allen
pädagogischen Fachpersonen auf den Ebenen “Haltung, Wissen, Handeln und
Team”. Weiterbildungsprogramme können folglich einen wertvollen Beitrag
zur Erweiterung des Inklusionsverständnis in Kitas leisten. Weiter
nennen sie, dass anzutreffende Merkmale in der Einrichtung wie zum
Beispiel eine ausgewogene Arbeitsdichte und Arbeitsrhythmus, die
Einstellung positiv beeinflussen können.

Ein Stolperstein ist, dass externe Unterstützung oder Beratung im Kanton
Luzern in Form von KITAplus (eine Fachperson aus dem Frühbereich berät
oder coacht das Personal der Kita) erst erfolgen kann, wenn das
betreffende Kind am Heilpädagogischen Dienst Luzern angemeldet ist.

\hypertarget{sec:schweregradBeeintruxe4chtigung}{%
\subsection{Schweregrad und Art der
Beeinträchtigung}\label{sec:schweregradBeeintruxe4chtigung}}

Miedander (\protect\hyperlink{ref-miedander}{1997, S. 13}) stellte nach
einer Befragung von 30 integrativen Einrichtungen fest, dass
Integrationschance aufgrund der Schwere der Beeinträchtigung
unterschiedlich eingeschätzt wurde. Ihre Erkenntnisse werden im
Anschluss zusammengefasst, da laut Sarimski
(\protect\hyperlink{ref-sarimskiBehinderteKinder2011}{2011}) solche
Einstellungen bis heute anzutreffen sind.

\begin{itemize}
\tightlist
\item
  Die Integration von Kindern mit einer körperlichen Behinderung wird
  von den pädagogischen Fachpersonen als problemlos angesehen. Es sei
  höchstens eine Frage des verfügbaren Personals.
\item
  Genauso problemlos wird die Integration von Kindern mit einer
  Sprachbehinderung bewertet. Interessant ist, dass bei
  Integrationsversuchen dieser Kindergruppe jedoch vermehrt von
  Problemen berichtet wurde.
\item
  Als nicht schwierig wird die Integration von blinden Kindern
  angesehen, sofern keine schwere Beeinträchtigungsform hinzukommt.
\item
  Kinder mit einer geistigen Behinderung stellen eine grössere
  Herausforderung für die Integration dar. Integrationserfolge seien
  schwerer zu erreichen im Vergleich mit anderen Behinderungsgruppen.
\item
  In der Bewertung der Integration wird zwischen Gehörlosen und
  Hörgeschädigten unterschieden. Kinder mit einer leichten Hörschädigung
  könnten ohne weiteres integriert werden. Hingegen sei die Integration
  von gehörlosen Kindern nicht empfehlenswert, da diese einen geringen
  Wortschatz ausbilden würden.
\item
  Bei der Kindergruppe mit Verhaltensauffälligkeiten weisen pädagogische
  Fachpersonen daraufhin, dass diese Gruppe im Vergleich mit anderen
  Sinnesbehinderungen weit grössere Probleme für die Integration
  aufwirft. Des Weiteren fühle sich das Personal überfordert, sofern
  keine Unterstützung veranlasst wird.
\end{itemize}

Nach Miedander (\protect\hyperlink{ref-miedander}{1997}) lässt sich
zusammenfassend sagen, dass die Art der Beeinträchtigung kein Grund für
einen Ausschluss in ein integratives Setting sein kann und darf. Jedoch
wird ersichtlich, dass je nach Art der Beeinträchtigung und
Beeinträchtigungsgrad unterschiedliche Anforderungen an das Personal
gestellt werden. Anbieter haben die Arbeit im integrativen Setting mit
körper- und sprachbehinderten, blinden oder gehörbeeinträchtigten
Kindern als leichter empfunden als mit Kindern welche gehörlos, eine
geistige Beeinträchtigung oder eine Verhaltensauffälligkeit aufweisen.
Pädagogische Fachpersonen weisen zusätzlich darauf hin, dass eine
pauschale Bewertung einer spezifischen Beeinträchtigung nicht möglich
sei. Jeder Fall sollte einzeln betrachtet und beurteilt werden.

In der Studie von Rafferty \& Griffin
(\protect\hyperlink{ref-raffertyux5cux26Griffin2005}{2005}) wird
bestätigt, dass Anbieter und Eltern Inklusion von Kindern mit einer
Sprach-, Motorik- oder Gehörbeeinträchtigung grösstenteils befürworten.
Bei vorhandenen emotionalen Thematiken, Autismus-Spektrum-Störungen oder
kognitiven Einschränkungen fällt die Bereitschaft geringer aus.

Weiter nennen Rafferty \& Griffin
(\protect\hyperlink{ref-raffertyux5cux26Griffin2005}{2005}) den
Schweregrad der Beeinträchtigung als Schlüsselfaktor für die Einstellung
zu Inklusion. Je schwerer die Beeinträchtigung, desto geringer fällt die
Bereitschaft aus. Avramidis \& Norwich
(\protect\hyperlink{ref-europeanjournal}{2002}) bestätigen, dass
Lehrpersonen eher dazu bereit sind Kinder mit einer milden, körperlichen
oder sensorischen Beeinträchtigung zu integrieren, als Kinder mit
komplexeren Bedürfnissen. Zudem fanden sie genügend Hinweise, die darauf
schliessen lassen, dass starke Lerneinschränkungen oder
Verhaltensprobleme sich negativ auf die Einstellung von Lehrpersonen
auswirken (siehe \cref{sec:haltungsfrage}). Es fehle an Wissen im Umgang
mit Kindern mit speziellen Bedürfnissen. Vor allem wenn eine schwere
Beeinträchtigung vorliege, in welcher die Kommunikationsform erschwert
oder ein erhöhter Pflegebedarf vorhanden sei, welcher spezifisches
Fachwissen voraussetze. Weiter wird erwähnt, dass ungünstige
Konstellationen in der Gruppe zur Überforderung der pädagogischen
Fachpersonen führen kann. Dies sei zum Beispiel der Fall, wenn mehrere
Kinder in einer Gruppe mit einem erhöhten Pflegebedarf oder ausgeprägten
Verhaltensauffälligkeiten aufeinander treffen und Kinder ohne
Beeinträchtigung hinzukommen, welche das Betreuungspersonal zusätzlich
herausfordert
(\protect\hyperlink{ref-sarimskiBehinderteKinder2011}{Sarimski, 2011}).
Haben die Eltern die Beeinträchtigung ihres Kindes akzeptiert und
kooperieren mit der Kita, so seien die Integrationschancen erhöht.

\hypertarget{sec:peersEinstellung}{%
\subsection{Peers und ihre Einstellung}\label{sec:peersEinstellung}}

Nachdem die Sichtweisen bezüglich Inklusion aus dem Blickwinkel von
Erwachsenen beschrieben worden ist, rücken in diesem Abschnitt die
Einstellungen von gleichaltrigen Kindern in den Vordergrund.
Schliesslich steht das Miteinander und das voneinander Lernen im
Zentrum. Dies wird erst möglich, wenn gegenseitige Akzeptanz und
Verständnis vorhanden sind. Laut Siperstein, Parker, Bardon \& Widaman
(\protect\hyperlink{ref-siperstein2007}{2007}) werden Kinder mit einer
intellektuellen Beeinträchtigung von Gleichaltrigen oftmals nicht
akzeptiert. Sie seien ihnen gegenüber negativ eingestellt und würden mit
ihnen in der Freizeit wenig zu tun haben wollen. Walker \& Berthelsen
(\protect\hyperlink{ref-walker}{2007}) widersprechen dieser Aussage. In
ihrer Studie haben sie Hinweise gefunden, dass Kinder mit
Beeinträchtigungen von anderen Kindern sozial akzeptiert würden und
Freundschaften schliessen, im Vergleich mit anderen jedoch weniger. Die
Studie von Walker \& Berthelsen (\protect\hyperlink{ref-walker}{2007})
wurde mit Kindern aus dem Vorschulbereich durchgeführt und die Studie
von Siperstein et al. (\protect\hyperlink{ref-siperstein2007}{2007}) mit
Kindern aus der “middle school” (Oberstufe). Die gegensätzlichen
Studienergebnisse könnten auf den Altersunterschied zurückzuführen sein.
Jüngere Kinder stellen vielleicht weniger hohe Ansprüche an ihre
Spielkameraden und sind jeweils noch sehr auf ihr eigenes Spiel
fokussiert. Ältere Kinder nehmen die Unterschiedlichkeit deutlich
bewusster wahr und sind vielleicht weniger dazu bereit Kompromisse
einzugehen oder empfinden die Freundschaft als weniger gewinnbringend
oder sogar als Last.

Gerade weil Peers eine wertvolle Ressource darstellen wird der Frage
nachgegangen, welche Faktoren eine negative Einstellung gegenüber Kinder
mit speziellen Bedürfnissen und Beeinträchtigungen begünstigen und wie
dagegen vorgegangen werden kann.

Fehlendes Wissen zu Beeinträchtigungen ist ein möglicher Faktor.
Ausserdem finden sich Lehrpersonen in der Situation wieder, zwei
unterschiedliche Kindergruppen aneinander zu gewöhnen. Dies kann
gelingen oder scheitern. Weiter kann eine negative Einstellung gegenüber
Beeinträchtigung zu Hause oder von der Gesellschaft vermittelt werden,
sowie ein allgemeines Unwohlsein gegenüber Verschiedenartigkeit
(\protect\hyperlink{ref-storytelllingprogram}{Giagazoglou \& Papadaniil,
2018}).

In \cref{sec:haltungsfrage} wurde bereits erwähnt, dass Einstellungen zu
Inklusion veränderbar sind, wieso also auch nicht die Einstellung von
Kindern gegenüber Kindern mit speziellen Bedürfnissen?

Giagazoglou \& Papadaniil
(\protect\hyperlink{ref-storytelllingprogram}{2018}) führten in einer 1.
Klasse mit sechs bis siebenjährigen Kindern eine Pilotstudie durch.
Folgende Frage stand im Vordergrund: “Lässt sich die Einstellung von
Peers in Bezug auf Beeinträchtigung anhand eines Förderprogramms
beeinflussen?” Das Programm nimmt zwölf Stunden in Anspruch und setzt
sich aus Lese-Sequenzen und Rollenspielen zusammen. Die Leitfigur ist
eine kleine Schildkröte mit Down-Syndrom. Aus ihrer Studie geht hervor,
dass ein signifikant positiver Effekt bezüglich Akzeptanz und
Verständnis für Kinder mit einer kognitiven Beeinträchtigung bewirkt
werden konnte. Dank einem Rollenspiel können sie in die Haut eines
anderen schlüpfen. Dies erleichtert ihnen den Perspektivenwechsel. Sie
erfahren demzufolge am eigenen Leib, wie schwierig es sein kann, den
Alltag mit einer Beeinträchtigung zu bewältigen. Mit diesen
Interventionen wird emphatisches Verhalten gefördert, was sich wiederum
auf die Akzeptanz und ein positives Klassenklima auswirken kann. Damit
ein solches Programm durchgeführt wird, ist jedoch ein Grundinteresse
seitens der Institution notwendig, sowie das Wissen, dass solche
Programme existieren. Schlussendlich ist dies eine Haltungsfrage.

In einer Kita sind Rollenspiele aufgrund des Alters der Kinder nicht
durchführbar. Was sich jedoch anbieten würde sind zum Beispiel
inszenierte Tischtheater. Zudem können die Kinder mit simplen Fragen wie
zum Beispiel “Wie fühlt sich die Schildkröte? Was könnte mit der
Schildkröte gespielt werden? Was kann die Schildkröte besonders gut und
was ist für sie schwieriger?” auf die Thematik der Andersartigkeit
sensibilisiert werden. Der Zugang über Bilderbücher ist eine weitere
wundervolle Methode, welche in den Kita-Alltag eingebaut werden kann.
Pädagogische Fachpersonen übernehmen zudem eine Vorbildfunktion
gegenüber den Kindern.

\hypertarget{sec:sozialeteilhabe}{%
\subsection{Soziale Teilhabe}\label{sec:sozialeteilhabe}}

“Dabeisein ist nicht alles - oder doch?” Dieser Titel bestückt den
Artikel von Sarimski
(\protect\hyperlink{ref-dabeiseinIstNichtAllesOderDoch}{2015}). Er setzt
sich mit der Frage auseinander, wie es um die soziale Teilhabe von
Kindern mit sehr schwerer und mehrfacher Behinderung in
Kindertagesstätten steht.

Nur weil ein Kind eine Kindertagesstätte besucht, darf nicht davon
ausgegangen werden, dass Kinder mit speziellen Bedürfnissen in das
soziale Geschehen eingebunden sind und somit von einer gelingenden
Inklusion gesprochen werden kann. Die Situation muss genauer betrachtet
werden. Der Titel “Dabeisein ist nicht alles - oder doch?” lässt zwei
Interpretationen zu: Sobald das Kind mit besonderen Bedürfnissen eine
Kita besucht, ist von sozialer Teilhabe auszugehen, dabei musss es nicht
aktiv am Geschehen teilnehmen. Es genügt, wenn es mit seinem Rollstuhl
im Kreis platziert ist. Andererseits könnte behauptet werden, dass erst
von sozialer Teilhabe die Rede ist, wenn es sich in die Aktivität
einbringt. Folgende Fragestellung lässt sich daraus ableiten: Was wird
unter sozialer Teilhabe verstanden?

Heimlich (\protect\hyperlink{ref-inklusionQualituxe4t_Heimlich}{2015})
betrachtet “soziale Teilhabe” als ein Teil von Inklusion, in welcher das
Kind selbstbestimmt partizipieren und bei gemeinsamen Aktivitäten, wie
alle anderen die Möglichkeit hat, sich einzubringen. Die World Health
Organization (\protect\hyperlink{ref-worldhealthorganisation}{2005})
verwendet den Begriff “Teilhabe” als Synonym zu Partizipation und
bezieht sich auf das “Einbezogensein” in einer Lebenssituation. Laut
Klein, Lorenz-Medick \& Bamikol-Veit
(\protect\hyperlink{ref-kleineva}{2012}) zeigt sich Teilhabe anhand von
Interaktion und Kooperation. Innerhalb einer Kindertagesstätte kann
Teilhabe durch pädagogische Fachpersonen bewusst gefördert oder
antizipiert werden. Gruppenaktivitäten oder Spielsequenzen eigenen sich
dafür wunderbar.

Sarimski (\protect\hyperlink{ref-dabeiseinIstNichtAllesOderDoch}{2015,
S. 149}) untersucht in einer quantitativen explorativen Studie die
soziale Teilhabe von Kindern mit schwerer und mehrfacher Behinderung in
sonderpädagogischen und integrativen Kindergärten. Zu folgenden
Erkenntnisse ist er gelangt:

\begin{itemize}
\tightlist
\item
  Für eine gelingende soziale Kontaktaufnahme waren visuelle und
  körperliche Zuwendung charakteristisch. Diese gingen gleich oft von
  Kindern mit schwerer und mehrfacher Behinderung als auch von anderen
  Kindern aus.
\item
  In 30 \% der Beobachtungszeit lässt sich von sozialer Teilhabe
  sprechen. Während der Hälfte der Zeit sind andere Kinder in der Nähe.
  Die Möglichkeit für eine Kontaktaufnahme wäre gegeben. In 25 \% der
  Zeit ist das Kind alleine. Die Mehrzahl der Kontaktaufnahmen gelingen
  im Freispiel.
\item
  Die Anzahl gelingender sozialer Kontaktaufnahmen von Kindern mit
  schwerer und mehrfacher Behinderung ist im Vergleich mit anderen
  Kindern niedrig. Hervorzuheben gilt, dass Interesse und Initiative der
  Kinder für eine Kontaktaufnahme vorhanden sind.
\item
  Es sind grosse individuelle Unterschiede im Gelingen von
  Kontaktaufnahmen zu verzeichnen.
\item
  Sie finden keinen Hinweis darauf, dass die Art der Behinderung
  Einfluss auf die Häufigkeit von sozialen Kontaktaufnahmen nimmt.
  Dasselbe gilt für den Ort.
\item
  Vielmehr sei die Bereitschaft der anderen Kinder oder die
  Unterstützung durch pädagogische Fachpersonen für eine soziale
  Teilhabe entscheidend.
\end{itemize}

In der Studie von Walker \& Berthelsen
(\protect\hyperlink{ref-walker}{2007}) schätzen Lehrpersonen Kinder mit
Beeinträchtigungen in der sozialen Kompetenz tiefer ein als Kinder ohne
Beeinträchtigungen. Sie seien weniger kompetent im prosozialem
Verhalten, neigen eher zu aggressiven Verhaltensweisen und würden sich
öfters zurück ziehen.

Des Weiteren ist eine Beobachtungsstudie zur sozialen Teilhabe von
Kindern mit Behinderung in Kindertagesstätten von Lütolf \& Schaub
(\protect\hyperlink{ref-beobachtungsstudie}{2019}) durchgeführt worden.
Dabei wurden die Aspekte “Beteiligung in Spiel- und Gruppenprozessen und
Interaktionen” untersucht und miteinander verglichen. Aus der
Beobachtungsstudie geht hervor:

\begin{quote}
Kinder mit einer Behinderung waren seltener in Interaktion als ihre
Peers, waren weniger oft Senderin oder Sender von Interaktionsangeboten
und standen seltener in Interaktion mit anderen Kindern aus der Gruppe.
Auch reagierten die Kinder mit Behinderung auf ein Interaktionsangebot
seltener aufrechterhaltend als ihre Peers.
(\protect\hyperlink{ref-beobachtungsstudie}{Lütolf \& Schaub, 2019, S.
187})
\end{quote}

Im allgemeinen weisen Lütolf \& Schaub
(\protect\hyperlink{ref-beobachtungsstudie}{2019}) daraufhin, dass
aufgrund der enormen Heterogenität der Kinder mit sonderpädagogischem
Förderbedarf eine grosse Vielfalt an unterschiedlichen Verhaltensweisen
gezeigt wird. Die Schwierigkeit von pädagogischen Fachpersonen besteht
einerseits darin, das gezeigte Verhalten in der Situation wahrzunehmen,
zu interpretieren und entsprechend zu reagieren. Andererseits sind
Kinder mit Beeinträchtigung darauf angewiesen, dass pädagogische
Fachpersonen soziale Kontaktaufnahmen initiieren oder fördern. Dies kann
zum Beispiel in Form von Vermitteln zwischen anderen Peers und dem Kind
sein oder dem Herstellen von Situationen, in welchen Interaktion möglich
werden. Dies stellt pädagogische Fachpersonen vor eine grosse
Herausforderung
(\protect\hyperlink{ref-schwerermehrfachbehinderung}{Gutekunst, Schreier
\& Sarimski, 2012}; \protect\hyperlink{ref-beobachtungsstudie}{Lütolf \&
Schaub, 2019};
\protect\hyperlink{ref-dabeiseinIstNichtAllesOderDoch}{Sarimski, 2015}).
Freispielsituationen, Übergänge oder Routinen, in welchen Fachpersonen
unterstützend und begleitend tätig sind, stellen eine grosse
Herausforderung dar. Laut Reyhing, Frei, Burkhardt Bossi \& Perren
(\protect\hyperlink{ref-fachkraftkindinteraktion}{2019}) sind dafür hohe
mentale Flexibilität und Sicherheit im Gestalten solcher
Lerngelegenheiten gefragt. Sie haben herausgefunden, dass Fachpersonen
die aktive Lernunterstützung in planbaren Einheiten wie zum Beispiel
Kreissequenzen leichter fällt als in freien Situationen.
Heilpädagogische Beratung (z.B. KITAplus Luzern) setzt dort an. Die
pädagogischen Fachpersonen der Kita werden mithilfe des Coaching auf die
Bedürfnisse der Kinder sensibilisiert. Sie werden bei der Interpretation
von gezeigten Verhaltensweiten von Kindern mit speziellen Bedürfnissen
unterstützt, erhalten einen Input, wie sie Übergänge oder Rituale im
Alltag einbauen können oder werden dabei angeleitet, wie sie Kinder mit
sonderpädagogischem Förderbedarf in Freispiel-Aktivitäten
miteinzubeziehen können.

Ausser Frage steht, dass pädagogische Fachkräfte das Gelingen sozialer
Teilhabe von Kindern mit Beeinträchtigung massgeblich mitbestimmen. Eine
weitere Rolle spielen die individuellen Voraussetzungen des Kindes
(\protect\hyperlink{ref-beobachtungsstudie}{Lütolf \& Schaub, 2019}).
Ein Kind mit schwerer oder mehrfacher Beeinträchtigung, welches sich zum
Beispiel anhand von blinzeln verständigt, hat weniger Möglichkeiten sich
aktiv in Situationen einzubringen oder in Interaktion mit anderen Peers
zu gelangen, als ein Kind, welches rein körperlich eingeschränkt ist.
Das blinzelnde Kind ist auf ein aufmerksames Gegenüber angewiesen,
welches sein Blinzeln zu deuten weiss.

Fakt ist, mit einem Besuch in der Kindertagesstätte kann nicht
automatisch von einer gelingenden sozialer Teilhabe gesprochen werden.
Jedes Kind bringt individuelle Voraussetzungen und Bedürfnisse mit um am
sozialen Geschehen in der Kindertagesstätte teilzunehmen. Die
Herausforderung der pädagogischen Fachpersonen besteht darin, jedes
einzelne Kind entsprechend seinen Möglichkeiten am Kita-Alltag
teilnehmen zu lassen.

\hypertarget{sec:forschungsgrundlageInklusion}{%
\subsection{Vorteile und Risiken von
Inklusion}\label{sec:forschungsgrundlageInklusion}}

Rafferty \& Griffin
(\protect\hyperlink{ref-raffertyux5cux26Griffin2005}{2005}) führten eine
Studie mit 237 Eltern von Kindern mit und ohne Beeinträchtigung durch.
Im Vordergrund stand die Ermittlung zu Vorteilen und Risiken von
Inklusion im Vorschulbereich. Parallel dazu sind 118 Anbieter, welche
ein inklusives Setting im Vorschulbereich anbieten, mit denselben Fragen
konfrontiert worden.

Die Studie nennt Vorteile der Inklusion für Kinder mit Beeinträchtigung.
Dabei werden Perspektiven der Eltern von Kindern mit oder ohne
speziellen Bedürfnissen berücksichtigt, sowie Ansicht der Anbieter. Im
Anschluss werden fünf Vorteile aus der Studie beschrieben, bei welchen
alle drei Parteien eine hohe Übereinstimmung erzielten. Danach werden
mögliche Risiken aufgezählt.

Vorteile für Kinder mit einer Beeinträchtigung im inklusiven Setting
nach Rafferty \& Griffin
(\protect\hyperlink{ref-raffertyux5cux26Griffin2005}{2005, S. 182}):

\begin{itemize}
\tightlist
\item
  Alle drei Parteien vertreten die Sichtweise, dass das inklusive
  Setting eine positive Auswirkung auf Kinder mit Beeinträchtigung hat.
\item
  Die Mehrheit der drei Parteien vertritt die Ansicht, dass Inklusion
  von Kindern mit Beeinträchtigung im Vorschulalter die Akzeptanz in der
  Gesellschaft fördert.
\item
  Zudem stimmen sie zu, dass das inklusive Setting zur Entwicklung ihrer
  Selbständigkeit beiträgt und sie somit angemessen auf die Realität
  vorbereitet werden.
\item
  Die Kinder haben vermehrt die Möglichkeit an unterschiedlichen
  Aktivitäten teilzunehmen.
\item
  Das inklusive Setting ermöglicht Lernen am Modell.
\end{itemize}

NSW Department of Education and Communities
(\protect\hyperlink{ref-centreforeducation2014}{2014}) (NSWDEC) weist
darauf hin, dass umfangreiche Beweise vorliegen, in welchen der Nutzen
der Fremdbetreuung für alle Kinder und vor allem für benachteiligte
Kinder erläutert werden. Sie machen darauf aufmerksam, dass inklusive
Fremdbetreuung oder frühe Bildung, ein Angebot aus vielen sein kann,
welches Kinder mit speziellen Bedürfnissen in der Entwicklung
unterstützt oder fördert.

Als Nächstes werden mögliche Risiken/ Sorgen betreffend Inklusion für
Kinder mit speziellen Bedürfnissen aufgezählt. Es werden drei von sechs
möglichen Risiken aus der Studie von Rafferty \& Griffin
(\protect\hyperlink{ref-raffertyux5cux26Griffin2005}{2005, S. 182})
genannt:

\begin{itemize}
\tightlist
\item
  Lehrpersonen könnten die nötige Qualifikation oder Schulung im Umgang
  mit Kindern mit speziellen Bedürfnissen fehlen.
\item
  Kinder mit Beeinträchtigung könnten weniger spezialisierte Hilfe oder
  individuelle Instruktionen von Lehrpersonen erhalten.
\item
  Kinder mit Beeinträchtigung laufen Gefahr, weniger Physiotherapie oder
  Logopädie zu erhalten.
\end{itemize}

Diese drei Risiken bewerten Eltern von Kindern mit oder ohne
Beeinträchtigung höher als Anbieter selbst. Beim dritten Punkt stellt
sich die Frage, ob die Übersetzung vom englischen ins Deutsche korrekt
erfolgte oder sich die beiden Autoren Rafferty \& Griffin
(\protect\hyperlink{ref-raffertyux5cux26Griffin2005}{2005}) auf den
Kindergarten beziehen. Die Kitas in der Schweiz haben einen
Betreuungsauftrag, jedoch keinen Förderauftrag. Individuelle Therapien
wie zum Beispiel Logopädie oder Physiotherapie müssen ausserhalb der
Kita organisiert werden. Ein Auszug aus den Richtlinien von kibesuisse
(Verband der Kinderbetreuung Schweiz) beschreibt den Auftrag von Kitas
folgendermassen:

\begin{quote}
Kindertagesstätten übernehmen eine zentrale Aufgabe bei der
frühkindlichen Bildung, Betreuung und Erziehung (FBBE), bei der
Vereinbarkeit von Familie und Erwerbstätigkeit oder Ausbildung sowie bei
der sozialen und sprachlichen Inklusion von Kindern. Die Betreuung in
Kindertagesstätten leistet damit einen wesentlichen Beitrag zur
Chancengerechtigkeit. (kibesuisse,
\protect\hyperlink{ref-richtlinienKindertagessstuxe4tten_2020}{2020, S.
7})
\end{quote}

Im Regelkindergarten werden Kinder durch Fachpersonen der Heilpädagogik
in Gruppen oder einzeln unterstützt. Auch der DaZ Unterricht (Deutsch
als Zweitsprache) erfolgt innerhalb des Schulunterrichtes in kleinen
Gruppen. Logopädie kann bei Bedarf in Anspruch genommen werden und
erfolgt über die Dienststelle für Logopädie in der betreffenden Gemeinde
(\protect\hyperlink{ref-logopuxe4dieKantonLuzern}{Stadt Luzern, 2022a}).
Ist ein Bedarf an Psychomotorik-Therapie vorhanden, erfolgt die
Anmeldung wie bei der Logopädie über die zuständige Therapiestelle in
der jeweiligen Gemeinde (\protect\hyperlink{ref-psychomotorik}{Stadt
Luzern, 2022b}).

Besucht ein Kind im Kanton Luzern hingegen den Heilpädagogischen
Kindergarten (separatives Setting), kann die schulinterne Logopädie oder
Musiktherapie in Anspruch genommen werden. Schulextern werden Ergo- und
Physiotherapie angeboten \footnote{Mehr Informationen auf der Homepage
  von www.volksschulbildung.lu.ch unter HPS Luzern, Förderung und
  Beratung, Therapieangebote.}.

Rafferty \& Griffin
(\protect\hyperlink{ref-raffertyux5cux26Griffin2005}{2005}) kamen in
ihrer Studie zum Schluss, dass Anbieter im Vergleich zu den beiden
Elterngruppen sich positiver gegenüber Inklusion ausgesprochen haben.
Dies ist ein erfreuliches Resultat und kann wie bereits in
\cref{sec:haltungsfrage} erläutert, zu den Gelingensfaktoren gezählt
werden. Auf der anderen Seite überrascht es, dass Eltern von Kindern mit
Beeinträchtigung sich nicht positiver gegenüber Inklusion geäussert
haben.

Es lassen sich folgende Gedanken dazu ableiten: Eltern von Kindern mit
Beeinträchtigung sind zurückhaltender oder scheuen die Organisation von
institutioneller Fremdbetreuung, obwohl sie laut Rafferty \& Griffin
(\protect\hyperlink{ref-raffertyux5cux26Griffin2005}{2005}) viele
Chancen im inklusiven Setting sehen. Ein Grund dafür könnte sein, dass
sie sich soweit selbst organisiert haben und eine institutionelle
Fremdbetreuung überflüssig wird. Es könnte aber auch sein, dass sich
Eltern den externen Blicken und Fragen betreffend ihres Kindes entziehen
möchten oder sie eine mögliche Ablehnung seitens der Kita befürchten und
vermeiden möchten. Vielleicht haben Eltern aber auch Mühe mit der
Ablösung und der Abgabe von Verantwortung.

\hypertarget{sec:werintegriert}{%
\section{Welche Kinder werden integriert/inklusiv
betreut?}\label{sec:werintegriert}}

In diesem Kapitel wird der Blick auf die Kindertagesstätten gerichtet
und anhand von Literatur und Studien erläutert, welche Kinder mit
sonderpädagogischem Förderbedarf in Kindertagesstätten anzutreffen sind.
In \cref{sec:schweregradBeeintruxe4chtigung} wurde bereits die
Integrationschancen von Kindern mit unterschiedlichen
Beeinträchtigungsarten und Schweregraden seitens des Kita-Personals
eingeschätzt.

Das NSWDEC (\protect\hyperlink{ref-centreforeducation2014}{2014}) weist
daraufhin, dass es methodische Limitierungen in Studien mit jungen
Kindern mit Behinderungen gibt. Das breite Angebot in der Fremdbetreuung
und der frühen Bildung, sowie die Vielfältigkeit der Auslegung von
Behinderung, erschweren es, eine allgemein gültige Schlussfolgerung
betreffend früher Bildung und Fremdbetreuung von Kindern mit besonderen
Bedürfnissen vorzunehmen.

Der Studienvergleich von Kißgen et al.
(\protect\hyperlink{ref-studienvergleichbayernrheinland}{2021}) hat
ergeben, dass in den untersuchten Kitas die diagnostizierte (drohende)
Behinderungsart “Verhaltensstörung” und “allgemeine
Entwicklungsverzögerung” vorherrschend sind. Sie betreuen deutlich
weniger Kinder mit einer Sinnesbehinderung, chronischer Krankheit,
geistigen Behinderung und Körperbehinderung (siehe
\cref{fig:Diagnostizierte-Behinderungen}). Des Weiteren geben die
befragten Kitas an, Risikokinder (ohne diagnostizierte Behinderung) zu
betreuen. Die Auffälligkeiten dieser Kinder beziehen sich auf die
Bereiche der Sprachentwicklung, des Sozialverhaltens und der
Emotionsregulation. Weiter sind Auffälligkeiten im Bereich der
Wahrnehmung, der Motorik, wie auch in der Kognition und der allgemeinen
Entwicklungsverzögerung gemeldet worden.

\imageWithCaption{files/Diagnostizierte-Behinderungen.png}{Diagnostizierte
Behinderungen im Vorschulalter nach Angabe der befragten Kita-Leitungen
von Wölfl, Wertfein \& Wirts
(\protect\hyperlink{ref-ivo-studie}{2017})}{width=0.8\textwidth}[fig:Diagnostizierte-Behinderungen]

Ein Erklärungsversuch von Kißgen et al.
(\protect\hyperlink{ref-studienvergleichbayernrheinland}{2021})
betreffend vorherrschenden Kindern mit einer Verhaltensauffälligkeit und
allgemeinen Entwicklungsverzögerung ist, dass diese erst nach Aufnahme
in einer Kita erkannt und diagnostiziert werden. Fakt ist, dass diese
Kinder das Personal fordern und einen besonderen Unterstützungsbedarf
aufweisen. Aufgrund der fehlenden Diagnose erhält die Kita aber keine
finanzielle Unterstützung.

Wie bereits in \cref{sec:schweregradBeeintruxe4chtigung} erläutert,
werden Kindern mit einer schweren Beeinträchtigung eine geringe
Integrationschance zugeschrieben. Nichtsdestotrotz sind
Mehrfachbehinderungen in \cref{fig:Diagnostizierte-Behinderungen} an
dritter Stelle zu finden. Wobei daraus nicht abzulesen ist, wie hoch der
Schweregrad der Beeinträchtigung ist und wie es mit dem Mehraufwand der
Kitas steht. Im Evaluationsbericht von Ecoplan
(\protect\hyperlink{ref-ecoplan}{2017}) sind unterschiedliche Ansichten
vorhanden bezüglich zu betreuende Kinder in Kindertagesstätten mit sehr
intensiven Betreuungsaufwand. Einige äussern, dass es nicht das Ziel
einer Kita sei, Kinder mit einer Eins-zu-Eins-Betreuung zu begleiten. Es
fehle an ausgebildetem Kita-Personal und geeigneter Infrastruktur.

Grundsätzlich darf die Art der Beeinträchtigung nicht gleichgesetzt
werden mit Betreuungsaufwand. Es kann zum Beispiel der Fall sein, dass
ein Kind mit einer Mehrfachbehinderung das Kita-Personal weniger
herausfordert als ein Kind mit einer Autismus-Spektrums-Störung. Zudem
spiele der Charakter des Kindes eine grosse Rolle
(\protect\hyperlink{ref-ecoplan}{Ecoplan, 2017}). Andererseits können
innerhalb der Beeinträchtigungsform grosse Unterschiede auftreten.
(NSWDEC, \protect\hyperlink{ref-centreforeducation2014}{2014}) schreiben
diesbezüglich folgendes:

\begin{quote}
For example, there is no such thing as a ‘universal’ experience of Down
syndrome or autism spectrum disorder. This means there can be wide
variance in levels of functioning and additional needs of individual
children even within the same type of diagnosed disability, resulting in
very small comparative sample sizes in evaluations of programs. (NSWDEC,
\protect\hyperlink{ref-centreforeducation2014}{2014, S. 3})
\end{quote}

Als Nächstes wird die Situation betreffend Fremdbetreuung im Kanton
Luzern betrachtet und dessen Angebot (KITAplus-Programm) vorgestellt.

\hypertarget{sec:kantonluzern}{%
\chapter{Kanton Luzern}\label{sec:kantonluzern}}

Eine Mehrzahl an Studien haben laut Anders
(\protect\hyperlink{ref-anders_2013}{2013}) ergeben, dass
institutionelle Fremdbetreuung von Kindern in den ersten drei
Lebensjahren einen Nulleffekt in Bezug auf die soziale-emotionale
Entwicklung nimmt. Hingegen wird ein positiver Effekt auf die
kognitive-sprachliche Entwicklung verzeichnet. Bei einem Besuch einer
frühkindlichen, institutionellen Betreuung und Bildung ab drei Jahren
seien positive Effekte im Bereich der kognitiv-leistungsbezogenen
Entwicklung deutlicher sichtbar. Dabei spiele die Dauer der Nutzung der
frühkindlichen, institutionelle Fremdbetreuung eine grössere Rolle als
die Intensität.

Laut Amberg \& Heller
(\protect\hyperlink{ref-kinderbetreuungluzern}{2018}) besuchten im Jahr
2017 im Kanton Luzern rund 6’866 Kinder ein familienergänzendes Angebot
(Kita, Tagesfamilie, Spielgruppe).

In Luzern wird seit 2009 auf das System der “Betreuungsgutscheine”
gesetzt. Dieses soll in den nächsten Jahren weiterentwickelt werden.
Bischof (\protect\hyperlink{ref-luzernerzeitung}{2021}) schreibt, dass
der Kanton Luzern die Subventionierung der familienergänzenden
Kinderbetreuung in der Stadt Luzern um einen Drittel erhöhen wird.
Familien mit mittlerem und tiefem Einkommen sollen dadurch entlastet
werden. Luzern habe einen Nachholbedarf, da die Kosten der Kita
gestiegen, die Höhe der Betreuungsgutscheine jedoch gleich geblieben
sind.

In der Stadt Luzern fand aufgrund der Initiative der Stiftung Kifa (Kind
und Familie) von 2012 bis 2017 das Pilotprojekt KITAplus statt. Das
KITAplus ist ein gemeinsames Projekt der Stiftung Kifa Schweiz, des
Heilpädagogischen Früherziehungsdienstes des Kantons Luzern (HFD), der
Stadt Luzern, kibesuisse (Verband Kinderbetreuung Schweiz) und der
Pädagogischen Hochschule Luzern.

Im Zentrum dieser Masterarbeit stehen KITAplus-Kinder. Aus diesem Grund
wird das Konzept von KITAplus Luzern in \cref{sec:kitaplus} ausführlich
beschrieben. Dabei werden Zielgruppe, Ablauf, Finanzierung und
Auswertung der Pilotphase von 2012 bis 2017 zusammengefasst. Grundlage
dafür ist der Konzeptbeschrieb von KITAplus Luzern aus dem Jahr 2020
(\protect\hyperlink{ref-konzeptKitaPlus}{KITAplus Luzern, 2020}). In
\cref{sec:fazit} (Ausblick) wird auf die Revision des
Volksschulbildungsgesetz (VBG) und deren Ausführungsbestimmungen
KITAplus im Kanton Luzern, welche ab dem 01.08.2022 in Kraft treten,
eingegangen.

\hypertarget{sec:kitaplus}{%
\section{KITAplus}\label{sec:kitaplus}}

Seit 2018 ist das KITAplus in das Regelangebot des Heilpädagogischen
Früherziehungsdienstes überführt worden. Das erfolgreiche
KITAplus-Modell wird inzwischen ihn ähnlicher Form in den Kantonen
Nidwalden, Uri, St.~Gallen, Basel-Landschaft und in der Stadt Bern
umgesetzt (\protect\hyperlink{ref-stiftungkifa}{Stiftung Kifa Schweiz,
2022}).

\hypertarget{zielgruppe}{%
\subsection{Zielgruppe}\label{zielgruppe}}

Die Zielgruppe von KITAplus sind Kinder mit besonderen Bedürfnissen. Sie
sind in ihrer “Entwicklungs- und Bildungsmöglichkeit” beeinträchtigt
(\protect\hyperlink{ref-konzeptKitaPlus}{KITAplus Luzern, 2020, S. 4}).
Den Kita-Alltag können sie ohne erhöhte Unterstützung nicht bewältigen.
Entscheidend für die Aufnahme in das KITAplus-Programm sind die
Einschätzungen der KITAplus-Mitarbeitenden des Heilpädagogischen
Früherziehungsdienstes Luzern und Sursee-Willisau. Grundlagen dafür sind
Verhaltensbeobachtungen und/ oder Abklärungsberichte.

Folgende Kinder werden im KITAplus berücksichtigt: Kinder mit
Entwicklungsbehinderungen, mit Entwicklungsverzögerungen und mit
gesundheitlichen Beeinträchtigungen (ausführlichere Infos dazu in
KITAplus Luzern (\protect\hyperlink{ref-konzeptKitaPlus}{2020}), S. 5).
Kinder mit einer medizinischen Indikation können nur dann im KITAplus
berücksichtig werden, wenn medizinisch ausgebildetes Fachpersonal in der
Kita vorhanden ist, die z.B. die Sondierung der Nahrung vornehmen.

Das KITAplus setzt bei folgenden Punkten an:

\begin{quote}
KITAplus setzt auf eine enge Begleitung der Mitarbeitenden in den
Kindertagesstätten bei pädagogischen und / oder medizinischen Fragen,
unterstützt die Kita-Leitung bei der Schaffung von Erfolg versprechenden
Rahmenbedingungen und klärt mit allen Beteiligten die Chancen,
Erwartungen und Ängste und unterstützt sie bei der Formulierung von
realistischen Zusammenarbeitszielen. Wichtigste Akteur, neben den
Kindern, Eltern und Kitas, ist der Heilpädagogische Früherziehungsdienst
des Kantons Luzern (HFD), welcher den direkten Kontakt mit den
Beteiligten sicherstellt.
(\protect\hyperlink{ref-konzeptKitaPlus}{KITAplus Luzern, 2020, S. 4})
\end{quote}

Das Ziel ist, Kinder mit speziellen Bedürfnissen in bestehende
Kindertagesstätten zu integrieren. Sie profitieren von der förderlichen
Umgebung und der ihnen angepassten Förderung und Betreuung. Dahinter
steht eine inklusive Grundhaltung
(\protect\hyperlink{ref-konzeptKitaPlus}{KITAplus Luzern, 2020}).

\hypertarget{voraussetzung-einer-kita}{%
\subsection{Voraussetzung einer Kita}\label{voraussetzung-einer-kita}}

Der Schlüsselbegriff dazu lautet “Haltung”. Ist die Kita motiviert und
offen Kinder mit speziellen Bedürfnissen in ihre Institution
aufzunehmen, ist ein grosser Meilenstein erreicht
(\protect\hyperlink{ref-konzeptKitaPlus}{KITAplus Luzern, 2020, S. 10}).
Sie nennen vier weitere Voraussetzungen:

\textbf{Institutionelle Voraussetzungen:}\\
Leitung und Personal stehen hinter dem KITAplus-Ansatz und öffnen ihre
Institution für Kinder mit speziellen Bedürfnissen. Zudem erklären sie
sich bereit, individuelle Lösungsansätze zu überprüfen und flexibel zu
bleiben.

\textbf{Pädagogische Voraussetzungen:}\\
Sie erklären sich dazu bereit regelmässige Rundtischgespräche mit Eltern
und KITAplus-Mitarbeitenden zu führen.

\textbf{Personelle Voraussetzungen:}\\
Regelmässige Austauschrunden müssen eingeplant werden. Zudem erklärt
sich die Leitung bereit, bei Bedarf die personellen Ressourcen zu
überprüfen und allenfalls anzupassen. Es muss eine allgemeine
Bereitschaft vorhanden sein, sich neues Fachwissen aneignen zu wollen.

\textbf{Räumliche Voraussetzungen:}\\
Infrastruktur und Räumlichkeiten entsprechen den individuellen
Voraussetzungen der Kinder oder könnten ihren Bedürfnissen entsprechend
angepasst werden.

Die Kita erklärt sich indirekt dazu bereit ihre gelebten pädagogische
Ansätze kritisch zu betrachten und allfällige Umstrukturierungen
vorzunehmen. Es entsteht ein Mehraufwand, welcher sie in Kauf nehmen
müssen. Gleichzeitig bietet sich der Kita eine Chance ihr Personal
weiterzubilden. Dank dem Coaching der Heilpädagogischen Fachperson wird
der Blick auf das Kind und seine Bedürfnisse geschärft. Sie lernen neue
Handlungsansätze zu entwickeln. Dies führt zu einer Steigerung der
Qualität in der Kita selbst.

\hypertarget{beantragung-von-kitaplus}{%
\subsection{Beantragung von KITAplus}\label{beantragung-von-kitaplus}}

Zwei Ausgangssituationen sind möglich. Das Kind besucht bereits eine
Kindertagesstätte oder eine Anmeldung ist in Planung. Unabhängig davon
lassen sich folgende drei Schritte beschreiben:

\textbf{Abklärungsphase}\\
In dieser Phase übernimmt der Heilpädagogische Früherziehungsdienst die
Koordination. Besucht das Kind bereits eine Kita, sind die
Mitarbeitenden des KITAplus zuständig. Ist ein Kita-Platz erst in
Planung so übernimmt die involvierte Heilpädagogische Fachperson die
Koordinationsfäden.

Die Eltern müssen zudem mit der Beantragung einverstanden und die
Finanzierung der Betreuungskosten geklärt sein. Das Kind sollte zudem
die Aufnahmekriterien des KITAplus und des Heilpädagogischen
Früherziehungsdienstes erfüllen (weitere Infos zur Abklärungsphase sind
in KITAplus Luzern (\protect\hyperlink{ref-konzeptKitaPlus}{2020}), S.
12 zu finden).

Eine Anmeldung erfolgt via Anmeldeformular \footnote{Das Anmeldeformular
  ist auf folgender Seite abrufbar:\\
  \url{https://www.kindertagesstaette-plus.ch/das-projekt/kitaplus-luzern}}.
Zur Anmeldung wird ein Bericht des Früherziehungsdienstes und, falls
vorhanden, ein Arztberichtes beigelegt. Die Abklärungsphase ist
abgeschlossen, sobald die Leitung den Aufnahmeentscheid bekannt gibt.
Danach folgt der nächste Schritt.

\textbf{Umsetzungsphase}\\
Das Kind besucht die Kita an den vereinbarten Tagen. Die Fachpersonen
der Kita werden je nach Bedürfnis und Anliegen von den zuständigen
KITAplus-Mitarbeitenden gecoacht. Zudem leiten sie regelmässige
Standortgespräche mit allen betreffenden Parteien. Müssen die
Infrastruktur oder die Personalressourcen angepasst werden, stehen sie
der Kita mit Rat zur Seite.

\textbf{Abschlussphase}\\
Das Ende der Teilnahme im KITAplus wird von den KITAplus-Mitarbeitenden
geplant und koordiniert. Die Leitung des Heilpädagogischen
Früherziehungsdienstes bestätigt den formalen Abschluss des KITAplus.

\hypertarget{finanzierung}{%
\subsection{Finanzierung}\label{finanzierung}}

Es wird zwischen “ordentlichen Betreuungskosten” in der Kita und den
Kosten der “Inklusion” unterschieden. Beim letzteren zählen Sonder- und
Koordinationskosten, sowie die Vergütung der KITAplus-Mitarbeitenden
dazu. Die Kostenfaktoren müssen bei jedem Kind einzeln betrachtet und im
Voraus abgeklärt werden.

Die “ordentlichen Betreuungskosten” sind von dem vor Ort zu liegenden
Finanzierungssystem abhängig.

Die Personalkosten der KITAplus-Mitarbeitenden werden über das
Regelbudget des Kantons Luzern finanziert.

Die Kita erhält einen Pauschalbetrag von 30 Franken pro Betreuungstag.
Er begleicht den Mehraufwand der Kita, welcher bei der Begleitung eines
Kindes mit besonderen Bedürfnissen entsteht. Die Übernahme des
Koordinationsbeitrags von 30 Franken wird den Gemeinden ans Herzen
gelegt.

Sollten bei einem Kind Sonderkosten anfallen, wie zum Beispiel die
Anschaffung eines Spezialstuhls oder werden zusätzliche
Personalressourcen notwendig, so wird den Gemeinden empfohlen, diese
Mehrkosten zu übernehmen. Eine weitere Möglichkeit ist die Zuziehung
Dritter, wie zum Beispiel die Stiftung Kifa Schweiz.

\hypertarget{sec:pilotphase2017}{%
\subsection{Auswertung Pilotphase}\label{sec:pilotphase2017}}

Die Evaluation der Pilotphase
(\protect\hyperlink{ref-evalPilotphase}{Tanner Merlo, Buholzer \&
Näpflin, 2014}) zeigt, dass Kinder mit einer Beeinträchtigung unabhängig
von ihrer Art und Schwere der Behinderung von Peers akzeptiert werden
und in unterschiedlichen Interaktionen in der Kita eingebunden sind. Sie
profitieren von vielfältigen Interaktionsangeboten und Rollenvorbildern
und können dadurch ihr Sozialverhalten erweitern. Im Pilotprojekt wird
ersichtlich, dass Kinder mit speziellen Bedürfnissen im Bereich der
sozial-emotionalen Kompetenzen grosse Fortschritte erzielen konnten.
Zudem wurde beobachtet, dass Kinder ohne Behinderung ein hohes
Einfühlungsvermögen entwickelten und gegenüber Kindern mit
Beeinträchtigung viel Verständnis aufbringen.

Eltern von Kindern mit speziellen Bedürfnissen berichten, dass ihre
Kinder in vielen Bereichen (sozial, sprachlich und emotional) von einem
Besuch der Kita profitieren. Das KITAplus-Konzept stösst bei Eltern von
Kindern ohne Beeinträchtigung auf grosse Akzeptanz. In den Augen vieler
Eltern gewinne eine Kita, welche Kinder mit speziellen Bedürfnissen
integriere, sogar an Attraktivität.

\hypertarget{weiterfuxfchrung}{%
\subsection{Weiterführung}\label{weiterfuxfchrung}}

Eine mögliche Erweiterung könnte in Richtung Prävention angestrebt
werden. Es wäre erstrebenswert, wenn Heilpädagogische Fachpersonen
regelmässig oder bei Bedarf Kitas besuchen würden. Das Kita-Personal
hätte dadurch die Möglichkeit bei aufkommenden Fragestellungen
betreffend Kindern mit oder ohne Beeinträchtigung Unterstützung zu
erhalten. Durch allgemeine Beobachtungssequenzen durch die
Heilpädagogische Fachperson könnten ruhige, unauffällige Kinder, welche
zum Beispiel ein auffälliges, repetitives Spielverhalten zeigen,
entdeckt werden. Diese Kindergruppe benötigt gezielte Anleitung und
Förderung, damit nächste Entwicklungsschritte angestossen werden können.
Somit würde das KITAplus verstärkt einen präventiven und inklusiven
Charakter annehmen.

\hypertarget{sec:forschungsmethode}{%
\chapter{Methodik}\label{sec:forschungsmethode}}

In diesem Kapitel steht das methodische Vorgehen im Zentrum. In
\cref{sec:forschungsansatz} wird die ausgewählte Forschungsmethode
mithilfe von Literatur begründet und hergeleitet. Danach wird das
Erhebungsinstrument (\cref{sec:erhebungsinstrument}) und dessen Aufbau
(\cref{sec:fragebogen}) vorgestellt. Darin werden unter anderem die
Antwortformate, sowie die Herleitung der einzelnen Fragen erläutert. Das
Vorgehen der Datenerhebung ist in \cref{sec:datenerhebung} geschildert.
Danach folgt die Erläuterung der Datenaufbereitung und die Beschreibung
der verwendeten Methodik zur Datenanalyse
(\cref{sec:datenaufbereitung}). Den Abschluss des Kapitels bildet die
Stichprobe (\cref{sec:stichprobe}).

\hypertarget{sec:forschungsansatz}{%
\section{Forschungsansatz}\label{sec:forschungsansatz}}

Für diese Forschungsstudie ist die schriftliche Befragung in Form von
zwei ausgearbeiteten, sich ergänzenden, standardisierten Fragebögen
gewählt worden. Dies entspricht der quantitativen Erhebungsmethode. In
dieser werden grundsätzlich grössere Datensätze verwendet. Sie werden
mit geeigneten statistischen Methoden ausgewertet, um im Anschluss
nachvollziehbare Schlussfolgerungen formulieren zu können, welche
repräsentativ sind (\protect\hyperlink{ref-hug2020}{Poscheschnik,
Lederer, Perzy \& Hug, 2020}). Der Datensatz von 37 Kindern ist in
dieser Studie zu klein um repräsentative Aussagen zu formulieren.

Laut Koch \& Ellinger (\protect\hyperlink{ref-Koch2015}{2015}) geht es
in der quantitativen empirischen Forschung darum, “Phänomene in
Häufigkeit und Verteilung darzustellen” (S. 41). Die aufgearbeitete
Theorien in \cref{sec:grundlagen} sollen mit den erhobenen Daten
verglichen werden. “Theorien werden hier verstanden als Abstraktionen
der Wirklichkeit, die versuchen, einen Ausschnitt dieser Wirklichkeit
allgemeingültig (über den Einzelfall hinaus) und widerspruchsfrei zu
erklären” (\protect\hyperlink{ref-Koch2015}{Koch \& Ellinger, 2015, S.
42}). Sie weisen auf ein deduktives Vorgehen hin. Dabei werden in
Anlehnung an die Theorie passende Hypothesen formuliert, welche im
Anschluss mithilfe von empirischen Daten überprüft werden. Sie erwähnen
aber auch, dass dies nicht immer möglich ist. Gründe dafür können
unzureichende Theorien sein. In diesem Fall wird eine explorative Phase
eingeschoben. Der Fokus wird dabei nicht auf die Überprüfung der
Hypothesen gerichtet, sondern auf Erstellung oder Erweiterung von
passenden Hypothesen. Im Bereich des KITAplus-Programm sind wenige
Auswertungsberichte vorhanden. In dieser Studie wird auf die Erstellung
von Hypothesen verzichtet. Die Deskription des Datensatzes steht im
Vordergrund. Koch \& Ellinger (\protect\hyperlink{ref-Koch2015}{2015})
bezeichnen dies als Populationsbeschreibung. Leitend sind dabei die
Forschungsfragen.

Die Wahl fiel auf die quantitative Methode, da anhand eines
standardisierten Fragebogens grössere Datenmengen in kurzer Zeit
generiert werden können. Zudem lässt diese Methode die Sammlung von
spezifischen Informationen von unterschiedlichen Kindern zu. Hinzukommt,
dass der Datensatz einen deskriptiven Blick für die Auswertung und somit
für die Beantwortung der aufgestellten Fragestellungen zulässt.

\hypertarget{sec:erhebungsinstrument}{%
\section{Erhebungsinstrument}\label{sec:erhebungsinstrument}}

In diesem Kapitel steht die Herleitung der beiden Fragebögen im
Vordergrund. Pro Kind sind zwei Fragebögen erstellt worden. Einer wird
von den KITAplus-Mitarbeitenden ausgefüllt
(\cref{sec:erhebungsfragebogen}) und einer von den Fachpersonen der
Heilpädagogischen Früherziehung
(\cref{sec:erhebungsfragebogenallgemein}). Die KITAplus-Mitarbeitenden
pflegen engen Kontakt mit Kitas und haben dadurch wertvollen Einblick in
das Kita-Setting. Hinzukommt, dass sie Fragen oder Anliegen welche die
Betreuenden bezüglich des betreffenden Kindes äussern, kennen. Im
Gegensatz zu den Fachpersonen der Heilpädagogischen Früherziehung
erleben sie das Kind in einer Gruppensituation. Die Fachpersonen der
Heilpädagogischen Früherziehung hingegen pflegen einen engen Kontakt mit
der Familie, können die Beweggründe für die Initiierung der
Fremdbetreuung herleiten und kennen das Kind mit seinen Stärken und
Schwächen. Der Einsatz zweier Fragebögen ermöglicht die Erfassung der
“Fremdbetreuungssituation” sowie “Persönliche Angaben” des Kindes.

Raab-Steiner \& Benesch (\protect\hyperlink{ref-raabsteiner}{2015})
sehen im standardisierten Fragebogen “… eines der typischen
Messinstrumente in den empirischen Sozialwissenschaften …” (S. 47). Die
Erstellung eines brauchbaren Fragebogendesigns für die schriftliche
Befragung, in welcher die Probanden den Fragebogen selbständig ausfüllen
müssen, ist laut Poscheschnik et al.
(\protect\hyperlink{ref-hug2020}{2020}) eine Kunst für sich.
Raab-Steiner \& Benesch (\protect\hyperlink{ref-raabsteiner}{2015})
erwähnen, dass die schriftliche Befragung eine hohe Strukturiertheit des
Inhalts erfordert und auf die steuernde Inputs des Herausgebers
verzichtet werden muss. Der Fragebogen sollte einige Punkte erfüllen,
damit er für die Auswertung verwendbare Daten liefert. Der Fragebogen
muss klar und verständlich formuliert sein, sodass keine
Missverständnisse entstehen können. Die Anleitung muss exakt sein und
die gestellten Fragen in eine logische Reihenfolge gebracht werden. Die
vorgegebenen Antworten sollten zudem für die Auswertung dienlich sein
(\protect\hyperlink{ref-hug2020}{Poscheschnik et al., 2020}). Nach
Raab-Steiner \& Benesch (\protect\hyperlink{ref-raabsteiner}{2015}) wird
zwischen vollstandardisierter, teilstandardisierter und
nichtstandardisierter Befragung unterschieden. Für diese Studie wurde
die vollstandardisierte Befragung mit geschlossenen, halboffenen und
zwei offenen Fragen gewählt (weitere Infos dazu in \cref{sec:fragebogen}
und \cref{sec:skalierung}).

Der Vorteil von standardisierten Fragebögen ist, dass sie den
Fachpersonen der Heilpädagogischen Früherziehung sowie den
KITAplus-Mitarbeitenden zur selben Zeit ausgehändigt werden können. Die
Verteilung gestaltet sich als effizient, da der grösste Teil der
Befragten am selben Dienst arbeiten. Die Rücklaufquote der Fragebögen
kann dank eines kleinen Datensatzes von 37 Kindern kontrolliert werden.
Ein Begleitschreiben informiert über den Grund der Befragung und dessen
Ablauf. Ein weiterer grosser Vorteil der Fragebögen ist die hohe
Anonymisierung. Darauf weist auch Poscheschnik et al.
(\protect\hyperlink{ref-hug2020}{2020}) hin. Im Anschluss wird der
Aufbau der Fragebögen sowie die Herleitung der Fragen erläutert.

\hypertarget{sec:fragebogen}{%
\section{Aufbau der Fragebögen}\label{sec:fragebogen}}

Das Fundament der Fragebögen bilden die formulierten Fragestellungen zu
Beginn der Arbeit. Als Erinnerung werden sie noch einmal aufgeführt:

\begin{enumerate}
\def\labelenumi{\arabic{enumi}.}
\item
  Welche Kinder mit sonderpädagogischem Förderbedarf nehmen im Kanton
  Luzern das KITAplus-Programm in Anspruch?
\item
  Aufgrund welcher Anliegen wird das KITAplus im Kanton Luzern
  beantragt?
\item
  Lassen sich Prädikatoren für eine Beantragung von KITAplus
  ausarbeiten?
\end{enumerate}

Die Fragebögen sind für eine bessere Übersicht in zwei Bereiche
aufgeteilt worden: “Personalangaben” und “Angaben zur
Fremdbetreuungssituation”. Damit die Fragebögen den Kindern zugeordnet
werden können, wird vor der Verteilung die Laufnummer und die Initialen
des Kindes in “Personalangaben” vermerkt.

Die Probanden erhalten zusätzlich zum Fragebogen ein
Informationsschreiben (mehr Infos dazu in \cref{sec:datenerhebung}).
Nichtsdestotrotz wird der Fragebogen mit einer kurzen Einleitung und
einem Schlusswort ergänzt (siehe \cref{sec:einleitungfragebogen}). In
\cref{sec:skalierung} werden die Antwortformate erläutert, bevor in
\cref{sec:personalangaben} und \cref{sec:fremdbetreuungangaben} die
Fragestellungen mit den dazugehörigen Gedanken aufgelistet werden. Dabei
wird nicht zwischen Fragebogen für KITAplus-Mitarbeitenden und
Heilpädagogischen Fachpersonen unterschieden. Die erstellten Fragebögen
sind in \cref{sec:erhebungsfragebogen} und
\cref{sec:erhebungsfragebogenallgemein} zu finden.

\hypertarget{sec:einleitungfragebogen}{%
\subsection{Einleitung \& Dankesworte}\label{sec:einleitungfragebogen}}

Wie auch Kallus Wolfgang (\protect\hyperlink{ref-kallus2016}{2016})
erwähnt, werden zu Beginn die Befragten darauf aufmerksam gemacht, den
Fragebogen ohne Rücksprache mit Eltern oder Kita auszufüllen. Am Ende
des Fragebogens wird ein Dank ausgesprochen und auf das Enddatum der
Rücksendung aufmerksam gemacht. Auf diese Notwendigkeit weisen auch
Raab-Steiner \& Benesch (\protect\hyperlink{ref-raabsteiner}{2015}) hin.

\hypertarget{sec:skalierung}{%
\subsection{Antwortformate}\label{sec:skalierung}}

In den Fragebögen sind geschlossene Antwortformate, Mischformen sowohl
auch zwei offene Fragen zu finden. Bei den Mischformen haben die
Befragten die Möglichkeit, passende Antwortalternativen hinzuzufügen.
Mit offenen Fragen wurde sparsam umgegangen, da Raab-Steiner \& Benesch
(\protect\hyperlink{ref-raabsteiner}{2015}) auf deren schwierige und
zeitaufwendige Auswertung hinweist. Bei Erstellung der Fragen und deren
Beantwortung wurde stets die statistische Auswertung im Blick behalten.

Die “Tendez zur Mitte”, wie sie zum Beispiel in Kallus Wolfgang
(\protect\hyperlink{ref-kallus2016}{2016}) erwähnt wird, möchte bei der
Frage “Wie hoch schätzt du den Unterstützungsbedarf des Kindes in der
Kita ein?” vermieden werden. Eine klare Positionierung soll erreicht
werden. Aus diesem Grund wurden vier Antwortmöglichkeiten zur Auswahl
gegeben (kein besonderer Unterstützungsbedarf / leicht / hoch / sehr
hoch). Hingegen ist bei der Frage “Wie schätzt du die Beeinträchtigung
des Kindes ein?” darauf verzichtet worden. Den Probanden stehen drei
Antwortformate zur Verfügung (leicht / mittel / hoch). Diese Skalierung
wird in der Kurzusammenfassung von Zimmermann
(\protect\hyperlink{ref-zimmermannExpertise}{2021, S. 2}) erwähnt.

Als Nächstes stehen die Fragen und deren Herleitung im Vordergrund. Dazu
sind die Fragen aus \cref{sec:erhebungsfragebogen} und
\cref{sec:erhebungsfragebogenallgemein} aufgelistet. Sie werden mit den
dazugehörigen Überlegungen ergänzt. Zuerst werden “Personalangaben”
erschlossen. Danach folgen Fragen zur “Fremdbetreuungssituation”
(\cref{sec:fremdbetreuungangaben}).

\hypertarget{sec:personalangaben}{%
\subsection{Personalangaben}\label{sec:personalangaben}}

\begin{itemize}
\tightlist
\item
  Frage: Alter des Kindes?
\end{itemize}

Wie sieht die Altersverteilung der Kinder aus? Kann ein Durchschnitt
errechnet werden? Sind es eher jüngere oder ältere Kinder welche die
Kita in Anspruch nehmen oder zeichnet sich eine Durchmischung ab?

\begin{itemize}
\tightlist
\item
  Frage: Biologisches Geschlecht?
\end{itemize}

Ist ein biologisches Geschlecht übervertreten oder erweist sich die
Verteilung als ausgeglichen?

\begin{itemize}
\tightlist
\item
  Frage: Welches ist die Erstsprache des Kindes?
\end{itemize}

Sticht eine Sprachgruppe hervor? Zeichnet sich eine Tendenz ab?

\begin{itemize}
\tightlist
\item
  Frage: Hat das Kind Geschwister? Bitte Jahrgang angeben und ob es
  institutionell fremdbetreut wird oder wurde.
\end{itemize}

Geschwister können als Vorbilder dienen und zugleich als interessante*r
Spielpartner*in, welche neue Spielinputs liefern. Geschwister können
aber eine Zusatzbelastung für Eltern bedeuten. Vielleicht kennen Eltern
das Fremdbetreuungsangebot “Kita” dank älteren Geschwistern?

\begin{itemize}
\tightlist
\item
  Frage: Welche Indikationen für Heilpädagogische Früherziehung sind
  gegeben?
\end{itemize}

Die gesammelten Indikationen können miteinander in der Auswertung
verglichen werden. Sticht eine Indikation hervor? Stimmt die Hypothese,
dass Kinder mit schwerer oder mehrfacher Behinderung selten bis gar
nicht in Kindertagesstätten betreut werden?

\hypertarget{sec:fremdbetreuungangaben}{%
\subsection{Angaben zur
Fremdbetreuung}\label{sec:fremdbetreuungangaben}}

\begin{itemize}
\tightlist
\item
  Frage: Wann hat das Kind mit der Kindertagesstätte gestartet?
\end{itemize}

Das Startdatum kann mit dem Datum des bewilligten KITAplus-Programms
verglichen werden. Wie lange dauert es im Schnitt bis ein
KITAplus-Coaching beantragt wird?

\begin{itemize}
\tightlist
\item
  Frage: Ist die Kindertagesstätte vor Aufnahme der HFE besucht worden?
\end{itemize}

Unterfrage: Ja: Wer hat das KITAplus vorgeschlagen/initiiert?

Grundsätzlich müsste der Hauptinitiator die Kita selbst sein. Ist diese
Annahme korrekt?

Unterfrage: Nein: Wer hat den Besuch einer Kindertagesstätte
vorgeschlagen/initiiert?

Kam der Wunsch von den Eltern? Hat die Heilpädagogische Fachperson die
Fremdbetreuung vorgeschlagen oder war es eine andere Instanz? Welche
Tendenz zeichnet sich ab?

\begin{itemize}
\tightlist
\item
  Frage: Aus welchen Gründen wird die Kita besucht?
\end{itemize}

Dazu wurden folgende Überlegungen gemacht: Welche Vorteile wird einem
Kita-Besuch zugeschrieben? Mehrere Antworten können dazu angekreuzt
werden.

\begin{itemize}
\tightlist
\item
  Frage: Welches ist der Hauptgrund für den Kita Besuch?
\end{itemize}

Damit wird das Hauptanliegen für den Kita Besuch herausgefiltert. Im
Wissen darum, dass die Heilpädagogische Fachperson den Fragebogen
ausfüllt.

\begin{itemize}
\tightlist
\item
  Frage: Wie schätzt du die Beeinträchtigung des Kindes ein?
\end{itemize}

Welche Stufe der Beeinträchtigung ist bei KITAplus-Kindern
vorherrschend? Beantragen Kitas das KITAplus auch bei Kindern mit einer
leichten Beeinträchtigung?

\begin{itemize}
\tightlist
\item
  Frage: Wie hoch schätzt du den Unterstützungsbedarf des Kindes in der
  Kita ein?
\end{itemize}

Die Indikation oder die Stufe der Beeinträchtigung sagen nichts darüber
aus, wie hoch der Unterstützungsbedarf in der Kita ausfällt. Es wird von
einer grossen Heterogenität innerhalb der Indikationen und der Stufen
der Beeinträchtigungen ausgegangen. Folgende Frage stellt sich zum
Beispiel: Wird KITAplus auch bei Kindern mit “keinem besonderen
Unterstützungsbedarf” beantragt oder muss der Unterstützungsbedarf
“hoch” oder “sehr hoch” betragen?

Die Frage betreffend Unterstützungsbedarf wird den
KITAplus-Mitarbeitenden sowie den Fachpersonen der Heilpädagogischen
Früherziehung gestellt. Schätzen die Fachpersonen die Kinder gleich ein
oder zeichnen sich Unterschiede ab?

\begin{itemize}
\tightlist
\item
  Frage: Ist oder wird eine Eins-zu-Eins-Betreuung beantragt?
\end{itemize}

Wie oft wird eine Eins-zu-Eins-Betreuung beantragt? Gibt es Kinder bei
welchen der Beeinträchtigungsgrad hoch eingeschätzt wird und der
Unterstützungsbedarf ebenso, aber keine Eins-zu-Eins-Betreuung beantragt
wurde?

\begin{itemize}
\tightlist
\item
  Frage: Welche Herausforderung trifft die Kita im Alltag mit dem
  KITAplus-Kind an?
\end{itemize}

Dies ist eine der zwei offenen Fragen. Mit welchen Herausforderungen
oder Fragen beschäftigen sich Kita-Mitarbeitende? Kämpfen Kitas mit
ähnlichen Fragen und Herausforderungen? Zeichnet sich eine Tendenz ab?
Lassen sich die Antworten kategorisieren?

\begin{itemize}
\tightlist
\item
  Frage: Welches Hauptanliegen wird aktuell verfolgt?
\end{itemize}

Dies ist die zweite und letzte offene Frage. Lässt sich daraus ein Fazit
ziehen? In welchen Bereichen zeichnet sich ein Schwerpunkt ab? Zeichnet
sich überhaupt ein Schwerpunkt ab?

\begin{itemize}
\tightlist
\item
  Frage: Wie wird die Kita finanziert?
\end{itemize}

Mit welchen finanziellen Mitteln werden die Kita-Kosten gedeckt? Lassen
sich anhand der Antworten Rückschlüsse auf den sozioökonomischen Status
der Familien ziehen?

Die Erhebung ist mit dem vorhandenen Datensatz des KITAplus ergänzt
worden. Folgende Daten sind daraus entnommen:

\begin{itemize}
\tightlist
\item
  Initialen Kind
\item
  Geburtsdatum
\item
  Startdatum des KITAplus-Coaching
\item
  Anzahl Tage pro Woche, in welcher die Kita besucht wird
\item
  Name der Kita
\end{itemize}

\hypertarget{sec:datenerhebung}{%
\section{Vorgehen der Datenerhebung}\label{sec:datenerhebung}}

Als Erstes sind die Einverständniserklärungen der Stellenleitungen des
Heilpädagogischen Früherziehungsdienstes Luzern, Sursee-Willisau, sowie
des Visiopädagogischen Dienstes Luzern eingeholt worden
(\cref{sec:einverstuxe4ndniserkluxe4rungen}). Parallel dazu wurden die
Vertraulichkeitserklärungen unterzeichnet
(\cref{sec:einverstuxe4ndniserkluxe4rungen}). Die Stellenleitungen sind
zudem mit einem Informationsschreiben
(\cref{sec:informationsschreibenleitungen}) über die Masterarbeit sowie
deren Inhalt informiert worden.

Erst danach wurden die Fragebögen persönlich abgegeben oder per Post
verschickt. Dies war Ende Januar 2022 der Fall. Bei allen versendeten
Fragebögen wurde ein frankierter Rückumschlag mitgeschickt mit dem Ziel,
eine möglichst hohe Rücklaufquote zu erreichen. Dies empfiehlt auch
Poscheschnik et al. (\protect\hyperlink{ref-hug2020}{2020}). Ein
beigelegtes Schreiben informiert über Inhalt, Zweck und Ablauf der
Studie (\cref{sec:begleitschreibenfragebogen}).

Kurz darauf wurden alle Kindertagesstätten, welche zur Zeit ein Coaching
durch eine KITAplus-Mitarbeitende erhalten, per Mail über die laufende
Studie informiert. Das Informationsschreiben ist in
\cref{sec:informationsschreibenkitas}.

Die Probanden hatten drei Wochen Zeit, den ausgefüllten Fragebogen zu
retournieren. Diese wurden bei Erhalt auf Vollständigkeit überprüft.

\hypertarget{sec:datenaufbereitung}{%
\section{Datenaufbereitung und
Datenauswertung}\label{sec:datenaufbereitung}}

Als Erstes wurden die Daten der Fragebögen codiert und als Eingangsdaten
in einem \texttt{jupyter}-Notebook bereinigt. Ungültige oder fehlende
Angaben wurden speziell gekennzeichnet, damit diese in den Auswertungen
nicht berücksichtigt werden. Die Antworten der offenen Fragen (Welche
Herausforderungen trifft die Kita im Alltag mit dem KITAplus-Kind an?
Welches Hauptanliegen wird aktuell verfolgt?) wurden definierten
Kategorien zugeordnet und im Anschluss codiert.

Die Programmiersprache \texttt{python} wurde schrittweise erlernt
(Übungen von “Computer Science Circle” sind dazu durchgearbeitet worden)
und im Anschluss zusammen mit dem Modul \texttt{matplotlib} für die
Erstellung der Grafiken verwendet. Danach wurde bestimmt, welche Daten
miteinander verglichen werden sollten und welche Darstellungsform dazu
notwendig sein werden (Balken-, Kreisdiagramm) um quantitative,
deskriptive Aussagen zu erhalten.

Einzig für den Vergleich der beiden Datensätze betreffend
Unterstützungsbedarf wurde der Spearman’sche Rangkorrelationskoeffizient
berechnet. Dies geschah mithilfe von aufgearbeiteter Literatur
(\protect\hyperlink{ref-spearmancoeff}{Spearman, 1904}).

Die Erarbeitung der Programmiersprache \texttt{python} ist ein grosses
Unterfangen und nimmt viel Zeit in Anspruch. Bei dieser Vorgehensweise
gibt es folgende Punkte zu bedenken oder zu kritisieren: Es kann nicht
mit einem zusätzlichen Programm bewiesen werden, dass keine Messfehler
unterlaufen sind. Der Code ist für Laien schwer nachvollziehbar. Das
Programmieren basiert auf einer sauberen Datenaufbereitung und
Arbeitsweise sowie logischen Überlegungen.

Während der Auswertung wurde ein Denkfehler im Fragebogen der
Heilpädagogischen Fachpersonen entdeckt. Es handelt sich um folgende
gestellte Frage: “Ist die Kindertagesstätte vor Aufnahme der HFE besucht
worden?” Dazu standen zwei Antwortformate bereit:

\begin{itemize}
\tightlist
\item
  “Ja: Wer hat das KITAplus vorgeschlagen/initiiert?”
\item
  “Nein: Wer hat den Besuch einer Kindertagesstätte
  vorgeschlagen/initiiert?”
\end{itemize}

Wird die Frage verneint, wird nur diese beantwortet. Jedoch hätte auch
die erste Frage beantwortet werden können. Fakt ist, dass beide Fragen
jeweils nur von einem Teil der Befragten ausgefüllt worden sind. Wären
beide Fragen von allen Beteiligten beantwortet worden, hätte das
Kreisdiagramm möglicherweise eine andere Verteilung angegeben. Der
Fragebogen müsste vor einer weiteren Erhebung sorgfältig überarbeitet
werden.

\hypertarget{sec:stichprobe}{%
\section{Stichprobe}\label{sec:stichprobe}}

Insgesamt werden 37 Kinder an der Erhebung Anfangs Februar 2022 erfasst.
Es gilt anzumerken, dass Kinder fortlaufend Kitas verlassen, KITAplus
beendet wird oder neue Kinder aufgenommen werden. Die Erhebung ist eine
Momentaufnahme der Situation.

Die Rücklaufquote der Fragebögen beträgt 100\%. Vier Kinder sind zu
diesem Zeitpunkt noch nicht von der KITAplus-Mitarbeitenden besucht
worden. Grund dafür ist die Corona-Pandemie. Abgemachte Termine mussten
mehrere Male verschoben werden. Bei diesen Kindern wurde die
Heilpädagogische Fachperson befragt. Drei Kinder nehmen keine
Heilpädagogische Früherziehung in Anspruch. Bei einem weiteren Kind
steht ein Früherziehungswechsel an. Die Fragebögen dieser vier Kinder
wurde einzig von den KITAplus-Mitarbeitenden ausgefüllt.

Die Fragebögen sind insgesamt von vier KITAplus-Mitarbeiterinnen
ausgefüllt worden. Davon sind 24 Kinder dem Heilpädagogischen
Früherziehungsdienst Luzern zugeordnet, ein Kind dem Visiopädagogischen
Dienst Luzern und zwölf Kinder dem Heilpädagogischen
Früherziehungsdienst Sursee-Willisau. Diese Informationen sind für die
Erhebung relevant. Eine KITAplus-Mitarbeiterin füllt somit mindestens
vier Fragebögen aus. Es kann davon ausgegangen werden, dass die
Mitarbeiterin die Kinder untereinander unbewusst vergleicht. Dies kann
einen Einfluss auf die Beantwortung der Fragen haben.

An der Erhebung sind 21 Heilpädagogische Früherzieherinnen beteiligt.

Die Stichprobe wird im Anschluss in zwei Unterkapitel aufgeteilt. Als
Erstes werden die Personalangaben (\cref{sec:personalangabenstic})
erläutert, danach die Angaben zur Fremdbetreuung
(\cref{sec:fremdbetreuungsangabenstic}). Dabei werden die erhobenen
Daten zuerst im Fliesstext beschrieben, bevor sie im Anschluss mit
Diagrammen dargestellt werden.

\hypertarget{sec:personalangabenstic}{%
\subsection{Personalangaben}\label{sec:personalangabenstic}}

Der Durchschnitt wird jeweils mit der Einheit \(\mu\) angegeben und die
Gesamtzahl der Kinder mit \(n\).

Das Durchschnittsalter der Kinder liegt bei 3.8 Jahren (siehe
\cref{fig:altersverteilung}). Anfang Februar 2022 sind 67 \% Jungen und
33 \% Mädchen im KITAplus-Programm (siehe
\cref{fig:beeintraechtigungsgradVsMannFrau}). Von 35 \% der Kinder ist
die Erstsprache Deutsch, 11 \% Tigrinya, jeweils 8 \% Italienisch und
Albanisch. 38 \% haben eine andere Erstsprache (genauere Auffächerung
siehe \cref{fig:erstsprache}). Die Geschwisterverteilung zeigt, dass bei
insgesamt 39 \% jüngere Geschwister vorhanden sind (alle jüngeren
Geschwister zusammen gezählt). Von diesen besuchen 28 \% eine Kita.
Werden alle älteren Geschwister addiert ergibt sich ein Prozentsatz von
45 \%. Ein Kita-Besuch wird bei ihnen mit 60 \% angegeben. 24 \% der
KITAplus-Kinder haben keine Geschwister (siehe \cref{fig:geschwister}).

Die Fachpersonen der Heilpädagogischen Früherziehung wurden im
Fragebogen gebeten die Indikationen der Kinder anzugeben. Pro Kind
wurden grösstenteils mehrere Indikationen ausgewählt, dies gilt es bei
der Auswertung zu beachten. Im Kreisdiagramm (\cref{fig:indikationHFE})
wird die Summe der kumulierten Aussagen gezeigt. Darin wird sichtbar,
dass 19 \% der Kinder einen allgemeinen Entwicklungsrückstand, je 14 \%
eine Sprachauffälligkeit und Wahrnehmungsproblematik aufweisen. Danach
folgt die Verhaltensauffälligkeit mit 12 \% und bei 41 \% liegt eine
andere Indikation vor.

\begin{figure}
     \centering
     \begin{subfigure}[b]{0.49\textwidth}
         \centering
         \includegraphics[width=\textwidth]{files/diagrams/Altersverteilung.pdf}
         \caption{Altersverteilung der Kinder ($n=37$, $\mu=3.8$ Jahre)}
         \label{fig:altersverteilung}
     \end{subfigure}
     \hfill
     \begin{subfigure}[b]{0.49\textwidth}
         \centering
         \includegraphics[width=\textwidth]{files/diagrams/Geschwister.pdf}
         \caption{Geschwisterverteilung ($n=33$)}
         \label{fig:geschwister}
     \end{subfigure}
     \caption{Altersverteilung und Geschwisterverteilung.}
\end{figure}

\imageWithCaption{files/diagrams/ErstspracheKind.pdf}{Welches ist die
Erstsprache des Kindes?}{}[fig:erstsprache]
\imageWithCaption{files/diagrams/IndikationHFE.pdf}{Indikation für
Heilpädagogische Früherziehung (Angaben der Heilpäd.
Fachpersonen)}{}[fig:indikationHFE]

\hypertarget{sec:fremdbetreuungsangabenstic}{%
\subsection{Angaben zur
Fremdbetreuung}\label{sec:fremdbetreuungsangabenstic}}

Diese 37 Kinder sind auf 24 verschiedene Kindertagesstätten im Kanton
Luzern verteilt. 68 \% der Beantragungen für KITAplus sind dem
Heilpädagogischen Früherziehungsdienst Luzern zuzuordnen und 32 \% der
Dienststelle Sursee-Willisau.

Im Durchschnitt wird die Kita 1.9 Tage besucht
(\cref{fig:tagesverteilung}). 12 \% der Kinder beanspruchen eine
Eins-zu-Eins-Betreuung (\cref{fig:beeintraechtigung1zu1}). Das
Kreisdiagramm (\cref{fig:KitaVorHFE}) zeigt auf, dass 45 \% der Kinder
(siehe gelbe Farbtöne im Diagramm) die Kita vor Aufnahme
Heilpädagogischer Früherziehung besucht haben. 55 \% hingegen (siehe
blaue Farbtöne im Diagramm) besuchen eine Kindertagesstätte nachdem
Heilpädagogische Früherziehung initiiert worden ist.

Die erste Unterfrage “Wer hat das KITAplus vorgeschlagen/initiiert?”
ergibt, dass das KITAplus mit 56 \% von der Kita selbst initiiert wurde.
Danach wird mit 31 \% Fachpersonen der Heilpädagogischen Früherziehung
angegeben und bei 13 \% KITAplus-Mitarbeitende. Achtung, diese Frage
wurde von denjenigen beantwortet, welche die Hauptfrage (“Ist die
Kindertagesstätte vor Aufnahme der HFE besucht worden?”) mit “Ja”
beantworteten. Die zweite Unterfrage wurde von diejenigen bearbeitet,
welche die Hauptfrage verneinten. Die Frage “Wer hat den Besuch einer
Kindertagesstätte vorgeschlagen/initiiert?” ergibt, dass Fachpersonen
der Heilpädagogischen Früherziehung mit 55 \% als Hauptinitiator der
Kita gilt, gefolgt von den Eltern mit 35 \%. In je 5 \% der Fälle ist
ein*e Sozialarbeiter*in oder eine Familienberatungsstelle beteiligt.

Die Arbeitstätigkeit der Eltern wird mit 36 \% als Hauptgrund für die
Initiierung des Kita-Platzes angegeben (siehe
\cref{fig:hauptgrundfuerKitaBesuch}). Danach folgt das
Aufeinandertreffen von gleichaltrigen Spielpartner*innen mit 21 \% und
an dritter Stelle mit 15 \% die Entlastung der Eltern. Das Diagramm in
\cref{fig:GruendeKitaBesuch} zeigt andere Ergebnisse. In diesem standen
die Gründe für die Kita-Initiierung im Vordergrund. Dabei konnten
mehrere Antworten ausgewählt werden. An erster Stelle mit 18 \% steht
das Treffen mit gleichaltrigen Spielpartner*innen, danach folgt mit 14
\% Wunsch der Eltern und strukturierter Tagesablauf. Die Entlastung der
Eltern erscheint mit 11 \% an fünfter Stelle.

Wird die Finanzierung betrachtet, ergibt die Befragung, dass in 29 \%
der Fälle Betreuungsgutscheine eingesetzt werden. In 24 \% finanzieren
die Eltern die Kita selbst. In 18 \% übernimmt das Sozialamt die Kosten
(für eine weitere Auffächerung siehe \cref{fig:finanzierung}). Eine
Anmerkung zur Legende: Bei den Unterteilungen “Nothilfe, Gemeinde und
Kifa Stiftung” wird davon ausgegangen, dass diese einen bestimmten
Betrag für die Eltern übernehmen. Wie zum Beispiel die Gemeinde die
Koordinationskosten oder die Kifa Stiftung die Kosten für den
zusätzlichen Betreuungsaufwand oder zumindest einen Teil davon. Der
violette Bereich, entspricht der alleinigen Kostenübernahme der Eltern.
Die Grafik muss mit Vorsicht betrachtet werden. Die finanzielle
Beteiligung der Kifa Stiftung wird höher eingeschätzt, als sie in der
Erhebung angegeben ist. Es wird vermutet, dass die Fragestellung von den
KITAplus-Mitarbeitenden unterschiedlich interpretiert wurde. Die Einen
haben sich bei der Beantwortung der Fragestellung einzig auf die
Kita-Kosten bezogen und die anderen haben die Eins-zu-Eins-Betreuung
dazugezählt. Dieselbe Vorsicht gilt bei der Betrachtung des
Gemeindebeitrages.

\imageWithCaption{files/diagrams/Tagesverteilung.pdf}{Anzahl Tage in der
Kita (\(n=37\), \(\mu=1.9\) Tage)}{}[fig:tagesverteilung]

\imageWithCaption{files/diagrams/KitaVorHFE.pdf}{Ist die Kita vor
Aufnahme der HFE besucht worden? - 55\% Nein: Wer hat den Besuch einer
Kita initiiert? - 45\% Ja: Wer hat das KITAplus
initiiert?}{}[fig:KitaVorHFE]

\imageWithCaption{files/diagrams/HGrundfürKitaBesuch.pdf}{Welches ist
der Hauptgrund für den Kita Besuch?}{}[fig:hauptgrundfuerKitaBesuch]

\imageWithCaption{files/diagrams/GründeKitaBesuch.pdf}{Aus welchen
Gründen wird die Kita besucht?}{}[fig:GruendeKitaBesuch]

\imageWithCaption{files/diagrams/FinanzierungKita.pdf}{Wie wird die Kita
finanziert?}{}[fig:finanzierung]

\hypertarget{sec:evaluation}{%
\chapter{Ergebnisse}\label{sec:evaluation}}

In diesem Kapitel werden ausgewählte Daten einander gegenübergestellt
und den drei Fragestellungen zugeordnet. Zuerst werden die entstandenen
Diagramme beschrieben. Dabei wird z.T. auf Grafiken aus der Stichprobe
(\cref{sec:stichprobe}) Bezug genommen. Danach folgt die Beantwortung
der Fragestellung. Die Interpretation der Daten und die Reflexion stehen
im \cref{sec:diskussion} (Diskussion) im Vordergrund. Der Abschluss der
Arbeit bildet der Ausblick (\cref{sec:fazit}).

Es gilt nochmals auf den kleinen Datensatz von 37 Kindern aufmerksam zu
machen. Deswegen sind die Aussagen der Diagramme nicht repräsentativ.
Sie beschreiben lediglich die Momentaufnahme des Datensatzes im Februar
2022.

\hypertarget{sec:1fragestellung}{%
\section{Erste Fragestellung}\label{sec:1fragestellung}}

\textbf{Welche Kinder mit sonderpädagogischem Förderbedarf nehmen im
Kanton Luzern das KITAplus-Programm in Anspruch?}\\
Die Grafik der Altersverteilung in \cref{fig:altersverteilung} zeigt,
dass nur sehr wenige KITAplus-Kinder unter 36 Monate alt sind. Das
Durchschnittsalter beträgt 3.8 Jahre. Wird das Augenmerk auf die
Erstsprache gerichtet (siehe \cref{fig:erstsprache}), wird ersichtlich,
dass 65 \% der KITAplus-Kinder Deutsch als Zweitsprache und 35 \%
Deutsch als Erstsprache sprechen. Mit 11 \% ist Tigrynia die meist
gesprochene Erstsprache. Insgesamt sind dreizehn Sprachen verzeichnet
worden. In einem weiteren Schritt werden Erstsprache (Unterteilung in
“Deutsch als Erstsprache” und “Deutsch als Zweitsprache”) der
Finanzierungsart gegenübergestellt (siehe
\cref{fig:erstspracheVsFinanzierung}).

Der Balken “Deutsch als Erstsprache” zeigt, dass die Kostenübernahme
Eltern mit 46 \% am höchsten ausgeprägt ist. Nicht vertreten sind in
diesem Balken das Asyl- und Flüchtlingswesen und die Nothilfe. Diese
kommen im Balken “Deutsch als Zweitsprache” vor. Betreuungsgutscheine
werden mit 36 \% am häufigsten eingesetzt. Das Asyl- und
Flüchtlingswesen wird mit 16 \% angegeben.

Werden die beiden Balken miteinander verglichen fällt auf, dass die
Gemeinde mit 4 \% bei “Deutsch als Zweitsprache” weniger involviert ist
als bei “Deutsch als Erstsprache” mit 15 \%. Betreuungsgutscheine kommen
bei Familien mit “Deutsch als Zweitsprache” doppelt so viel zum Einsatz
wie bei deutschsprachigen Familien. Zudem werden Kita-Kosten von
deutschsprachigen Familien knapp vier Mal mehr übernommen als von
Familien mit Deutsch als Zweitsprache. Allgemein lässt sich festhalten,
dass Kita-Kosten nur in wenigen Fällen von den Eltern alleine getragen
werden. Die Grafik (\cref{fig:finanzierung}) zeigt, dass 76 \% der
Eltern finanzielle Unterstützung erhalten.

\imageWithCaption{files/diagrams/ErstspracheVsFinanzierung.pdf}{Erstsprache
im Vergleich zur Finanzierungsart
(\(n=37\))}{}[fig:erstspracheVsFinanzierung]

Das KITAplus ist bei knapp der Hälfte der Kinder zeitgleich mit dem
Kita-Start erfolgt (\cref{fig:kitaStartbisKPlus}). Bei sechs Kindern
wird eine Zeitdifferenz von acht Monaten angegeben. Bei einzelnen
Kindern sind jedoch bis zu drei Jahre zu verzeichnen. Der errechnete
Durchschnitt des Kita-Starts bis zur Aufnahme des KITAplus-Coachings
liegt bei 211 Tagen (sieben Monate).

\imageWithCaption{files/diagrams/KitaStartbisKPlus.pdf}{Anzahl
vergangene Tage seit Kita-Start bis KITAplus-Start (\(n=32\),
\(\mu=211.4\) Tage)}{}[fig:kitaStartbisKPlus]

Wie bereits in \cref{sec:personalangabenstic} erwähnt, ist das männliche
Geschlecht mit 67 \% im KITAplus vorherrschend. Das biologische
Geschlecht wird in \cref{fig:beeintraechtigungsgradVsMannFrau} mit dem
eingeschätzten Beeinträchtigungsgrad verglichen. Dabei wird ersichtlich,
dass das biologische Geschlecht “männlich” in allen Bereichen
überdurchschnittlich und homogen vertreten ist, ausser bei der
Einstufung “leicht”, wobei dort nur ein einziges Kind eingestuft wurde.

\begin{figure}
     \centering
     \begin{subfigure}[b]{0.49\textwidth}
         \centering
         \includegraphics[width=\textwidth]{files/diagrams/BeinträchtigungsgradVsMannFrau.pdf}
         \caption{Beeinträchtigungsgrad versus biologisches Geschlecht ($n=33$)}
         \label{fig:beeintraechtigungsgradVsMannFrau}
     \end{subfigure}
     \hfill
     \begin{subfigure}[b]{0.49\textwidth}
         \centering
         \includegraphics[width=\textwidth]{files/diagrams/BeinträchtigungsgradVs1zu1.pdf}
         \caption{Beeinträchtigungsgrad versus 1:1 Betreuung ($n=33$)}
         \label{fig:beeintraechtigung1zu1}
     \end{subfigure}
     \caption{Vergleiche mit dem Beeinträchtigungsgrad.}
\end{figure}

Die Faktoren “Beeinträchtigungsgrad” und “Eins-zu-Eins-Betreuung” sind
in \cref{fig:beeintraechtigung1zu1} gegenüber gestellt. Daraus kann
abgelesen werden, dass Kindern mit Eins-zu-Eins-Betreuung auch ein
“starker” Beeinträchtigungsgrad zugeschrieben wird. Der
Beeinträchtigungsgrad wird im Anschluss einzeln betrachtet. Dabei ist zu
erkennen, dass die Abstufungen mittel und stark vorherrschend sind:

Die Gesamtzahl der Kinder beträgt (\(n=33\)).

\begin{itemize}
\tightlist
\item
  Beeinträchtigungsgrad stark (18 Kinder) = 55 \%
\item
  Beeinträchtigungsgrad mittel (14 Kinder) = 42 \%
\item
  Beeinträchtigungsgrad leicht (1 Kind) = 3 \%
\end{itemize}

Bei der Gegenüberstellung der beiden Faktoren “Eins-zu-Eins-Betreuung”
und “Unterstützungsbedarf” (siehe
\cref{fig:unterstuetzungsbedarfKPlusVs1zu1}) wird ersichtlich, dass
Kindern mit einer Eins-zu-Eins-Betreuung ein sehr hoher
Unterstützungsbedarf zugeschrieben wird. Aber nicht alle Kinder mit
einem sehr hohen Unterstützungsbedarf nehmen eine Eins-zu-Eins-Betreuung
in Anspruch.

\begin{figure}
     \centering
     \begin{subfigure}[b]{0.49\textwidth}
         \centering
         \includegraphics[width=\textwidth]{files/diagrams/UnterstützungsbedarfKPlusVs1zu1.pdf}
         \caption{Unterstützungsbedarf versus 1:1 Betreuung ($n=32$)}
         \label{fig:unterstuetzungsbedarfKPlusVs1zu1}
     \end{subfigure}
     \hfill
     \begin{subfigure}[b]{0.49\textwidth}
         \centering
         \includegraphics[width=\textwidth]{files/diagrams/BeinträchtigungsgradVsSuppKPlus.pdf}
         \caption{Beeinträchtigungsgrad versus Unterstützungsbedarf (Einschätzung der KITAplus-Mitarbeitenden, $n=32$)}
         \label{fig:beeintraechtigungsgradVsSuppKPlus}
     \end{subfigure}
     \caption{Vergleiche mit Unterstützungsbedarf nach KITAplus.}
\end{figure}

Die nächste Grafik entstand aus folgendem Gedankenspiel: Sind Kinder mit
einer starken Beeinträchtigung immer auf eine sehr hohe Unterstützung
seitens des Kita-Personals angewiesen? Dank der Grafik in
\cref{fig:beeintraechtigungsgradVsSuppKPlus} wird ersichtlich, dass der
Unterstützungsbedarf bei Kindern mit einer starken Beeinträchtigung zu
jeweils 50 \% hoch und sehr hoch eingeschätzt wird. Der
Unterstützungsbedarf von Kindern mit mittlerer Beeinträchtigung wird
überwiegend hoch (69 \%) bewertet. Weiter lassen sich folgende Schlüsse
aus der Grafik ableiten. Alle KITAplus-Kinder weisen einen erhöhten
Unterstützungsbedarf auf, da die Abstufung “kein besonderer
Unterstützungsbedarf” nie angekreuzt wurde. Kinder, welche einen
leichten Unterstützungsbedarf benötigen sind in der Minderheit. Der
Unterstützungsbedarf “hoch” sticht mit 56 \% aus der Gesamtwertung
heraus. Anzumerken gilt es hier, dass die Einschätzung des
Unterstützungsbedarfs sowie des Beeinträchtigungsgrades subjektiv sind.
Damit eine klare und objektive Einschätzung vorgenommen werden kann,
müssten die Abstufungen klaren Kriterien unterliegen.

Als Nächstes stellte sich die Frage: “Benötigen Kinder mit einer
bestimmten Indikation mehr Unterstützung als Kinder mit einer anderen
Indikation?” Diese Frage kann aus folgendem Grund nicht beantwortet
werden: Pro Kind sind immer mehrere Indikationen vorhanden. Es wäre
daher nicht korrekt, einer einzelnen Indikation etwas zuzuschreiben ohne
zu wissen, ob dieser Faktor ausschlaggebend für den erhöhten
Unterstützungsbedarf ist oder nicht. Daher wird darauf verzichtet.

Die gleiche Vorsicht gilt bei der Betrachtung des Kreisdiagramms, in
welchem die Indikationen dargestellt sind (\cref{fig:indikationHFE} in
\cref{sec:personalangabenstic}). Pro Kind sind mehrere Indikationen
ausgewählt worden. In der Minderheit sind laut Grafik Kinder mit einer
Sinnesbehinderung, einer Bewegungsstörung oder einer
Entwicklungsgefährdung. Zudem fällt auf, dass der Verdacht und die
bereits vorhandene Diagnose einer Autismus-Spektrums-Störung mit 7 \%
gleich oft genannt wurden. Würden diese Werte zusammengezählt, wäre die
Indikation Autismus-Spektrums-Störung mit 14 \% an zweiter Stelle,
zusammen mit den Indikationen Sprachauffälligkeit und
Wahrnehmungsproblematik. Weiter Gedanken und Interpretationen sind in
\cref{sec:diskussion} zu finden.

Wie bereits in \cref{sec:fremdbetreuungsangabenstic} erwähnt, wird die
Kindertagesstätte im Durchschnitt 1.9 Tage pro Woche besucht.
Kindertagesstätten empfehlen oftmals einen minimalen Aufenthalt von
einem ganzen Tag oder zwei Halbtagen. Es stellt sich folgende Frage:
“Wird die Kita von Kindern mit sehr hohem Unterstützungsbedarf länger
besucht (Anzahl Tage in der Woche) als von Kindern mit leichtem oder
hohem Unterstützungsbedarf?” Dazu wurden folgende Mittelwerte \(\mu\)
berechnet:

\begin{itemize}
\tightlist
\item
  Kinder mit sehr hohem Unterstützungsbedarf (\(n = 9\)) besuchen die
  Kita \(\mu\) = 2.3 Tage pro Woche
\item
  Kinder mit hohem Unterstützungsbedarf (\(n = 18\)) besuchen die Kita
  \(\mu\) = 2.2 Tage pro Woche
\item
  Kinder mit leichtem Unterstützungsbedarf (\(n = 5\)) besuchen die Kita
  \(\mu\) = 1.8 Tage pro Woche
\end{itemize}

Die Differenz von hohem zu sehr hohem Unterstützungsbedarf ist mit 0.1
Tage minimal. Trotz allem spiegelt sich eine leichte Steigerung von
leichtem bis sehr hohem Unterstützungsbedarf auch in der Anzahl Tage
nieder.

Der Unterstützungsbedarf wurde von Heilpädagogischen Fachpersonen und
KITAplus-Mitarbeitenden eingeschätzt. In
\cref{fig:unterstuetzungsbedarfHFEVsKPlus} werden die Antworten
miteinander verglichen. Daraus resultiert, dass die höchste
Übereinstimmung im Unterstützungsbedarf “hoch” erzielt wurde. Zudem
schätzen die KITAplus-Mitarbeitenden den Unterstützungsbedarf der Kinder
leicht höher ein als die Heilpädagogischen Fachpersonen. Der
Spearman’sche Rangkorrelationskoeffizient
(\protect\hyperlink{ref-spearmancoeff}{Spearman, 1904}) ergibt
\(\rho = 0.21\), was einer schwach positiven Korrelation entspricht.

\imageWithCaption{files/diagrams/UnterstützungsbedarfHFEVsKPlus.pdf}{Unterstützungsbedarf:
Vergleich der Werte von KITAplus und Heilpäd. Fachpersonen
(\(n=28\))}{}[fig:unterstuetzungsbedarfHFEVsKPlus]

Die Fragestellung zu Beginn kann nicht innerhalb von ein oder zwei
Sätzen beantwortet werden. Sie fällt so heterogen aus wie die Kinder,
welche das KITAplus beanspruchen. Jedes Kind bringt seine individuelle
Ausgangslage mit, wie auch die Kindertagesstätte selbst, in welcher sie
betreut werden. Drei Gemeinsamkeiten verbindet sie alle: Es sind Kinder
mit sonderpädagogischem Förderbedarf, welche ein institutionelles
Fremdbetreuungsangebot im Kanton Luzern in Anspruch nehmen und Fragen
bei den jeweiligen Fachpersonen betreffend Unterstützung auslösen.

Folgende Punkte aus diesem Abschnitt und der Stichprobe
(\cref{sec:stichprobe}) können zur gestellten Fragestellung
zusammenfassend festgehalten werden:

\begin{itemize}
\tightlist
\item
  Das Durchschnittsalter beträgt 3.8 Jahre.
\item
  Die Kita wird im Durchschnitt 1.9 Tage besucht.
\item
  Das biologische Geschlecht “männlich” ist mit 67 \% und das Geschlecht
  “weiblich” mit 33 \% vertreten.
\item
  Kinder mit Deutsch als Zweitsprache sind mit 65 \% und Kinder mit
  Erstsprache Deutsch 35 \% vertreten.
\item
  Beeinträchtigungsgrade “mittel” (42 \%) und “stark” (55 \%) sind
  vorherrschend.
\item
  22 \% der Kinder mit Beeinträchtigung “stark” beanspruchen eine
  Eins-zu-Eins-Betreuung.
\item
  Der Unterstützungsbedarf “hoch” liegt mit 56 \% an erster Stelle.
\item
  75 \% der Familien werden bei der Finanzierung der Kita unterstützt.
\item
  76 \% der Kinder haben Geschwister, 24 \% haben keine Geschwister.
\item
  Bei 36 \% der Kinder wird Arbeitstätigkeit der Eltern als Hauptgrund
  für den Kita-Besuch angegeben, bei 21 \% Aufeinandertreffen von Peers
  und bei 15 \% Entlastung der Eltern.
\end{itemize}

\hypertarget{sec:fragestellung2beantwortung}{%
\section{Zweite Fragestellung}\label{sec:fragestellung2beantwortung}}

\textbf{Aufgrund welcher Anliegen wird das KITAplus im Kanton Luzern
beantragt?}\\
Dazu wurden im Fragebogen zwei offene Fragen gestellt. Die Erste
ergründet die Herausforderungen mit welchen sich die Kita konfrontiert
sieht. Die zweite Frage beschäftigt sich mit den aktuell verfolgten
Hauptanliegen. Die Fragen werden im Anschluss mit Grafiken erläutert,
sowie mit Aussagen aus den Fragebögen ergänzt. Die Fragestellung wird am
Ende des Abschnitts beantwortet.

Es gilt darauf hinzuweisen, dass die Befragten jeweils mehrere Antworten
pro Kind notierten. Die Kita sieht sich demnach mit unterschiedlichen
Herausforderungen oder Anliegen pro Kind konfrontiert. Die angegebenen
Prozentsätze zeigen, wie viele Male die Kategorie insgesamt genannt
wurde.

\hypertarget{herausforderungen-kitas}{%
\subsection{Herausforderungen Kitas}\label{herausforderungen-kitas}}

\imageWithCaption{files/diagrams/HerausforderungenKita.pdf}{Welche
Herausforderungen trifft die Kita im Alltag mit dem KITAplus-Kind an?
(1. offene Fragestellung)}{}[fig:herausforderungenkita]

Aus dem Kreisdiagramm (\cref{fig:herausforderungenkita}) geht hervor,
dass Anliegen betreffend “sozialer Teilhabe” mit 33 \% am meisten
genannt worden sind. Diese beinhalten Aussagen wie zum Beispiel:

\begin{itemize}
\tightlist
\item
  “Sozialverhalten im Freispiel”
\item
  “Begleitung, damit Kind angemessen am sozialem Geschehen teilhaben
  kann”
\item
  “Einbinden in das Gruppengeschehen”
\item
  “Teilnehmen an Ritualen und Abläufen”
\item
  “Kontaktaufnahme mit anderen Kindern erschwert”
\end{itemize}

An zweiter Stelle mit 22 \% wird “Herausforderndes Verhalten” genannt.
Diesbezüglich wurden Aussagen formuliert wie:

\begin{itemize}
\tightlist
\item
  “Einhalten von Regeln”
\item
  “Aufmerksamkeit auf Handlung lenken”
\item
  “weint schnell”
\item
  “sprunghaftes Spiel”
\item
  “lautes Schreien”
\end{itemize}

Danach folgt die Kategorie “Selbstständigkeit” mit 16 \%. Folgende
Aussagen wurden dazu formuliert:

\begin{itemize}
\tightlist
\item
  “Selbstversorgung erschwert (An- und Ausziehen, Toilettengang)”
\item
  “Selbständigkeit: Langsam essen, Mund nicht vollstopfen, ab und zu mit
  Löffel essen”
\item
  “Selbständigkeit”
\end{itemize}

Im Anschluss folgt die Kategorie “Kommunikation” mit 12 \%. Folgende
Aussagen sind dieser Kategorie zugeordnet:

\begin{itemize}
\tightlist
\item
  “Interesse an Wörtern wecken”
\item
  “Sprachentwicklung: Sein Interesse an neuen Wörtern wahrnehmen oder
  unterstützen”
\item
  “Kommunikation”
\end{itemize}

Die Kategorie “Andere” fasst unterschiedliche Herausforderungen der
Kitas zusammen. Sie ist 6 \% genannt worden:

\begin{itemize}
\tightlist
\item
  “Langsame Entwicklung”
\item
  “Zeitaufwand bezüglich Unterstützung und Begleitung”
\item
  “Zeit für Coaching einplanen”
\end{itemize}

Die Kategorie “Körper-/Mehrfachbehinderung” mit 5 \% umfasst die
Anliegen:

\begin{itemize}
\tightlist
\item
  “D. ist schwer mehrfachbehindert: Transport in den 3. Stock ohne Lift
  ist körperlich anstrengend für die Betreuerinnen”
\item
  “Schwerst-mehrfachbehindertes Kind: Liegend/Sitzschale”
\item
  “Mehrfachbehinderungen”
\end{itemize}

Die Kategorie “Selbstregulation” ist mit 4 \% vertreten. Dazu wurde die
Aussage formuliert: “Gefühlsregulation”. Den Abschluss mit 1 \% bildet
die Kategorie “Bedürfnisse erkennen”.

\hypertarget{hauptanliegen-kitas}{%
\subsection{Hauptanliegen Kitas}\label{hauptanliegen-kitas}}

Weiter wird die Frage “Welches Hauptanliegen wird aktuell verfolgt?”
betrachtet. Die Erhebung zeigt, dass das Hauptanliegen “soziale
Teilhabe” mit 36 \% hervorsticht (\cref{fig:hauptanliegenKitaBesuch}).
Folgende Antworten wurden dazu genannt:

\begin{itemize}
\tightlist
\item
  “Einbinden in rituelle Abläufe”
\item
  “Interaktion: Gemeinsames Spiel anregen, begleiten, führen”
\item
  “Daran denken, dass er bei Kreissituation auch teilnimmt”
\item
  “Teilhabe an Aktivitäten”
\item
  “Soziales Spiel”
\item
  “In Spieltätigkeit mit anderen Kindern einbinden/unterstützen +
  begleiten”
\item
  “Interaktion mit anderen Kindern”
\end{itemize}

Die Antworten aus den Fragebögen sind stichwortartig verfasst, deshalb
sollen formulierte Fragen helfen, die Anliegen der Kitas
zusammenzufassen:

\begin{itemize}
\tightlist
\item
  Welche Unterstützung oder Begleitung benötigt ein Kind mit
  sonderpädagogischem Förderbedarf, damit Interaktionen oder
  Spieltätigkeiten mit anderen Kindern möglich werden?
\item
  Wie und wann kann ein Kind mit sonderpädagogischem Förderbedarf in
  Rituale der Kita miteinbezogen werden?
\end{itemize}

An nächster Stelle wird das Hauptanliegen “Strukturierter Tagesablauf”
mit 15 \% genannt. Folgende Antworten sind der Kategorie zugeteilt:

\begin{itemize}
\tightlist
\item
  “Regeln und Strukturen”
\item
  “Strukturierter Tagesablauf (Rituale)”
\item
  “Klarer Rahmen/ Abläufe für Sicherheit und Regulation”
\item
  “Reframen: Wenn er Grenzen nicht einhält, noch 1 Chance geben, dann
  Stillbeschäftigung”
\end{itemize}

Dazu lassen sich folgende Fragen ableiten:

\begin{itemize}
\tightlist
\item
  Welche Rituale helfen den Tag zu strukturieren?
\item
  Welche Regeln gelten in der Kita? Welche Konsequenzen folgen bei einem
  Verstoss?
\end{itemize}

Mit 14 \% wird das Anliegen “Selbständigkeit” angegeben. Dieser
Kategorie sind Themen rund um die Esssituation sowie allgemeine
Selbständigkeit zugeordnet. Folgende Fragen können daraus abgeleitet
werden:

\begin{itemize}
\tightlist
\item
  Mit welchen Massnahmen kann das Kind auf dem Weg zur Selbständigkeit
  unterstützt werden?
\item
  Was kann /darf vom Kind gefordert werden, ohne das es überfordert
  wird?
\end{itemize}

Die Kategorie “Kommunikation” folgt mit 12 \%. Diese Hauptanliegen
wurden formuliert:

\begin{itemize}
\tightlist
\item
  “Möglichkeit haben (geben) Bedürfnisse mitzuteilen (Metacom/Gebärden)”
\item
  “Interaktion/ Kommunikation (Gebärden). Möglichkeiten anbieten
  Bedürfnisse mitzuteilen”
\item
  “herausfinden, welche Kommunikationsmittel einsetzen (Fotos, Pictos,
  Sprache..)”
\item
  “deutsche Sprache”
\end{itemize}

Daraus lässt sich die Fragestellung ableiten:

\begin{itemize}
\tightlist
\item
  Wie können Kinder mit sonderpädagogischem Förderbedarf in der
  Kommunikation ihrer Bedürfnisse unterstützt werden? Wie und was kann
  in der Kita umgesetzt werden?
\end{itemize}

Anschliessend folgt die Kategorie “Selbstregulation” mit 9 \%. Folgende
Frage fasst das Hauptanliegen zusammen:

\begin{itemize}
\tightlist
\item
  Wie können Kita-Mitarbeitende die Kinder in Selbstregulation
  unterstützen? Welche Strategien können sie den Kindern vermitteln?
\end{itemize}

Bei den letzten sechs Kategorien wird auf weitere Ausführungen
verzichtet. Die gebildeten Kategorien sind zugleich die formulierten
Aussagen der Befragten:

\begin{itemize}
\tightlist
\item
  “Spiel- \& Handlungsinputs geben”: 5 \%
\item
  “Lagerung des Kindes”: 3 \%
\item
  “Stärkung Team”: 3 \%
\item
  “Bedürfnisse des Kindes wahrnehmen”: 2 \%
\item
  “Ausdauer erweitern”: 2 \%
\end{itemize}

\imageWithCaption{files/diagrams/HauptanliegenKitaBesuch.pdf}{Welches
Hauptanliegen wird aktuell verfolgt? (2. offene
Fragestellung)}{}[fig:hauptanliegenKitaBesuch]

Die beiden Grafiken (\cref{fig:herausforderungenkita} und
\cref{fig:hauptanliegenKitaBesuch}) zeigen, dass Herausforderungen und
Hauptanliegen der Kindertagesstätten in der Kategorie “soziale Teilhabe”
anzusiedeln sind. Die beiden Kategorien “Selbständigkeit” und
“Kommunikation” folgen in beiden Grafiken an dritter und vierter Stelle.

Die Fragestellung zu Beginn des Abschnitts
(\cref{sec:fragestellung2beantwortung}) kann wie folgt beantwortet
werden: Das KITAplus-Coaching wird vorwiegend bezüglich Anliegen der
sozialen Teilhabe beantragt. In den beiden Bereichen “Kommunikation” und
“Selbständigkeit” ist ein weiterer hoher Coaching-Bedarf vorhanden.
Grundsätzlich wird aus der Erhebung ersichtlich, dass Herausforderungen
und Hauptanliegen je nach Kind variieren. Im KITAplus-Coaching werden
pro Kind unterschiedliche Themen behandelt.

\hypertarget{sec:fragestellung3beantwortung}{%
\section{Dritte Fragestellung}\label{sec:fragestellung3beantwortung}}

\textbf{Lassen sich Prädikatoren für eine Beantragung von KITAplus
ausarbeiten?}\\
Für die Beantwortung der Fragestellung sind die Erkenntnisse der 1.
Fragestellung in \cref{sec:1fragestellung} zentral. In dieser
Masterarbeit lassen sich keine Prädikatoren für die Beantragung von
KITAplus ausarbeiten. Grund dafür ist der kleine Datensatz (\(n=33\)).
Im Anschluss werden Ergebnisse aufgezählt, welche aus der Erhebung im
Februar 2022 hervorstechen und auf eine Tendenz hinweisen.

\begin{itemize}
\tightlist
\item
  Biologisches Geschlecht “männlich”
\item
  Deutsch als Zweitsprache
\item
  Unterstützungsbedarf “hoch” oder “sehr hoch”
\item
  Herausforderung und Hauptanliegen der Kitas bezüglich Kinder mit
  sonderpädagogischem Förderbedarf liegen im Bereich der “sozialen
  Teilhabe”
\end{itemize}

Die Indikation wird bewusst weggelassen, da wie bereits erwähnt, jeweils
mehrere Indikationen pro Kind genannt wurden. Die Beantragung eines
KITAplus-Coaching hängt primär von der Kita ab. Nehmen sie die
besonderen Bedürfnisse des Kindes wahr? Ist ihnen das KITAplus-Programm
bekannt? Ist die Kita gegenüber externer, fachlicher Unterstützung
offen? Hat das KITAplus-Programm freie Plätze?

Grundsätzlich ist es nicht entscheidend welche Kinder das
KITAplus-Programm beanspruchen. Schlussendlich ist es von Bedeutung,
dass Kindertagesstätten die Bedürfnisse der Kinder wahrnehmen und bei
Bedarf Hilfe in Form eines Coaching beanspruchen.

Im anschliessenden Kapitel (\cref{sec:diskussion}), sind weitere
Gedanken und Interpretationen zu finden.

\hypertarget{sec:diskussion}{%
\chapter{Diskussion}\label{sec:diskussion}}

Die Fragestellungen wurden mithilfe der Ergebnisse beantwortet. Als
Nächstes erfolgt die Interpretation und Reflexion ausgewählter Aspekte
wie Indikation, Erstsprache, Unterstützungsbedarf, Finanzierung der Kita
und Beeinträchtigungsgrad. Am Ende des Kapitels wird zudem auf
Verbesserungsmöglichkeiten des Fragebogens eingegangen.

Die Erhebung hat gezeigt, dass pro Kind selten eine Indikation angegeben
wird. Meistens wird auf mehrere verwiesen. Die Situation gestaltet sich
einfacher, sobald eine Diagnose oder ein Syndrom vorhanden ist. Es sind
Zusammenhänge zwischen den Indikationen auszumachen. Die
Wahrscheinlichkeit, dass zum Beispiel ein Kind mit einer
Sprachauffälligkeit längerfristig im Verhalten negativ auffällt, ist
gross. Ein Kind mit einer geistigen Behinderung bzw. Verdacht auf eine
geistige Behinderung wird höchstwahrscheinlich einen homogen
Entwicklungsrückstand aufweisen. Im Frühbereich besteht die Problematik
oder Herausforderung, dass oftmals noch keine Diagnose gestellt werden
kann, da die Kinder schlichtweg zu jung dafür sind. Die Auswertung in
einem Kreisdiagramm wie in \cref{fig:indikationHFE} ist mit Vorsicht zu
geniessen. Eine weitere Möglichkeit wäre ein Netzdiagramm zu erstellen.
Damit könnten Behinderungsprofile von Kindern erschlossen und
miteinander verglichen werden. Vielleicht gilt es aber die grosse
Vielfalt und die Limitierungen in der Erfassung dieser zu akzeptieren.
Auf methodische Limitierung weist bereits NSWDEC
(\protect\hyperlink{ref-centreforeducation2014}{2014}) hin. Was jedoch
verbessert oder klar definiert werden kann sind die einzelnen Begriffe
der Indikationen. Am Heilpädagogischen Dienst Luzern sind keine
Definitionen oder Beschreibungen dazu vorhanden. Wann wird die
Indikation “Allgemeiner Entwicklungsrückstand” und wann
“Entwicklungsverzögerung” verwendet? Dies ist unklar. Ein
Kriterienkatalog oder Definitionen würden Klarheit verschaffen.

Aus der Erhebung geht hervor, dass Kinder mit einem “Allgemeinen
Entwicklungsrückstand” (19 \%) vorherrschend im KITAplus-Programm
vertreten sind. Darauf folgen die Indikationen “Sprachauffälligkeit” und
“Wahrnehmungsproblematik” mit je 14 \% und “Verhaltensauffälligkeit” mit
12 \% (siehe \cref{fig:indikationHFE}).

Die Ergebnisse aus der Erhebung weisen Ähnlichkeiten mit denen von
Kißgen et al.
(\protect\hyperlink{ref-studienvergleichbayernrheinland}{2021}) (siehe
\cref{fig:Diagnostizierte-Behinderungen}) auf. Die Studien verdeutlichen
die unterschiedlich verwendeten Indikationsbezeichnungen. Dies erschwert
den Vergleich der Ergebnisse. Die ersten vier Indikationen der Erhebung
sind auch unter den ersten vier in der Studie von Kißgen et al.
(\protect\hyperlink{ref-studienvergleichbayernrheinland}{2021}) zu
finden. Die Differenzierung zur Indikation Wahrnehmungsproblematik fehlt
in der Studie.

Eine eindeutige Übereinstimmung zeichnet sich in den wenig vertretenen
Indikationen “Sinnesbehinderung” und “geistigen Behinderung” ab. In
ihrer Studie sind zudem Kinder mit einer Mehrfachbehinderung an dritter
Stelle. Diese Indikation fehlt in der Erhebung. Auf den ersten Blick
wird nicht ersichtlich, wie viele Kinder mit einer Mehrfachbehinderung
in Kitas betreut werden. Dazu müssen die Antworten der Fragestellung
“Welche Herausforderung trifft die Kita im Alltag mit dem KITAplus-Kind
an?” betrachtet werden. Dabei stellt sich heraus, dass insgesamt zwei
Kinder mit einer schwer Mehrfachbehinderung und ein Kind mit einer
Mehrfachbehinderung das Fremdbetreuungsangebot in Anspruch nehmen. Damit
die beiden Studien miteinander verglichen werden können, müssten
dieselben Indikationen verwendet werden.

Wird der Beeinträchtigungsgrad und die Eins-zu-Eins-Betreuung
betrachtet, ist erstaunlich, dass lediglich 12 \% der Kinder mit
Beeinträchtigungsgrad stark eine Eins-zu-Eins-Betreuung beanspruchen.
Gründe dafür könnten fehlendes Betreuungspersonal in der Kita oder die
aufwändige Beantragung sein. Die Kita könnte deshalb versuchen ohne
erhöhten Betreuungsschlüssel klarzukommen. Folgen davon könnten sein,
dass dem Kind mit speziellen Bedürfnissen, den Peers und dem
Betreuungspersonal der Kita nicht mehr gerecht werden kann. Daraus
resultieren könnten Unzufriedenheit, Überforderung und starke Belastung
der Betreuenden. Im schlimmsten Fall muss das Kind mit
Mehrfachbehinderung die Kita verlassen. Deshalb sollten die
Anforderungen für eine Eins-zu-Eins-Betreuung heruntergesetzt werden.
Schlussendlich steht die Thematik “Vertrauen” im Vordergrund. Der Kanton
muss gegenüber den Kindertagesstätten Vertrauen aufbringen, dass diese
nur eine Eins-zu-Eins-Betreuung beantragen, wenn diese notwendig ist.
Die KITAplus-Mitarbeitenden könnten in diesen Prozess miteinbezogen
werden. Sie könnten die Situation vor Ort beurteilen und eine Empfehlung
dazu abgeben.

Als Nächstes wird auf den Unterstützungsbedarf eingegangen. Die
Darstellung in \cref{fig:unterstuetzungsbedarfHFEVsKPlus} deutet darauf
hin, dass KITAplus-Mitarbeitende den Unterstützungsbedarf höher
einschätzen als die Heilpädagogischen Fachpersonen. Diese Tendenz könnte
sich dadurch begründen, dass KITAplus-Mitarbeitende die Kinder in einer
Gruppe erleben und die Herausforderungen, mit welchen die Kita
konfrontiert ist, Eins-zu-Eins erfährt. Daraus kann abgeleitet werden,
dass der Unterstützungsbedarf einzelner Kinder im Vordergrund stehen
soll und nicht die jeweilige Beeinträchtigung. Der Betreuungsschlüssel
soll sich daran richten und nicht am Grad der Beeinträchtigung. Es kann
nicht davon ausgegangen werden, dass ein Kind mit einer leichten
Beeinträchtigung einen leichten Unterstützungsbedarf beansprucht.
Schlussendlich soll die Integration gelingen und den Kindern mit und
ohne Beeinträchtigung sowie den Betreuenden in der Kita gerecht werden.

Im Diagramm \cref{fig:KitaVorHFE} wird ersichtlich, dass
Heilpädagogische Fachpersonen die Kita als Chance für ein Kind mit
sonderpädagogischem Förderbedarf betrachten und den Besuch aktiv
vorschlagen. Zudem betrachten sie das KITAplus-Programm als sinnvolle
Massnahme und wertvolle Unterstützung für das betreffende Kind sowie
auch für das Kita-Personal. Diese Interpretationen sind aus dem hohen
Prozentsatz der Grafik \cref{fig:KitaVorHFE} zu entnehmen. 31 \% der
Heilpädagogischen Fachpersonen geben an, das KITAplus initiiert zu
haben. Die Gefahr besteht, dass Kitas sich verpflichtet fühlen, der
Empfehlung der Aufnahme von KITAplus-Coaching, nachzukommen. Folge davon
könnte sein, dass sich das Coaching als zäh erweist, da im Grunde
genommen eine Kita das Coaching nicht wünscht. Die Grafik in
\cref{fig:kitaStartbisKPlus} zeigt, dass bei knapp der Hälfte der Kinder
das KITAplus zeitgleich mit dem Kita-Start initiiert wurde. Entweder
haben die Kitas auf Empfehlung der Heilpädagogischen Fachperson das
Coaching beantragt oder es tauchten bald Fragen auf, zu welchen sie
Unterstützung benötigten.

65 \% der KITAplus-Kinder haben Deutsch als Zweitsprache
(\cref{fig:erstsprache}). Dies ist ein auffällig hoher Prozentsatz.
Zudem sind 64 \% dieser Familien auf finanzielle Unterstützung
angewiesen. Daraus lässt sich folgende Frage ableiten: “Welchen Einfluss
nimmt ein tiefer sozioökonomischer Status auf die Entwicklung eines
Kindes?” Die Frage kann in dieser Arbeit nicht beantwortet werden, sie
wäre Ausgangslage für eine neue Forschungsstudie. Es ist jedoch davon
auszugehen, dass der sozioökonomischer Status einer Familie zu den
Risikofaktoren gezählt werden kann.

Die Erhebung zeigt, dass der grösste Teil der Eltern bei der
Finanzierung der Kita-Kosten auf Unterstützung angewiesen sind. Kitas
sind teuer, für viele Eltern zu teuer. Dies ist tragisch. Der Kanton
könnte diesem entgegenwirken indem er Kitas finanziell unterstützt oder
subventioniert. Eltern wie auch Heilpädagogische Fachpersonen sehen in
der Fremdbetreuung eine grosse Ressource für die Entwicklung des Kindes.
Frühe Förderung und Betreuung sollten mehr Aufmerksamkeit und
Wertschätzung erhalten. Kindertagesstätte sind nicht nur ein
Entlastungsangebot für Eltern, ihnen wird wie in
\cref{fig:hauptgrundfuerKitaBesuch} ersichtlich, vielfältige Aufgaben
zugeschrieben. Es ist ein Ort, in welchem Kinder ausserhalb ihres
vertrauten Umfelds Erfahrungen, Inputs, Regeln und soziale Kontakte
knüpfen können.

Bei einer erneuten Erhebung müssen die Fragebögen überarbeitet und
angepasst werden. Einer der beiden offenen Frage könnte zum Beispiel
weggelassen werden. Die Antworten fielen stichwortartig aus und die
Kategorisierung der Antworten gestaltete sich aufwändig. Mit
vorgegebenen Antworten, welche zur Auswahl stehen würden, könnte diesem
entgegengewirkt werden. Zudem sollte wie bereits erwähnt die Hauptfrage
mit den beiden Unterfragen (Ist die Kindertagesstätte vor Aufnahme der
HFE besucht worden?) angepasst werden. Daraus könnten drei Hauptfragen
entstehen und die Problematik wäre gelöst. Bei der Auswahl der
Finanzierungsart könnte die Kategorie “Gemeinde” mit “Übernahme des
Koordinationsbeitrag” ergänzt werden. Eine weitere Kategorie könnte wie
folgt heissen: “Kifa Stiftung: Übernahme/Unterstützung für zusätzlichen
Betreuungsaufwand”.

Die Arbeit wird mit dem Kapitel Ausblick abgeschlossen.

\hypertarget{sec:fazit}{%
\chapter{Ausblick}\label{sec:fazit}}

Es wäre wünschenswert eine weitere Studie, zum Beispiel mit allen
KITAplus-Kindern in der Schweiz, zu lancieren. Ein grösserer Datensatz
wäre vorhanden und somit könnten signifikante Aussagen formuliert
werden. Die Ergebnisse könnten die Resultate der jetzigen Studie
bestätigen oder widerlegen. Ausserdem könnte ein zweiter Versuch
gestartet werden, Prädikatoren für das KITAplus zu formulieren, welches
in dieser Studie nicht gelang.

Weitere Forschungsprojekte bezüglich KITAplus könnten zum Beispiel in
folgende Richtungen lanciert werden:

\begin{itemize}
\tightlist
\item
  Welche Erkenntnisse nehmen Kita-Mitarbeitende aus dem Coaching mit,
  von denen sie sagen können, dass sie ihr Handeln nachhaltig
  beeinflussen?
\item
  Welche Rolle nehmen Kindertagesstätten in der Früherkennung von
  Entwicklungsauffälligkeiten ein?
\item
  Lassen sich Unterschiede im KITAplus zwischen Stadt und Land
  unterscheiden?
\end{itemize}

Das KITAplus-Programm leistet ohne Diskussion einen wertvollen Beitrag
zur Inklusionsthematik. Dies widerspiegelt sich in den folgenden
positiven Bilanzen:

\begin{itemize}
\item
  Auswertung der Pilotphase
  (\protect\hyperlink{ref-konzeptKitaPlus}{KITAplus Luzern, 2020, S.
  14})
\item
  Aktuell hohe Beteiligung von Kindertagesstätten (siehe
  \cref{sec:fremdbetreuungsangabenstic})
\item
  Verordnung zum Gesetz über die Volksschulbildung (VBV,
  \protect\hyperlink{ref-volksschulbildungsverordnung}{2022, S. 9}),
  welche am 01.08.2022 in Kraft tritt. Es werden unter anderem die
  anfallenden Mehrkosten von KITAplus geregelt. Es handelt sich dabei um
  folgende Paragraphen:

  \begin{itemize}
  \item
    \textbf{§ 1 Abs. 1:} \emph{Als Sonderschulung gelten: a)
    heilpädagogische Früherziehung sowie Betreuung und Förderung von
    Kindern mit Behinderung in familienergänzenden Betreuungsangeboten
    (KITAplus).} (VBV,
    \protect\hyperlink{ref-volksschulbildungsverordnung}{2022, S. 8})
  \item
    \textbf{§ 15 Abs. 1bis:} \emph{Die Fachstelle berät Fachpersonen bei
    der Betreuung und Förderung von Kindern mit Behinderung in
    familienergänzenden Betreuungsangeboten (KITAplus).}
    (\protect\hyperlink{ref-volksschulbildungsverordnung}{Regierungsrat
    Kanton Luzern, 2022, S. 8})
  \item
    \textbf{§ 30b:} \emph{Kanton und Gemeinde bezahlen je hälftig die
    behinderungsbedingten Mehrkosten für die Betreuung von Kindern in
    familienergänzenden Betreuungsangeboten (KITAplus) sowie die Kosten
    der Beratung der Fachpersonen in solchen Betreuungsangeboten. Die
    Aufteilung des Gemeindebeitrages richtet sich nach § 29.}
    (\protect\hyperlink{ref-volksschulbildungsverordnung}{Regierungsrat
    Kanton Luzern, 2022, S. 9})
  \end{itemize}
\end{itemize}

Dank der Erlassung der Gesetzgebung gewinnt das KITAplus an Stellenwert.
Erfreulich ist, dass die Finanzierung der Mehrkosten klar definiert ist.
Dies könnte zur Entlastung der Kitas beitragen. Grundsätzlich kann die
Gesetzgebung als Meilenstein und Startschuss der Inklusionsthematik in
familienergänzenden Betreuungsangeboten im Kanton Luzern betrachtet
werden. Der Kanton Luzern setzt ein klares Zeichen: Auch junge Kinder
mit einer Beeinträchtigung haben ein Recht auf Teilhabe in
institutionellen Fremdbetreuungsangeboten.

Welche Veränderungen bringen die Erneuerungen im VBG und die Überführung
in den Kanton mit sich? Folgende Fragen lassen sich diesbezüglich
formulieren: Welche Kriterien müssen erfüllt sein, damit ein
KITAplus-Coaching bewilligt wird? Welche Dokumente müssen eingereicht
werden, damit die zuständige Person, welche das Coaching bewilligt, ein
vollständiges Bild der Betreuungssituation erhält? Wie viel Einfluss
haben die KITAplus-Mitarbeitenden dabei? Welche Rolle nehmen sie dabei
ein? Das Angebot sollte weiterhin niederschwellig und schnell angeboten
werden können, damit ein Abbruch der Fremdbetreuung verhindert wird.

Für die Zukunft wäre es wünschenswert, wenn institutionelle
Fremdbetreuungsangebote gefördert und ausgebaut würden, sodass allen
Familien der Zugang erleichtert wird. Die Hoffnung besteht, dass die
Politik Kindertagesstätte als Chance der frühen kindlichen Entwicklung
und Bildung anerkennt und den Ausbau tatkräftig mit finanziellen Mitteln
unterstützt.

\hypertarget{literaturverzeichnis}{%
\chapter{Literaturverzeichnis}\label{literaturverzeichnis}}

\hypertarget{refs}{}
\begin{CSLReferences}{1}{0}
\leavevmode\vadjust pre{\hypertarget{ref-albersmittendrin}{}}%
Albers, T. (2011). \emph{Mittendrin statt nur dabei. Inklusion in Krippe
und Kindergarten}. München: Ernst Reinhardt Verlag.

\leavevmode\vadjust pre{\hypertarget{ref-albers_vielfalt_2012}{}}%
Albers, T. (2012). Kinder mit Behinderungen in Krippe und Kita - von der
Integration zur Inklusion. In T. Albers, S. Bree, E. Jung \& S. Seitz
(Hrsg.), \emph{Vielfalt von Anfang an. Inklusion in Krippe und Kita} (S.
51–57). Freiburg im Breisgau: Herder.

\leavevmode\vadjust pre{\hypertarget{ref-kinderbetreuungluzern}{}}%
Amberg, H. \& Heller, R. (2018). Kinderbetreuung im Kanton Luzern –
Betreuungsangebote Vorschulalter. Erhebung 2017 zuhanden der
Dienststelle Soziales und Gesellschaft (DISG) des Kantons Luzern.
\emph{Interface Politikstudien Forschung Beratung}. Verfügbar unter:
\url{https://kinderbetreuung.lu.ch/-/media/Kinderbetreuung/Dokumente/Anbietende/Kinderbetr_Erhebungsbericht_Kinderbetreuung_Kanton_Luzern_2017.pdf?la=de-CH}

\leavevmode\vadjust pre{\hypertarget{ref-anders_2013}{}}%
Anders, Y. (2013). Stichwort: {Auswirkungen} frühkindlicher
institutioneller {Betreuung} und {Bildung}. \emph{Zeitschrift für
Erziehungswissenschaft}, \emph{16}(2), 237–275.
https://doi.org/\href{https://doi.org/10.1007/s11618-013-0357-5}{10.1007/s11618-013-0357-5}

\leavevmode\vadjust pre{\hypertarget{ref-europeanjournal}{}}%
Avramidis, E. \& Norwich, B. (2002). Teachers’ attitudes towards
integration/inclusion: a review of the literature. \emph{European
Journal of Special Needs Education}, \emph{12}(2), 129–147.
https://doi.org/\href{https://doi.org/10.1080/08856250210129056}{10.1080/08856250210129056}

\leavevmode\vadjust pre{\hypertarget{ref-luzernerzeitung}{}}%
Bischof, H. (2021). Stadt Luzern will familienergänzende Kinderbetreuung
finanziell stärker unterstützen. Luzerner Zeitung. Verfügbar unter:
\url{https://www.luzernerzeitung.ch/zentralschweiz/luzern/gutscheinsystem-stadt-luzern-will-familienergaenzende-kinderbetreuung-finanziell-staerker-unterstuetzen-ld.2141026}

\leavevmode\vadjust pre{\hypertarget{ref-bfs}{}}%
Bundesamt für Statistik. (2020). \emph{Vereinbarkeit von Beruf und
Familie in der Schweiz und im europäischen Vergleich 2018}. Neuchâtel.
Verfügbar unter:
\url{https://www.bfs.admin.ch/bfsstatic/dam/assets/14877706/master}

\leavevmode\vadjust pre{\hypertarget{ref-buysse}{}}%
Buysse, V., Goldman, B. D. \& Skinner, M. L. (2002). Setting Effects on
Friendship Formation among Young Children with and without Disabilities.
\emph{Exceptional Children}, \emph{68}(4), 503–517.
https://doi.org/\href{https://doi.org/10.1177/001440290206800406}{10.1177/001440290206800406}

\leavevmode\vadjust pre{\hypertarget{ref-ecoplan}{}}%
Ecoplan. (2017). \emph{Kinder mit Behinderungen in Kindertagesstätten.
Evaluation des Pilotsprojekts.} Bern: Jugendamt der Stadt Bern.

\leavevmode\vadjust pre{\hypertarget{ref-zakirova}{}}%
Engstrand, R. \& Roll-Pettersson, L. (2012). Inclusion of preschool
children with autism in Sweden: Attitudes and perceived efficacy of
preschool teachers. \emph{Journal of Research in Special Educational
Needs}, \emph{14}(3).
https://doi.org/\href{https://doi.org/10.1111/j.1471-3802.2012.01252.x}{10.1111/j.1471-3802.2012.01252.x}

\leavevmode\vadjust pre{\hypertarget{ref-europaischeagentur}{}}%
European Agency for Special Needs and Inclusive Education. (2017).
\emph{Inklusive frühkindliche Bildung und Erziehung: Neue Einblicke und
Instrumente – Zusammenfassender Abschlussbericht.} (M. Kyriazopoulou, P.
Bartolo, E. Björck-Åkesson, C. Giné \& F. Bellour, Hrsg.). Odense,
Dänemark. Verfügbar unter:
\url{https://www.european-agency.org/sites/default/files/iece-summary-de.pdf}

\leavevmode\vadjust pre{\hypertarget{ref-web-selbstreflex}{}}%
European Agency for Special Needs and Inclusive Education. (2022).
Selbstreflexionsbogen. Verfügbar unter:
\url{https://www.european-agency.org/Deutsch/publications}

\leavevmode\vadjust pre{\hypertarget{ref-angebotsmangel}{}}%
Fischer, A., Häfliger, M. \& Pestalozzi, A. (2021). Kinderbetreuung und
Behinderung: Angebotsmangel trotz substanziellem Bedarf. \emph{FORUM},
(104), 52–56.

\leavevmode\vadjust pre{\hypertarget{ref-storytelllingprogram}{}}%
Giagazoglou, P. \& Papadaniil, M. (2018). Effects of a Storytelling
Program with Drama Techniques to Understand and Accept Intellectual
Disability in Students 6 - 7 Years Old. A Pilot Study. \emph{Advances in
Physical Education}, \emph{8}(2), 224–237.
https://doi.org/\href{https://doi.org/10.4236/ape.2018.82020}{10.4236/ape.2018.82020}

\leavevmode\vadjust pre{\hypertarget{ref-unerfullbarevision}{}}%
Graumann, O. (2018). \emph{Inklusion - eine unerfüllbare Vision? Eine
kritische Bestandesaufnahme.} Opladen: Barbara Budrich.

\leavevmode\vadjust pre{\hypertarget{ref-schwerermehrfachbehinderung}{}}%
Gutekunst, A., Schreier, S. \& Sarimski, K. (2012). Kinder mit schwerer
und mehrfacher Behinderung im integrativen Kindergarten. Eine besondere
Herausforderung. \emph{Frühförderung interdisziplinär}, \emph{31}(1),
26–32. München: Reinhardt.
https://doi.org/\href{https://doi.org/10.2378/fi2012.art03d}{10.2378/fi2012.art03d}

\leavevmode\vadjust pre{\hypertarget{ref-inklusionQualituxe4t_Heimlich}{}}%
Heimlich, U. (2015). Inklusion und Qualität. Auf dem Weg zur inklusiven
Kindertageseinrichtung. \emph{Frühförderung interdisziplinär},
\emph{35}(1), 28–39. München: Reinhardt.
https://doi.org/\href{https://doi.org/10.2378/fi2016.art03d}{10.2378/fi2016.art03d}

\leavevmode\vadjust pre{\hypertarget{ref-hinz2002}{}}%
Hinz, A. (2002). Von der Integration zur Inklusion- terminologisches
Spiel oder konzeptionelle Weiterentwicklung? Verfügbar unter:
\url{http://www.jugendsozialarbeit.de/media/raw/hinz_inklusion.pdf}

\leavevmode\vadjust pre{\hypertarget{ref-Hinz2013}{}}%
Hinz, A. (2013). Inklusion – von der Unkenntnis zur Unkenntlichkeit!? -
Kritische Anmerkungen zu einem Jahrzehnt Diskurs über schulische
Inklusion in Deutschland. \emph{Zeitschrift für Inklusion}, (1).
Verfügbar unter:
\url{https://www.inklusion-online.net/index.php/inklusion-online/article/view/26}

\leavevmode\vadjust pre{\hypertarget{ref-kindergartenheute2019}{}}%
Hundegger, V. (2019). Eine Kita für alle - Inklusion im pädagogischen
Alltag. \emph{praxis kompakt}. Freiburg: Herder.

\leavevmode\vadjust pre{\hypertarget{ref-kallus2016}{}}%
Kallus Wolfgang, K. (2016). \emph{Erstellung von Fragebogen.} (2.
Auflage). Wien: facultas.

\leavevmode\vadjust pre{\hypertarget{ref-studienvergleichbayernrheinland}{}}%
Kißgen, R., Wirts, C., Limburg, D., Wertfein, M., Franke, S., Wölfl, J.
et al. (2021). Zur inklusiven Betreuung von Kindern mit (drohender)
Behinderung in Kindertageseinrichtungen in Bayern und im Rheinland. Ein
Studienvergleich. \emph{Frühförderung interdisziplinär}, \emph{40}(2),
64–77. Reinhardt.
https://doi.org/\href{https://doi.org/10.2378/fi2021.art07d}{10.2378/fi2021.art07d}

\leavevmode\vadjust pre{\hypertarget{ref-konzeptKitaPlus}{}}%
KITAplus Luzern. (2020). \emph{Konzept. Familienergänzende
Kinderbetreuung f{ü}r Kinder mit besonderen Bed{ü}rfnissen in Luzerner
Kindertagesst{ä}tten}. Luzern: Stiftung Kind und Familie, Stadt Luzern,
Heilpädagogischen Früherziehungsdienstes HFD, kibesuisse, Pädagogische
Hochschule Luzern.

\leavevmode\vadjust pre{\hypertarget{ref-kleineva}{}}%
Klein, E., Lorenz-Medick, H. \& Bamikol-Veit, H. (2012). \emph{Was
Kinder im Rahmen einer inklusiven Tagesbetreuung benötigen.} Hessen: LAG
Frühe Hilfen. Verfügbar unter:
\url{http://www.fruehe-hilfen-hessen.de/uploads/media/SchriftenreiheLAGNr.2.pdf}

\leavevmode\vadjust pre{\hypertarget{ref-alltagsTheorienUeberInklusion}{}}%
Knauf, H. \& Graffe, S. (2016). Alltagstheorien über Inklusion.
\emph{Frühe Bildung}, \emph{5}(4), 187–197. Göttingen: Hogrefe.
https://doi.org/\href{https://doi.org/10.1026/2191-9186/a000281}{10.1026/2191-9186/a000281}

\leavevmode\vadjust pre{\hypertarget{ref-Koch2015}{}}%
Koch, K. \& Ellinger, S. (2015). \emph{Empirische Forschungsmethoden in
der Heil- und Sonderpädagogik.} Göttingen: Hogrefe. Verfügbar unter:
\url{https://elibrary.hogrefe.com/book/99.110005/9783840922435}

\leavevmode\vadjust pre{\hypertarget{ref-peer-learning}{}}%
Kordulla, A. (2017). \emph{Peer-Learning im Übergang von der Kita in die
Grundschule. Unter besonderer Berücksichtigung der Kinderperspektiven.}
(Band 3). Bad Heilbrunn: Julius Klinkhardt.

\leavevmode\vadjust pre{\hypertarget{ref-lanners}{}}%
Lanners, R. (2018). Inklusion von klein auf. Das Projekt «Inklusive
Frühkindliche Bildung» der European Agency. \emph{Schweizerische
Zeitschrift für Heilpädagogik}, \emph{24 (4)}, 34–37. Bern. Verfügbar
unter:
\href{https://www.szh-csps.ch/z2018-04-05}{www.szh-csps.ch/z2018-04-05}

\leavevmode\vadjust pre{\hypertarget{ref-luders2014}{}}%
Lüders, C. (2014). „{Irgendeinen} {Begriff} braucht es ja….“*.
\emph{Soziale Passagen}, \emph{6}(1), 21–53. Wiesbaden: Springer.
https://doi.org/\href{https://doi.org/10.1007/s12592-014-0164-8}{10.1007/s12592-014-0164-8}

\leavevmode\vadjust pre{\hypertarget{ref-beobachtungsstudie}{}}%
Lütolf, M. \& Schaub, S. (2019). Soziale Teilhabe von Kindern mit
Behinderung in der Kindertagesst{ä}tte. Eine Beobachtungsstudie.
\emph{Fr{ü}hf{ö}rderung interdisziplin{ä}r}, \emph{38}(4), 176–190.
München: Reinhardt.
https://doi.org/\href{https://doi.org/10.2378/fi2019.art24d}{10.2378/fi2019.art24d}

\leavevmode\vadjust pre{\hypertarget{ref-tiki}{}}%
Lütolf, M. \& Schaub, S. (2021). Teilhabe in der Kindertagesstätte
(TiKi) Schlussbericht. Zürich: Institut für Behinderung und
Partizipation (HfH).

\leavevmode\vadjust pre{\hypertarget{ref-miedander}{}}%
Miedander, L. (1997). \emph{Gemeinsame Erziehung behinderter und nicht
behinderter Kinder. Materialien zur pädagogischen Arbeit im
Kindergarten.} München: DJI-Verlag.

\leavevmode\vadjust pre{\hypertarget{ref-centreforeducation2014}{}}%
NSW Department of Education and Communities. (2014). \emph{Children with
disability in inclusive early childhood education and care}. Australia.

\leavevmode\vadjust pre{\hypertarget{ref-hug2020}{}}%
Poscheschnik, G., Lederer, B., Perzy, A. \& Hug, T. (2020).
Datenerhebung und Datenaufbereitung. In T. Hug \& G. Poscheschnik
(Hrsg.), (3. Auflage, S. 125–187). München: UVK.

\leavevmode\vadjust pre{\hypertarget{ref-raabsteiner}{}}%
Raab-Steiner, E. \& Benesch, M. (2015). \emph{Der Fragebogen. Von der
Forschungsidee zur SPSS-Auswertung} (4., aktualisierte und überarbeitete
Aufl.). Wien: facultas.

\leavevmode\vadjust pre{\hypertarget{ref-raffertyux5cux26Griffin2005}{}}%
Rafferty, Y. \& Griffin, K. W. (2005). Benefits and Risks of Reverse
Inclusion for Preschoolers with and without Disabilities: Perspectives
of Parents and Providers. \emph{Journal of Early Intervention},
\emph{27}(3), 173–192.
https://doi.org/\href{https://doi.org/10.1177/105381510502700305}{10.1177/105381510502700305}

\leavevmode\vadjust pre{\hypertarget{ref-volksschulbildungsverordnung}{}}%
Regierungsrat Kanton Luzern. (2022). \emph{Verordnung zum Gesetz über
die Volksschulbildung (Volksschulbildungsverordnung, VBV). Änderung vom
4. Januar 2022.} Verfügbar unter:
\url{https://volksschulbildung.lu.ch/-/media/Volksschulbildung/Dokumente/recht_finanzen/schulrecht/Aenderung_SRL_405_und_409_Beschluss_RR.pdf?la=de-CH}

\leavevmode\vadjust pre{\hypertarget{ref-fachkraftkindinteraktion}{}}%
Reyhing, Y., Frei, D., Burkhardt Bossi, C. \& Perren, S. (2019). Die
Bedeutung situativer Charakteristiken und struktureller
Rahmenbedingungen für die Qualität der unterstützenden
Fachkraft-Kind-Interaktion in Kindertagesstätten. \emph{Zeitschrift für
Pädagogische Psychologie}, \emph{33}(1), 33–47. Göttingen: Hogrefe.
https://doi.org/\href{https://doi.org/10.1024/1010-0652/a000233}{10.1024/1010-0652/a000233}

\leavevmode\vadjust pre{\hypertarget{ref-rohrmann_inklusion_2014}{}}%
Rohrmann, E. (2014). Inklusion? {Inklusion}! \emph{Soziale Passagen},
\emph{6}(1), 161–166. Wiesbaden: Springer.
https://doi.org/\href{https://doi.org/10.1007/s12592-014-0161-y}{10.1007/s12592-014-0161-y}

\leavevmode\vadjust pre{\hypertarget{ref-saalfrank_inklusion_2017}{}}%
Saalfrank, W. T. \& Zierer, K. (2017). \emph{Inklusion}. Paderborn:
Ferdinand Schöningh.

\leavevmode\vadjust pre{\hypertarget{ref-sarimskiBehinderteKinder2011}{}}%
Sarimski, K. (2011). \emph{Behinderte Kinder in inklusiven
Kindertagesstätten}. Stuttgart: Kohlhammer.

\leavevmode\vadjust pre{\hypertarget{ref-dabeiseinIstNichtAllesOderDoch}{}}%
Sarimski, K. (2015). Dabeisein ist nicht alles – oder doch?
\emph{Frühförderung interdisziplinär}, \emph{34}(3), 141–151. München:
Reinhardt.
https://doi.org/\href{https://doi.org/10.2378/fi2015.art18d}{10.2378/fi2015.art18d}

\leavevmode\vadjust pre{\hypertarget{ref-schattenmann_inklusion_2014}{}}%
Schattenmann, E. (2014). \emph{Inklusion und {Bewusstseinsbildung}:
{Die} {Notwendigkeit} bewusstseinsbildender {Maßnahmen} zur
{Verwirklichung} von {Inklusion} in {Deutschland}} (4. Auflage).
Oberhausen: ATHENA.

\leavevmode\vadjust pre{\hypertarget{ref-uno}{}}%
Schweizerische Eidgenossenschaft.Übereinkommen der UNO über die Rechte
von Menschen mit Behinderungen. Verfügbar unter:
\url{https://www.edi.admin.ch/edi/de/home/fachstellen/ebgb/recht/international0/uebereinkommen-der-uno-ueber-die-rechte-von-menschen-mit-behinde.html}

\leavevmode\vadjust pre{\hypertarget{ref-seitzfinnern}{}}%
Seitz, S. \& Finnern, N. (2012). Inklusion in Kindertageseinrichtungen -
eigentlich ganz normal. In nifbe (Hrsg.), (S. 15–25). Freiburg: Herder.

\leavevmode\vadjust pre{\hypertarget{ref-siperstein2007}{}}%
Siperstein, G. N., Parker, R. C., Bardon, J. N. \& Widaman, K. F.
(2007). A National Study of Youth Attitudes toward the Inclusion of
Students with Intellectual Disabilities. \emph{Exceptional Children},
\emph{73}(4), 435–455.
https://doi.org/\href{https://doi.org/10.1177/001440290707300403}{10.1177/001440290707300403}

\leavevmode\vadjust pre{\hypertarget{ref-spearmancoeff}{}}%
Spearman, C. (1904). The Proof and Measurement of Association between
Two Things. \emph{The American Journal of Psychology}, \emph{15}(1),
72–101. University of Illinois Press. Verfügbar unter:
\url{http://www.jstor.org/stable/1412159}

\leavevmode\vadjust pre{\hypertarget{ref-logopuxe4dieKantonLuzern}{}}%
Stadt Luzern. (2022a). \emph{Logopädie Luzern}. Verfügbar unter:
\url{https://www.logopaedieluzern.ch/schulkinder-1}

\leavevmode\vadjust pre{\hypertarget{ref-psychomotorik}{}}%
Stadt Luzern. (2022b). \emph{Psychomotorik Therapiestelle}. Verfügbar
unter: \url{https://www.stadtluzern.ch/dienstleistungeninformation/555}

\leavevmode\vadjust pre{\hypertarget{ref-stamm}{}}%
Stamm, M. (2009). \emph{Frühkindliche Bildung in der Schweiz. Eine
Grundlagenstudie im Auftrag der UNESCO-Kommission Schweiz}. (Universität
Fribourg, Hrsg.). Fribourg: o. V. Verfügbar unter:
\url{https://www.margritstamm.ch/images/Grundlagenstudie.pdf}

\leavevmode\vadjust pre{\hypertarget{ref-stiftungkifa}{}}%
Stiftung Kifa Schweiz. (2022). \emph{KITAplus: Für Kinder mit Besonderen
Bedürfnissen}. Zofingen. Verfügbar unter:
\url{https://www.stiftung-kifa.ch/de/entlastung/kitaplus.html}

\leavevmode\vadjust pre{\hypertarget{ref-evalPilotphase}{}}%
Tanner Merlo, S., Buholzer, A. \& Näpflin, C. (2014). \emph{Evaluation
der Pilotphase von Kita plus - Bericht zuhanden der Stiftung Kind und
Familie KiFa Schweiz}. Nr. 42. Luzern: PH Luzern. Verfügbar unter:
\url{https://kipdf.com/evaluation-der-pilotphase-von-kita-plus_5aaff4641723dd389ca3f985.html}

\leavevmode\vadjust pre{\hypertarget{ref-eineKriseDieKeineSeinDarf}{}}%
Trescher, H. (2018). Inklusion in der Kita. Eine Krise, die keine sein
darf? \emph{Der pädagogische Blick}, \emph{26}(03), 176–187. Frankfurt:
Beltz Juventa.

\leavevmode\vadjust pre{\hypertarget{ref-richtlinienKindertagessstuxe4tten_2020}{}}%
Verband Kinderbetreuung Schweiz. (2020). \emph{Richtlinien für die
Betreuung von Kindern in Kindertagesstätten.} Verfügbar unter:
\url{https://www.kibesuisse.ch/fileadmin/Dateiablage/kibesuisse_Publikationen_Deutsch/2020_kibesuisse_Richtlinien_Kita.pdf}

\leavevmode\vadjust pre{\hypertarget{ref-walker}{}}%
Walker, S. \& Berthelsen, D. (2007). The social participation of young
children with developmental disabilities in inclusive early childhood
programs. \emph{Electronic Journal for Inclusive Education},
\emph{2}(2). Verfügbar unter:
\url{https://www.researchgate.net/publication/27472112_The_social_participation_of_young_children_with_developmental_disabilities_in_inclusive_early_childhood_programs}

\leavevmode\vadjust pre{\hypertarget{ref-einstellungPuxe4dagogischerFachkruxe4ften_Weltzien}{}}%
Weltzien, D. \& Söhnen, S. A. (2020). \emph{Einstellungen pädagogischer
Fachkräfte zur Inklusion: Welchen Einfluss haben individuelle
Erfahrungen und teambezogene Faktoren in Kindertageseinrichtungen?
Sonderauswertungen aus dem Projekt InkluKiT}. Nr. 2. (D. Weltzien \& K.
Fröhlich-Gildhoff, Hrsg.)\emph{Perspektiven der empirischen Kinder- und
Jugenforschung} (Band 6, S. 86–106). Freiburg: FEL.

\leavevmode\vadjust pre{\hypertarget{ref-wiedebusch2015}{}}%
Wiedebusch, S., Lohmann, A. \& Hensen, G. (2015). Chronisch kranke
Kinder in Kindertageseinrichtungen. Eine Befragung pädagogischer
Fachkräfte. \emph{Frühförderung interdisziplinär}, \emph{34}(3),
152–163. München: Ernst Reinhardt Verlag.
https://doi.org/\href{https://doi.org/10.2378/fi2015.art19d}{10.2378/fi2015.art19d}

\leavevmode\vadjust pre{\hypertarget{ref-wocken}{}}%
Wocken, H. (2009). Inklusion \& Integration. Ein Versuch, die
Integration vor der Abwertung und die Inklusion vor Träumereien zu
bewahren. Frankfurt. Verfügbar unter:
\url{https://inklusion20.de/material/inklusion/Inklusion\%20vs\%20Integration_Wocken.pdf}

\leavevmode\vadjust pre{\hypertarget{ref-ivo-studie}{}}%
Wölfl, J., Wertfein, M. \& Wirts, C. (2017). \emph{IVO – Eine Studie zur
Umsetzung von Inklusion als gemeinsame Aufgabe von
Kindertageseinrichtungen und Frühförderung in Bayern.
Kita-Ergebnisbericht}. Nr. 30. (Staatsinstitut für Frühpädagogik (IFP),
Hrsg.). München. Verfügbar unter:
\url{https://www.ifp.bayern.de/imperia/md/content/stmas/ifp/projektbericht_ivo_nr_30.pdf}

\leavevmode\vadjust pre{\hypertarget{ref-worldhealthorganisation}{}}%
World Health Organization. (2005). \emph{Internationale Klassifikation
der Funktionsfähigkeit, Behinderung und Gesundheit (ICF). Deutsche
Übersetzung.} Köln: Deutschen Institut für Medizinische Dokumentation
und Information (DIMDI). Verfügbar unter:
\url{https://www.soziale-initiative.net/wp-content/uploads/2013/09/icf_endfassung-2005-10-01.pdf}

\leavevmode\vadjust pre{\hypertarget{ref-zimmermannExpertise}{}}%
Zimmermann, M. (2021). \emph{Expertise über Kosten und Finanzierung
eines Programms zur inklusiven Vorschulbetreuung von Kindern mit
besonderen Bedürfnissen im Kanton Luzern}. Hochschule Luzern. Verfügbar
unter:
\url{https://www.kindertagesstaette-plus.ch/fileadmin/images/01_Startseite/04_Politik-und-Recht/20210901_KITAplus_Expertise_HSLU.pdf}

\end{CSLReferences}

\listoffigures

\listoftables

\appendix
\lehead{Anhang \thechapter}

\hypertarget{sec:informationsschreibenleitungen}{%
\chapter{Informationsschreiben
Dienststellenleitungen}\label{sec:informationsschreibenleitungen}}

\includePDF{files/Sample.pdf}[1]{1}[width=0.8\textwidth]

\hypertarget{sec:begleitschreibenfragebogen}{%
\chapter{Begleitschreiben für den
Fragebogen}\label{sec:begleitschreibenfragebogen}}

\includePDF{files/Sample.pdf}[1]{1}[width=0.8\textwidth]

\hypertarget{sec:informationsschreibenkitas}{%
\chapter{Informationsschreiben
Kindertagesstätten}\label{sec:informationsschreibenkitas}}

\includePDF{files/Sample.pdf}[1]{1}[width=0.8\textwidth]

\hypertarget{sec:erhebungsfragebogen}{%
\chapter{Erhebungsfragebogen
KITAplus-Mitarbeitenden}\label{sec:erhebungsfragebogen}}

Alle Fragen betreffen die aktuelle Fremdbetreuungssituation. Bitte
beantworte die Fragen aus deiner Sicht (KITAplus Mitarbeiter*in) und mit
deinem Hintergrundwissen, welches du aus der Zusammenarbeit mit der Kita
gewonnen hast.

\hypertarget{personalangaben}{%
\section*{Personalangaben}\label{personalangaben}}

Diesen Fragebogen füllst du für folgendes Kind aus:

\begin{itemize}
\tightlist
\item
  Laufnummer: \xhrule[,fill=2cm,thickness=0.5pt]
\item
  Initialen Kind: \xhrule[,fill=2cm,thickness=0.5pt]
\end{itemize}

\hypertarget{angaben-zur-fremdbetreuung}{%
\section*{Angaben zur Fremdbetreuung}\label{angaben-zur-fremdbetreuung}}

\hypertarget{wie-schuxe4tzt-du-die-beeintruxe4chtigung-des-kindes-ein}{%
\subsection*{Wie schätzt du die Beeinträchtigung des Kindes
ein?}\label{wie-schuxe4tzt-du-die-beeintruxe4chtigung-des-kindes-ein}}

\begin{itemize}
\tightlist
\item[$\square$]
  leicht
\item[$\square$]
  mittel
\item[$\square$]
  stark
\end{itemize}

\hypertarget{wie-hoch-schuxe4tzt-du-den-unterstuxfctzungsbedarf-des-kindes-in-der-kita-ein}{%
\subsection*{Wie hoch schätzt du den Unterstützungsbedarf des Kindes in
der Kita
ein?}\label{wie-hoch-schuxe4tzt-du-den-unterstuxfctzungsbedarf-des-kindes-in-der-kita-ein}}

\begin{itemize}
\tightlist
\item[$\square$]
  kein besonderer Unterstützungsbedarf
\item[$\square$]
  leicht
\item[$\square$]
  hoch
\item[$\square$]
  sehr hoch
\end{itemize}

\hypertarget{ist-oder-wird-eine-11-betreuung-beantragt}{%
\subsection*{Ist oder wird eine 1:1 Betreuung
beantragt?}\label{ist-oder-wird-eine-11-betreuung-beantragt}}

\begin{itemize}
\tightlist
\item[$\square$]
  Ja
\item[$\square$]
  Nein
\end{itemize}

\hypertarget{welche-herausforderung-trifft-die-kita-im-alltag-mit-dem-kitaplus-kind-an}{%
\subsection*{Welche Herausforderung trifft die Kita im Alltag mit dem
KITAplus Kind
an?}\label{welche-herausforderung-trifft-die-kita-im-alltag-mit-dem-kitaplus-kind-an}}

\xhrule[,fill=1.0000\textwidth,thickness=0.5pt]\\
\xhrule[,fill=1.0000\textwidth,thickness=0.5pt]\\
\xhrule[,fill=1.0000\textwidth,thickness=0.5pt]\\
\xhrule[,fill=1.0000\textwidth,thickness=0.5pt]\\
\xhrule[,fill=1.0000\textwidth,thickness=0.5pt]

\hypertarget{welches-hauptanliegen-wird-aktuell-verfolgt}{%
\subsection*{Welches Hauptanliegen wird aktuell
verfolgt?}\label{welches-hauptanliegen-wird-aktuell-verfolgt}}

\xhrule[,fill=1.0000\textwidth,thickness=0.5pt]\\
\xhrule[,fill=1.0000\textwidth,thickness=0.5pt]

\hypertarget{wie-wird-die-kita-finanziert}{%
\subsection*{Wie wird die Kita
finanziert?}\label{wie-wird-die-kita-finanziert}}

\begin{itemize}
\tightlist
\item[$\square$]
  Eltern
\item[$\square$]
  Einsatz von Betreuungsgutscheinen
\item[$\square$]
  Sozialamt
\item[$\square$]
  Asyl- und Flüchtlingswesen
\item[$\square$]
  \xhrule[,fill=8cm,thickness=0.5pt]
\end{itemize}

Du hast es geschafft!

\textbf{Herzlichen Dank, dass du dir Zeit genommen hast. Bitte
retourniere nun den ausgefüllten Fragebogen an mich direkt oder per
beigelegtem Couvert bis spätestens Freitag, 18. Februar.}

\hypertarget{sec:erhebungsfragebogenallgemein}{%
\chapter{Erhebungsfragebogen Heilpädagogische
Fachpersonen}\label{sec:erhebungsfragebogenallgemein}}

Der Fragebogen besteht aus zwei Teilen. Als erstes werden Informationen
zum Kind, der vorhandenen Beeinträchtigung und der Familie erfragt. In
einem zweiten Schritt steht die Fremdbetreuungssituation im Vordergrund.
Bitte fülle den Fragebogen für dich aus, ohne Rücksprache mit den Eltern
zu nehmen.

\hypertarget{personalangaben-1}{%
\section*{Personalangaben}\label{personalangaben-1}}

Diesen Fragebogen füllst du für folgendes Kind aus:

\begin{itemize}
\tightlist
\item
  Laufnummer: \xhrule[,fill=2cm,thickness=0.5pt]
\item
  Initialen Kind: \xhrule[,fill=2cm,thickness=0.5pt]
\end{itemize}

\hypertarget{biologisches-geschlecht}{%
\subsection*{Biologisches Geschlecht?}\label{biologisches-geschlecht}}

\begin{itemize}
\tightlist
\item[$\square$]
  Weiblich
\item[$\square$]
  Männlich
\end{itemize}

\hypertarget{alter-des-kindes}{%
\subsection*{Alter des Kindes?}\label{alter-des-kindes}}

Bitte in Jahr und Monat angeben (z.B. 3;4 Jahre):
\xhrule[,fill=3cm,thickness=0.5pt]

\hypertarget{welches-ist-die-erstsprache-des-kindes}{%
\subsection*{Welches ist die Erstsprache des
Kindes?}\label{welches-ist-die-erstsprache-des-kindes}}

\begin{longtable}[]{@{}
  >{\raggedright\arraybackslash}p{(\columnwidth - 2\tabcolsep) * \real{0.3375}}
  >{\raggedright\arraybackslash}p{(\columnwidth - 2\tabcolsep) * \real{0.4750}}@{}}
\toprule
\endhead
\begin{minipage}[t]{\linewidth}\raggedright
\vspace{0.3cm}

\begin{itemize}
\item[$\square$]
  Deutsch
\item[$\square$]
  Italienisch
\item[$\square$]
  Französisch
\item[$\square$]
  Portugiesisch
\item[$\square$]
  Englisch
\item[$\square$]
  Albanisch
\item[$\square$]
  Serbisch
\item[$\square$]
  Türkisch

  \vspace{0.3cm}
\end{itemize}
\end{minipage} & \begin{minipage}[t]{\linewidth}\raggedright
\vspace{0.3cm}

\begin{itemize}
\tightlist
\item[$\square$]
  Pashto
\item[$\square$]
  Tamil
\item[$\square$]
  Tibetisch
\item[$\square$]
  Tigrinya
\item[$\square$]
  Persisch
\item[$\square$]
  Griechisch
\item[$\square$]
  Andere: \xhrulefill[,fill=3cm,thickness=0.5pt]
\end{itemize}
\end{minipage} \\
\bottomrule
\end{longtable}

\hypertarget{hat-das-kind-geschwister-bitte-jahrgang-angeben-und-ob-es-institutionell-fremdbetreut-wird-oder-wurde.}{%
\subsection*{Hat das Kind Geschwister? Bitte Jahrgang angeben und ob es
institutionell fremdbetreut wird oder
wurde.}\label{hat-das-kind-geschwister-bitte-jahrgang-angeben-und-ob-es-institutionell-fremdbetreut-wird-oder-wurde.}}

\begin{itemize}
\item[$\square$]
  Nein
\item[$\square$]
  Ja:

  \begin{itemize}
  \item
    Jahrgang: \xhrule[,fill=2cm,thickness=0.5pt] Wird oder wurde sie/er
    institutionell fremdbetreut?

    \begin{itemize}
    \tightlist
    \item[$\square$]
      Ja
    \item[$\square$]
      Nein
    \end{itemize}
  \item
    Jahrgang: \xhrule[,fill=2cm,thickness=0.5pt] Wird oder wurde sie/er
    institutionell fremdbetreut?

    \begin{itemize}
    \tightlist
    \item[$\square$]
      Ja
    \item[$\square$]
      Nein
    \end{itemize}
  \item
    Jahrgang: \xhrule[,fill=2cm,thickness=0.5pt] Wird oder wurde sie/er
    institutionell fremdbetreut?

    \begin{itemize}
    \tightlist
    \item[$\square$]
      Ja
    \item[$\square$]
      Nein
    \end{itemize}
  \item
    Jahrgang: \xhrule[,fill=2cm,thickness=0.5pt] Wird oder wurde sie/er
    institutionell fremdbetreut?

    \begin{itemize}
    \tightlist
    \item[$\square$]
      Ja
    \item[$\square$]
      Nein
    \end{itemize}
  \end{itemize}
\end{itemize}

\hypertarget{welche-indikationen-fuxfcr-heilpuxe4dagogische-fruxfcherziehung-sind-gegeben}{%
\subsection*{Welche Indikationen für Heilpädagogische Früherziehung sind
gegeben?}\label{welche-indikationen-fuxfcr-heilpuxe4dagogische-fruxfcherziehung-sind-gegeben}}

\begin{itemize}
\tightlist
\item[$\square$]
  Allgemeiner Entwicklungsrückstand
\item[$\square$]
  Entwicklungsverzögerung
\item[$\square$]
  Entwicklungsgefährdung
\item[$\square$]
  Bewegungsstörung
\item[$\square$]
  Geistige Behinderung
\item[$\square$]
  Sinnesbehinderung (Hör-/Sehbehinderung)
\item[$\square$]
  Sprachauffälligkeit
\item[$\square$]
  Verhaltensauffälligkeit
\item[$\square$]
  Wahrnehmungsproblematik
\item[$\square$]
  Autismus-Spektrum-Störung
\item[$\square$]
  Verdacht ASS
\item[$\square$]
  Syndromale Störung: \xhrulefill[,fill=8cm,thickness=0.5pt]
\item[$\square$]
  Andere: \xhrulefill[,fill=8cm,thickness=0.5pt]
\end{itemize}

\hypertarget{angaben-zur-fremdbetreuung-1}{%
\section*{Angaben zur
Fremdbetreuung}\label{angaben-zur-fremdbetreuung-1}}

\hypertarget{wann-hat-das-kind-mit-der-kindertagesstuxe4tte-gestartet}{%
\subsection*{Wann hat das Kind mit der Kindertagesstätte
gestartet?}\label{wann-hat-das-kind-mit-der-kindertagesstuxe4tte-gestartet}}

Notiere das ungefähre Startdatum: \xhrule[,fill=3cm,thickness=0.5pt]

\hypertarget{ist-die-kindertagesstuxe4tte-vor-aufnahme-der-hfe-besucht-worden}{%
\subsection*{Ist die Kindertagesstätte vor Aufnahme der HFE besucht
worden?}\label{ist-die-kindertagesstuxe4tte-vor-aufnahme-der-hfe-besucht-worden}}

\begin{itemize}
\tightlist
\item[$\square$]
  Ja: Wer hat das KITAplus vorgeschlagen/initiiert?
\end{itemize}

Notiere: \xhrulefill[,fill=8cm,thickness=0.5pt]

\begin{itemize}
\tightlist
\item[$\square$]
  Nein: Wer hat den Besuch einer Kindertagesstätte
  vorgeschlagen/initiiert?
\end{itemize}

Notiere: \xhrulefill[,fill=8cm,thickness=0.5pt]

\hypertarget{wie-hoch-schuxe4tzt-du-den-unterstuxfctzungsbedarf-des-kindes-in-der-kita-ein-1}{%
\subsection*{Wie hoch schätzt du den Unterstützungsbedarf des Kindes in
der Kita
ein?}\label{wie-hoch-schuxe4tzt-du-den-unterstuxfctzungsbedarf-des-kindes-in-der-kita-ein-1}}

\begin{itemize}
\tightlist
\item[$\square$]
  kein besonderer Unterstützungsbedarf
\item[$\square$]
  leicht
\item[$\square$]
  hoch
\item[$\square$]
  sehr hoch
\end{itemize}

\hypertarget{aus-welchen-gruxfcnden-wird-die-kita-besucht}{%
\subsection*{Aus welchen Gründen wird die Kita
besucht?}\label{aus-welchen-gruxfcnden-wird-die-kita-besucht}}

\begin{itemize}
\tightlist
\item[$\square$]
  Die Eltern wünschen dies (1)
\item[$\square$]
  Die Eltern sind arbeitstätig (2)
\item[$\square$]
  Die Eltern sollen entlastet werden (3)
\item[$\square$]
  Das Kind soll auf gleichaltrige Spielpartner*innen treffen (4)
\item[$\square$]
  Das Kind soll in Kontakt kommen mit der Deutschen Sprache (5)
\item[$\square$]
  Das Kind soll neue Spielinputs erhalten (6)
\item[$\square$]
  Das Kind soll einen strukturierten Tagesablauf erhalten (7)
\item[$\square$]
  Das Kind soll auf den Kindergarten vorbereitet werden (8)
\item[$\square$]
  Die Familie und das Kind sollen einen Erstkontakt mit dem
  schweizerischen Bildungssystem erhalten (9)
\end{itemize}

\hypertarget{welches-ist-der-hauptgrund-fuxfcr-den-kita-besuch}{%
\subsection*{Welches ist der Hauptgrund für den Kita
Besuch?}\label{welches-ist-der-hauptgrund-fuxfcr-den-kita-besuch}}

\begin{itemize}
\tightlist
\item
  Notiere \textbf{eine} Nummer der vorherigen genannten Antworten:
  \xhrule[,fill=2cm,thickness=0.5pt]
\end{itemize}

Du hast es geschafft!

\textbf{Herzlichen Dank, dass du dir Zeit genommen hast. Bitte
retourniere nun den ausgefüllten Fragebogen an mich direkt oder per
beigelegtem Couvert bis spätestens Freitag, 18. Februar.}

\hypertarget{sec:einverstuxe4ndniserkluxe4rungen}{%
\chapter{Einverständnis- und
Vertraulichkeitserklärungen}\label{sec:einverstuxe4ndniserkluxe4rungen}}

\includePDF{files/Sample.pdf}[1]{1}[width=0.8\textwidth]

\hypertarget{eigenstuxe4ndigkeitserkluxe4rung}{%
\chapter{Eigenständigkeitserklärung}\label{eigenstuxe4ndigkeitserkluxe4rung}}

\includePDF{files/Sample.pdf}[1]{1}[width=0.8\textwidth]
%


\end{document}
